\chapter{Funktionen}
\label{sec:funktionen}

\section{Prädikate und prädikative Konstituenten}
\label{sec:praedikateundpraedikativekonstituenten}

\subsection{Das Prädikat}
\label{sec:daspraedikat}

\index{Prädikat}

In diesem Kapitel werden einige besondere Formen von sogenannten \textit{Prädikaten} bzw.\ deren Bildung angesprochen.
Es muss also diskutiert werden, was genau ein Prädikat eigentlich sein soll, zumal der Begriff ganz nonchalant bereits in vorherigen Kapiteln benutzt wurde und in fast jedem Buch über Grammatik früher oder später auftaucht.

\index{Subjekt}

Wenn man einfach so vom \textit{Prädikat} spricht, meint man meist das \textit{Satzprädikat} und nicht andere prädikative Konstituenten, die in Abschnitt~\ref{sec:praedikative} besprochen werden.
Der Begriff wird logisch-semantisch traditionell dem Begriff des \textit{Subjekts} gegenübergestellt.
Dabei wird die Struktur einer logischen Aussage als zweigeteilt analysiert.
Das Prädikat wird verstanden als etwas, das eine Aussage über das Subjekt (einen Gegenstand im weitesten Sinn) formuliert.
Definitionen auf Basis solcher Überlegungen sind hier fehl am Platze, da sie viel zu weit in die Semantik und philosophische Logik führen.

Oft wird das finite Verb mit seinen infiniten Ergänzungen als Satzprädikat definiert.
In (\ref{ex:daspraedikat016}) würden also \textit{konnte} und \textit{hören} zusammen das Prädikat bilden.
Leider will man üblicherweise auch \textit{ist schön} in (\ref{ex:daspraedikat018}) als Prädikat klassifizieren, also ein Kopulaverb mit einem prädikativen Adjektiv.
In (\ref{ex:daspraedikat019}) müsste man entscheiden, ob \textit{meint} alleine das Prädikat bildet, oder ob \textit{zu hören} oder sogar \textit{die Sonate zu hören} zum Prädikat gehört.

\begin{exe}
  \ex\label{ex:daspraedikat016}
  \begin{xlist}
    \ex{\label{ex:daspraedikat017} Alma konnte die Sonate hören.}
    \ex{\label{ex:daspraedikat018} Die Johannes-Passion ist schön.}
    \ex{\label{ex:daspraedikat019} Alma meint [die Sonate zu hören].}
  \end{xlist}
\end{exe}

Das Verb \textit{meinen} ist nun aber ohne das Verb im zweiten Status (hier \textit{zu hören}) genauso unvollständig wie Modalverben ohne ein Verb im ersten Status.
Aus Gründen, die in Abschnitt~\ref{sec:modalverbenundaehnliches} besprochen werden, kann \textit{die Sonate zu hören} aber auch eine eigenständige Konstituente bilden.
Das potentielle Prädikat \textit{meint zu hören} würde sich also nicht als Phrase oder Verbkomplex darstellen lassen, sondern bestünde aus einem finiten Verb und Teilen einer anderen Phrase.

\index{Satzglied}

\label{abs:daspraedikat020}Ein vermeintlich besserer Definitionsversuch bezieht sich auf den Satzgliedstatus von Konstituenten.
Das Prädikat bestünde dann aus dem finiten Verb und allen von ihm abhängigen Konstituenten außer den Satzgliedern (vgl.\ dazu Abschnitt~\ref{sec:konstituentenundsatzglieder}).
Ein Satzglied wird üblicherweise als eine Konstituente bezeichnet, die sich eigenständig im Satz bewegen lässt.
Das Deutsche erlaubt es allerdings, dass Teile von Verbkomplexen alleine ins Vorfeld gestellt werden, in (\ref{ex:daspraedikat021}) \zB \textit{kaufen können}.
Damit könnte nach der letztgenannten Definition \textit{kaufen können} nicht Teil des Prädikats sein.
Auch mit dieser Definition ist also niemandem verbindlich geholfen.

\begin{exe}
  \ex{\label{ex:daspraedikat021} [Kaufen können \Ti]\ORii\ möchte\ORi\ Alma die Wolldecke \Tii.}
\end{exe}

Eine exakte Definition dessen, was Prädikate sind, wird wegen der genannten Probleme hier nicht angeboten.
Vielmehr wird der Standpunkt vertreten, dass es sich bei dem Prädikatsbegriff grammatisch gesehen um einen Sammelbegriff handelt, von dem Linguisten ein intuitives Verständnis haben, der aber erst in Zusammenhang mit einer formalisierten Semantik genau definiert werden kann.
Die exakte Einführung eines Begriffes hat nur dann einen Nutzen, wenn eine Generalisierung damit erfasst werden kann.
Wir müssten also grammatische Eigenschaften finden, die im Rahmen der deskriptiven Grammatik allen sogenannten Prädikaten gemein sind.
Dies scheint vergleichsweise schwierig, und für die Fremdsprachenvermittlung oder den Grammatikunterricht an Schulen ist der Prädikatsbegriff schlicht entbehrlich und kann meist durch \textit{finites Verb}, \textit{finites Verb und davon abhängige infinite Verben} usw. ersetzt werden, je nachdem, was gerade gemeint ist.

Einige andere Konstituenten werden auch als Prädikate oder als prädikative Konstituenten beschrieben.
Sie sind vom hier diskutierten Satzprädikat teilweise deutlich verschieden, und deswegen ist ihnen Abschnitt~\ref{sec:praedikative} gewidmet.

\subsection{Prädikative}
\label{sec:praedikative}

\index{Ergänzung!prädikativ}
Ein häufig anzutreffender Begriff, der vom Prädikatsbegriff abgeleitet ist, ist der des \textit{Prädikativums}, der \textit{Prädikativergänzung}, \textit{Prädikativangabe} usw.
Man spricht auch davon, Phrasen seien \textit{prädikativ}.
Im Prinzip werden als prädikativ gerne die Elemente definiert, die Teil des Prädikats sind, oder die ein eigenes Prädikat bilden.
Der Begriff ist damit grammatisch so heterogen wie der Begriff des Prädikats selbst -- und im Kern semantisch.

\index{Prädikatsnomen}
\index{Kopula}

Als \textit{Prädikatsnomen} bzw. \textit{prädikative PP} usw.\ bei Kopulaverben werden die eingeklammerten Konstituenten in (\ref{ex:praedikative022}) bezeichnet.
Sie stellen den Prototyp des Prädikativums bzw.\ der Prädikativergänzung dar.

\begin{exe}
  \ex\label{ex:praedikative022}
  \begin{xlist}
    \ex{\label{ex:praedikative023} Stig wird [gesund].}
    \ex{\label{ex:praedikative024} Stig bleibt [ein Arzt].}
    \ex{\label{ex:praedikative025} Stig ist, [wie er ist].}
    \ex{\label{ex:praedikative026} Stig ist [in Kopenhagen].}
  \end{xlist}
\end{exe}

Typischerweise ist in einer Struktur mit einem der Kopulaverben \textit{sein}, \textit{bleiben} und \textit{werden} sowie einer Subjekts-NP (im Nominativ) auch eine \textit{Prädikatsergänzung} in Form einer AP, NP, PP usw.\ zu erwarten.
Im Fall, dass eine prädikative NP vorliegt, stehen beide Ergänzungen der Kopula (nahezu immer) im Nominativ.
Siehe auch Abschnitt~\ref{sec:subjekte} und vor allem Vertiefung~\ref{vert:praednom} auf Seite~\pageref{vert:praednom}.

In den jetzt zu beschreibenden anderen Fällen ohne Kopulaverb ist die Diagnose nicht ganz so einfach.
Als Faustregel bzw.\ Behelfstest kann gelten, dass ein Prädikativum P einen semantisch kompatiblen Zusatz in Form von \textit{x sein/werden P} zulassen sollte, wobei \textit{x} hier für eine NP (oder einen Ergänzungssatz) steht, die im ursprünglichen Satz vorkommt.
Der Testsatz wird in den weiteren Beispielen jeweils hinter den ursprünglichen Satz geschrieben (nach \Folgt).
Als prädikativ werden \zB Konstituenten bezeichnet, die den Resultatszustand des vom Objekt bezeichneten Gegenstandes spezifizieren.\index{Prädikat!resultativ}
Diese sogenannten \textit{Resultativprädikate} werden in (\ref{ex:praedikative027}) illustriert.

\begin{exe}
  \ex\label{ex:praedikative027}
  \begin{xlist}
    \ex{\label{ex:praedikative028} Er fischt den Teich [leer].
      \Folgt\ Der Teich wird [leer].}
    \ex{\label{ex:praedikative029} Sie färbt den Pullover [grün].
      \Folgt\ Der Pullover wird [grün].}
    \ex{\label{ex:praedikative030} Er stampft die Äpfel [zu Brei].
      \Folgt\ Die Äpfel werden [zu Brei].}
  \end{xlist}
\end{exe}

Der Unterschied zwischen (\ref{ex:praedikative028}) und (\ref{ex:praedikative029}) ist, dass \textit{färben} in (\ref{ex:praedikative029}) auch ohne die AP ein transitives Verb ist, das \textit{den Pullover} als Akkusativ nehmen kann.
Folglich ist \textit{grün} hier auch weglassbar.
Bei (\ref{ex:praedikative028}) ist der Akkusativ ohne die AP so nicht möglich, und es müsste \textit{im Teich} heißen, wenn \textit{leer} weggelassen wird.
Weiterhin kann der Zustand des durch ein Subjekt oder Objekt bezeichneten Gegenstandes bei einer Handlung oder einem Vorgang als Angabe zum Verb realisiert werden, vgl.\ (\ref{ex:praedikative031}).
Auch hier benutzt man öfters den Begriff des \textit{prädikativen Adjektivs} oder ähnlich.

\begin{exe}
  \ex{\label{ex:praedikative031} Stig kam [übellaunig] in die Personalversammlung.\\
    \Folgt\ Stig war [übellaunig].}
\end{exe}

Schließlich gelten bestimmte Ergänzungen zu Verben wie \textit{gelten} (\textit{als}), \textit{halten} (\textit{für}) und \textit{schmecken}, die syntaktisch und semantisch heterogen sind, auch oft als Prädikativergänzungen, s.\ (\ref{ex:praedikative032}).

\begin{exe}
  \ex\label{ex:praedikative032}
  \begin{xlist}
    \ex{\label{ex:praedikative033} Ich halte den Begriff [für unnütz].\\
      \Folgt\ *Der Begriff ist/wird [für unnütz].}
    \ex{\label{ex:praedikative034} Sie gelten bei mir [als Langweiler].\\
      \Folgt\ *Sie sind/werden [als Langweiler].}
    \ex{\label{ex:praedikative035} Das Eis schmeckt [toll].\\
      \Folgt\ *Das Eis ist/wird [toll].}
  \end{xlist}
\end{exe}

Der Test schlägt nicht an.
Würde man hier der Motivation der Benennung als \textit{prädikativ} nachgehen, müsste man auf semantische Argumentationen ausweichen.
Formgrammatisch betrachtet wäre es völlig ausreichend, in allen Fällen von (\ref{ex:praedikative027}) bis (\ref{ex:praedikative032}) einfach von Adjektivergänzungen usw.\ zu sprechen und den Begriff des Prädikativums für die zweite Ergänzung der Kopulaverben zu reservieren.
Genau das geschieht mit Definition~\ref{def:praedikativ}.


\Definition{Prädikativ}{\label{def:praedikativ}%
Das \textit{Prädikativum} (die \textit{Prädikatsergänzung}) ist die Ergänzung von Kopulaverben, die nicht das Subjekt ist.
Die entsprechenden NP, PP usw.\ werden als \textit{prädikative NP}, \textit{prädikative PP} usw.\ bezeichnet.
\index{Prädikativ}
}

Eng zum Begriff des Satzprädikats gehört der Begriff des \textit{Subjekts}.
Dementsprechend verlangt Definition~\ref{def:praedikativ} nach einer Definition dessen, was ein Subjekt sein soll.
Informell wurde der Begriff bereits verwendet, aber Abschnitt~\ref{sec:subjekte} liefert jetzt eine gründlichere Diskussion.

\Zusammenfassung{%
Das Satzprädikat ist schwer zu definieren (am ehesten noch als die direkt voneinander abhängigen finiten und infiniten Verbformen eines Satzes).
Prädikative Konstituenten im weiteren Sinn sind eine heterogene Klasse, lassen sich aber auf den Prototyp der zweiten Kopula-Ergänzung (Nicht-Subjekt) zurückführen.
}


\section{Subjekte}
\label{sec:subjekte}

\subsection{Subjekte als Nominativ-Ergänzungen}
\label{sec:subjektealsnominativergaenzungen}

\index{Nominativ}
\index{Subjekt}
\index{Ergänzung!Nominativ--}

In diesem Abschnitt wird der Frage nachgegangen, welchen Stellenwert der traditionelle Begriff des \textit{Subjekts} in einer systematischen Grammatik hat.
Immerhin ist der Begriff im Schul- und Fremdsprachenunterricht immer noch zentral.
Naiv gedacht könnte man meinen, dass jeder Satz des Deutschen ein Subjekt und ein Prädikat haben muss.
Sätze wie (\ref{ex:subjektealsnominativergaenzungen036}) zeigen, dass das Weglassen von Subjekten gerne zu Ungrammatikalität führt.
Das potentielle Subjekt ist hier jeweils in eckige Klammern gesetzt.


\begin{exe}
  \ex\label{ex:subjektealsnominativergaenzungen036}
  \begin{xlist}
    \ex[ ]{\label{ex:subjektealsnominativergaenzungen037} [Frau Brüggenolte] backt einen Kuchen.}
    \ex[*]{\label{ex:subjektealsnominativergaenzungen038} Backt einen Kuchen.}
    \ex[*]{\label{ex:subjektealsnominativergaenzungen039} Einen Kuchen backt.}
    \ex[ ]{\label{ex:subjektealsnominativergaenzungen040} [Herr Uhl] raucht.}
    \ex[*]{\label{ex:subjektealsnominativergaenzungen041} Raucht.}
    \ex[ ]{\label{ex:subjektealsnominativergaenzungen042} [Es] regnet.}
    \ex[*]{\label{ex:subjektealsnominativergaenzungen043} Regnet.}
  \end{xlist}
\end{exe}


Es existieren zahlreiche Definitionen des Subjektbegriffs, und viele sind semantisch und daher nicht anhand von formgrammatischen Kriterien rekonstruierbar.
Die Gegenüberstellung von Subjekt und Prädikat als Begriffspaar ist insofern problematisch, als sie uns zwingt, den Prädikatsbegriff explizit zu machen, was evtl.\ gar nicht nötig ist, vor allem aber schwieriger als die Explizierung des Subjektsbegriffs (s.\ Abschnitt~\ref{sec:praedikateundpraedikativekonstituenten}).
Wenn wir uns ganz pragmatisch anschauen, was normalerweise als Subjekt bezeichnet wird, gibt es eine wesentlich einfachere Definition, die allerdings den Begriff des Subjekts nahezu überflüssig macht.

In (\ref{ex:subjektealsnominativergaenzungen044}) erweitern wir die Liste der Beispiele für Subjekte um einige weitere Typen bzw.\ um dieselben Typen in anderen Konstruktionen.

\begin{exe}
  \ex\label{ex:subjektealsnominativergaenzungen044}
  \begin{xlist}
    \ex{\label{ex:subjektealsnominativergaenzungen045} Zu Weihnachten backt [Frau Brüggenolte] Kekse.}
    \ex{\label{ex:subjektealsnominativergaenzungen046} [Herr Oelschlägel] nervt Herrn Uhl.}
    \ex{\label{ex:subjektealsnominativergaenzungen047} [Dass Herr Oelschlägel jeden Tag staubsaugt], nervt Herrn Uhl.}
    \ex{\label{ex:subjektealsnominativergaenzungen048} [Zu Fuß den Fahrstuhl zu überholen], machte mir als Kind Spaß.}
  \end{xlist}
\end{exe}

Es gilt zu ermitteln, was alle eingeklammerten Konstituenten auszeichnet.
Es fällt sofort auf, dass in allen Beispielen, in denen eine NP im Nominativ vorhanden ist, deren Kasus vom Verb regiert wird, diese immer dem traditionellen grammatischen Subjekt entspricht, vgl.\ (\ref{ex:subjektealsnominativergaenzungen045}) und (\ref{ex:subjektealsnominativergaenzungen046}).
Genau diese NP im Nominativ ist es auch, die mit dem finiten Verb kongruiert.

\index{Nebensatz}
In den Beispielen (\ref{ex:subjektealsnominativergaenzungen047}) und (\ref{ex:subjektealsnominativergaenzungen048}) gibt es keine NP im Nominativ, sondern satzförmige Ergänzungen, die traditionell auch als Subjekt bezeichnet würden.%
\footnote{Die Konstruktion mit \textit{zu}-Infinitiv wie in (\ref{ex:subjektealsnominativergaenzungen048}) erfüllt eigentlich nicht unsere vielleicht etwas strenge Definition eines Nebensatzes, weil sie kein finites Verb enthält.
Solche Infinitive werden in Abschnitt~\ref{sec:infinitivkontrolle} besprochen.}
Wir können in allen Fällen diese satzförmigen Ergänzungen durch ein Pronomen oder eine NP ersetzen, die im Nominativ steht, vgl.\ (\ref{ex:subjektealsnominativergaenzungen049}).
Zur Rekonstruktion der Bedeutung muss dann natürlich aus dem Kontext bekannt sein, was das Pronomen semantisch kodiert, was also im gegebenen Kontext seine Bedeutung ist.
Der Grammatik ist dies egal.
Die Umformungen sind auch außerhalb solcher Kontexte völlig grammatisch.
Das Subjekt ist also im Kern mit der Nominativ-Ergänzung des Verbs identisch, s.\ Definition~\ref{def:subjekt}.
\begin{exe}
  \ex\label{ex:subjektealsnominativergaenzungen049}
  \begin{xlist}
    \ex{\label{ex:subjektealsnominativergaenzungen050} Das nervt Herrn Uhl.}
    \ex{\label{ex:subjektealsnominativergaenzungen051} Das machte mir als Kind Spaß.}
  \end{xlist}
\end{exe}

\index{Ergänzungssatz}

\Definition{Subjekt}{\label{def:subjekt}%
Das \textit{Subjekt} ist die Nominativ-Ergänzung oder eine satzförmige Konstituente, die anstelle einer Nominativ-Ergänzung steht.
Die sogenannte \textit{Subjekt-Verb-Kongruenz} besteht zwischen dem regierten Nominativ und dem regierenden finiten Verb.
Ergänzungssätze und Infinitivkonstruktionen als Subjekte (\zB sog.\ \textit{Subjektsätze}) haben keine Merkmale, mit denen das finite Verb kongruieren könnte.
Das finite Verb steht dann kongruenzlos in der dritten Person Singular.
\index{Subjekt}
}

\index{Passiv}
\index{Imperativ}
\index{Subjekt}

Nominativ-Ergänzungen bzw.\ Subjekte haben einige besondere Eigenschaften.
Es fällt auf, dass wie in Beispiel (\ref{ex:subjektealsnominativergaenzungen052}) im Passiv die Nominativ"=Ergänzung des zugehörigen Aktivs wegfällt oder zur optionalen PP mit \textit{von} wird (vgl.\ Abschnitt~\ref{sec:passiv}).
Außerdem wird im Imperativ (\ref{ex:subjektealsnominativergaenzungen055}) das Subjekt unterdrückt.

\begin{exe}
  \ex\label{ex:subjektealsnominativergaenzungen052}
  \begin{xlist}
    \ex{\label{ex:subjektealsnominativergaenzungen053} [Die Mechanikerinnen] reparieren den Fahrstuhl.}
    \ex{\label{ex:subjektealsnominativergaenzungen054} Der Fahrstuhl wird repariert.}
  \end{xlist}
  \ex\label{ex:subjektealsnominativergaenzungen055}
  \begin{xlist}
    \ex{\label{ex:subjektealsnominativergaenzungen056} [Du] reparierst den Fahrstuhl.}
    \ex{\label{ex:subjektealsnominativergaenzungen057} Repariere den Fahrstuhl!}
  \end{xlist}
\end{exe}

Weiterhin gibt es Sätze mit nur einer Ergänzung, die im Dativ oder Akkusativ steht, wie in (\ref{ex:subjektealsnominativergaenzungen058}) und (\ref{ex:subjektealsnominativergaenzungen059}).
Für sie kann ebenfalls die Frage gestellt werden, ob sie ein Subjekt enthalten.

\begin{exe}
  \ex{\label{ex:subjektealsnominativergaenzungen058} Mir graut.}
  \ex{\label{ex:subjektealsnominativergaenzungen059} Uns graut.}
\end{exe}

Die Form \textit{mir} ist eindeutig als Dativ identifizierbar, passt also nicht zu der gegebenen Definition eines Subjekts als struktureller Nominativ.
Außerdem ist \textit{graut} dritte Person, es kongruiert also nicht mit \textit{mir}, das statisch erste Person ist.
An (\ref{ex:subjektealsnominativergaenzungen059}) sieht man außerdem, dass es keine Numeruskongruenz zwischen \textit{uns} (Plural) und
\textit{graut} (Singular) gibt.
Wir nehmen also an, dass \textit{mir} in (\ref{ex:subjektealsnominativergaenzungen058}) nicht als Subjekt betrachtet werden kann, weil ihm die wichtigen definitorischen Eigenschaften fehlen.
Es gibt demnach Sätze ohne grammatisches Subjekt.
Außerdem ist die Definition des Subjekts im Grunde auf den der Nominativ-Ergänzung (und einen Nebensatz o.\,Ä.\ an der Stelle eines Nominativs) reduzierbar, weswegen man eigentlich auch gut ohne den Subjektsbegriff auskommen könnte.
Der traditionelle Begriff ist aber zumindest definitorisch gut eingegrenzt worden.


\begin{Vertiefung}{Prädikative Nominative}

\label{vert:praednom}

\noindent Bei Nominativen wie \textit{der große Erfolg} in (\ref{ex:subjektealsnominativergaenzungen060}) muss man sich nun fragen, ob sie auch Subjekte sind.
Immerhin gibt es in diesen Sätzen zwei Nominative.

\begin{exe}
  \ex\label{ex:subjektealsnominativergaenzungen060}
  \begin{xlist}
    \ex{\label{ex:subjektealsnominativergaenzungen061} [Die Reparatur]$_\textrm{S}$ ist [der große Erfolg]$_\textrm{P}$.}
    \ex{\label{ex:subjektealsnominativergaenzungen062} [Die Reparatur]$_\textrm{S}$ wird [der große Erfolg]$_\textrm{P}$ genannt.}
  \end{xlist}
\end{exe}

Es wird jetzt vorgeschlagen, dass es sich in Fällen mit Kopulaverben (hier \textit{ist}) und Verben wie \textit{nennen} um strukturell ähnliche Fälle von einem zweiten Nominativ handelt, der keine Subjektseigenschaften hat.
Die sogenannten \textit{prädikativen Nominative} sind in (\ref{ex:subjektealsnominativergaenzungen060}) mit [~]$_\textrm{P}$ markiert, die Subjekte mit [~]$_\textrm{S}$.
Es fällt zunächst auf, dass der andere Nominativ \textit{die Reparatur} jeweils in der strukturellen Position steht, in der auch ein satzförmiges Subjekt stehen könnte, wie in (\ref{ex:subjektealsnominativergaenzungen063}).


\begin{exe}
  \ex\label{ex:subjektealsnominativergaenzungen063}
  \begin{xlist}
    \ex{\label{ex:subjektealsnominativergaenzungen064} [Dass der Fahrstuhl funktioniert]$_\textrm{S}$ ist [der große Erfolg]$_\textrm{P}$.}
    \ex{\label{ex:subjektealsnominativergaenzungen065} [Den Fahrstuhl erfolgreich zu reparieren]$_\textrm{S}$ wird [der große Erfolg]$_\textrm{P}$ genannt.}
  \end{xlist}
\end{exe}

Auch zeigt die Imperativbildung bei Kopulaverben (\ref{ex:subjektealsnominativergaenzungen066}), dass es in solchen Konstruktionen einen der beiden Nominative gibt (hier \textit{du}), der dem oben definierten Subjektsbegriff genügt.

\begin{exe}
  \ex\label{ex:subjektealsnominativergaenzungen066}
  \begin{xlist}
    \ex{\label{ex:subjektealsnominativergaenzungen067} [Du]$_\textrm{S}$ bist [der Assessor]$_\textrm{P}$.}
    \ex{\label{ex:subjektealsnominativergaenzungen068} Sei [der Assessor]$_\textrm{P}$!}
  \end{xlist}
\end{exe}

Darüber hinaus gibt es Fälle mit zwei NPs, bei denen eine alleine aufgrund der Kongruenz recht deutlich als Subjekt infragekommt, wie in (\ref{ex:subjektealsnominativergaenzungen069}).

\begin{exe}
  \ex{\label{ex:subjektealsnominativergaenzungen069} [Wir]$_\textrm{S}$ sind [das Volk]$_\textrm{P}$.}
\end{exe}

Man kann also davon ausgehen, dass einer der Nominative der Subjektsdefinition genügt, der andere jeweils nicht.
Besonders Kopulaverben haben also in den hier besprochenen Strukturen nicht zwei gleichartige Nominative, sondern einen subjektartigen und einen prädikativen.

Besonders markant bezüglich (\ref{ex:subjektealsnominativergaenzungen062}) und (\ref{ex:subjektealsnominativergaenzungen065}) ist außerdem, dass sie eigentlich Passive sind, die Aktivsätzen wie denen in (\ref{ex:subjektealsnominativergaenzungen070}) entsprechen.

\begin{exe}
  \ex\label{ex:subjektealsnominativergaenzungen070}
  \begin{xlist}
    \ex{\label{ex:subjektealsnominativergaenzungen071} Man nennt [die Reparatur] [den großen Erfolg]$_\textrm{P}$.}
    \ex{\label{ex:subjektealsnominativergaenzungen072} Man nennt [den Fahrstuhl zu reparieren] [den großen Erfolg]$_\textrm{P}$.}
  \end{xlist}
\end{exe}

Was im Passiv der Subjektsnominativ (\textit{die Reparatur}) und der Prädikatsnominativ (\textit{der große Erfolg}) sind, taucht im zugehörigen Aktivsatz beides als Akkusativ auf.\index{Nominativ}\index{Akkusativ}
Es ist also gar nicht zielführend, ausdrücklich vom Prädikatsnominativ zu sprechen, denn es handelt sich vielmehr um eine zusätzliche NP-Ergänzung bei bestimmten Verben, deren Kasus durch eine Art von Kongruenz zustande kommt.

\end{Vertiefung}


\subsection{Arten von \textit{es} im Nominativ}
\label{sec:artenvonesimnominativ}

Zur Behandlung des Subjekts gehört unbedingt eine Diskussion des Nominativ-Pronomens \textit{es}.
Es muss entschieden werden, ob es wesentliche Unterschiede zwischen den verschiedenen Vorkommen des Pronomens \textit{es} im Nominativ in (\ref{ex:artenvonesimnominativ073}) gibt.

\begin{exe}
  \ex\label{ex:artenvonesimnominativ073}
  \begin{xlist}
    \ex{\label{ex:artenvonesimnominativ074} Es öffnet die Tür.}
    \ex{\label{ex:artenvonesimnominativ075} Es regt mich auf, dass die Politik schon wieder versagt.}
    \ex{\label{ex:artenvonesimnominativ076} Es öffnet ein Kind die Tür.}
    \ex{\label{ex:artenvonesimnominativ077} Es wird jetzt gearbeitet.}
    \ex{\label{ex:artenvonesimnominativ078} Es friert mich.}
    \ex{\label{ex:artenvonesimnominativ079} Es regnet in Strömen.}
  \end{xlist}
\end{exe}

Zu beachten ist dabei, dass in (\ref{ex:artenvonesimnominativ075}) ein Nominativ-\textit{es} zusammen mit einem Subjektsatz vorkommt, der normalerweise die Stelle einer NP im Nominativ besetzt.
Beispiel (\ref{ex:artenvonesimnominativ076}) enthält zwei Nominativ-NPs (eine davon \textit{es}).
Die beiden Sätze enthalten aber keine Verben bzw.\ keine Konstruktionen, in denen sogenannte \textit{Prädikatsnominative} (s.\ Vertiefung~\ref{vert:praednom} auf Seite~\pageref{vert:praednom}) vorkommen, so dass eine andere Erklärung für den doppelten Nominativ gefunden werden muss.

Die Argumentation soll hier wieder möglichst auf Tests beruhen, die nachvollziehbar zwischen den verschiedenen Verwendungsweisen zu differenzieren helfen.
Zuerst wird in (\ref{ex:artenvonesimnominativ080}) getestet, ob \textit{es} durch das Pronomen \textit{dieses} ersetzt werden kann.


\begin{exe}
  \ex\label{ex:artenvonesimnominativ080}
  \begin{xlist}
    \ex[ ]{\label{ex:artenvonesimnominativ081} Dieses öffnet die Tür.}
    \ex[*]{\label{ex:artenvonesimnominativ082} Dieses regt mich auf, dass die Politik schon wieder versagt.}
    \ex[*]{\label{ex:artenvonesimnominativ083} Dieses öffnet ein Kind die Tür.}
    \ex[*]{\label{ex:artenvonesimnominativ084} Dieses wird jetzt gearbeitet.}
    \ex[*]{\label{ex:artenvonesimnominativ085} Dieses friert mich.}
    \ex[*]{\label{ex:artenvonesimnominativ086} Dieses regnet in Strömen.}
  \end{xlist}
\end{exe}


Für alle Sätze außer (\ref{ex:artenvonesimnominativ081}) geht dies nicht.
Das liegt daran, dass diese anderen Varianten von \textit{es} semantisch völlig leer sind.
Sie verweisen also nicht auf Objekte in der Welt bzw.\ bezeichnen nichts.
Während es also semantisch leere Verwendungen von \textit{es} gibt, hat \textit{dieses} immer eine normale pronominale Semantik.
Dem Pronomen wird hier von \textit{öffnen} eine Agens"=Rolle zugewiesen.

In (\ref{ex:artenvonesimnominativ075}) ist \textit{es} offensichtlich ein Korrelat zum Ergänzungssatz (vgl.\ dazu Abschnitt \ref{sec:ergaenzungssaetze}).\index{Korrelat}
Es nimmt zwar die Rolle auf, die das Verb an sein Subjekt vergibt, reicht sie aber an den Ergänzungssatz weiter.
Das Pronomen \textit{dieses} ist (ganz unabhängig von der Rollenvergabe) kein zulässiges Korrelat, was zur Ungrammatikalität von (\ref{ex:artenvonesimnominativ082}) führt.
Den Status von \textit{es} als Korrelat eines Subjektsatzes kann man gut testen.
Einerseits muss überhaupt ein Subjektsatz vorhanden sein.
Andererseits muss \textit{es} durch diesen ersetzbar sein, vgl.\ (\ref{ex:artenvonesimnominativ087}) als semantisch äquivalente Umformung von (\ref{ex:artenvonesimnominativ075}).


\begin{exe}
  \ex{\label{ex:artenvonesimnominativ087} Dass die Politik schon wieder versagt, regt mich auf.}
\end{exe}


Die Fälle der normalen regierten pronominalen NP in (\ref{ex:artenvonesimnominativ074}) und des Korrelats in (\ref{ex:artenvonesimnominativ075}) -- beide mit semantischer Rolle -- können wir als hinreichend klassifiziert zur Seite legen.
Die übrigen \textit{es}, die alle semantisch leer sind, keine eigene Rolle durch das Verb zugewiesen bekommen und deswegen die Ersetzung durch \textit{dieses} auch nicht zulassen, können bezüglich ihres grammatischen Verhaltens weiter differenziert werden.%
\footnote{Die Rollen werden jeweils an andere Konstituenten vergeben wie in (\ref{ex:artenvonesimnominativ076}) und (\ref{ex:artenvonesimnominativ078}), oder es wird gar keine Rolle vom Verb vergeben wie in (\ref{ex:artenvonesimnominativ077}) und (\ref{ex:artenvonesimnominativ079}).}
Da diese \textit{es}-Varianten offensichtlich keine Bedeutung i.\,e.\,S.\ haben, könnte es \zB sein, dass sie weglassbar (optional) sind.
Die Weglassprobe wird in (\ref{ex:artenvonesimnominativ088}) durchgeführt.


\begin{exe}
  \ex\label{ex:artenvonesimnominativ088}
  \begin{xlist}
    \ex[ ]{\label{ex:artenvonesimnominativ089} Ein Kind öffnet die Tür.}
    \ex[ ]{\label{ex:artenvonesimnominativ090} Jetzt wird gearbeitet.}
    \ex[ ]{\label{ex:artenvonesimnominativ091} Mich friert.}
    \ex[*]{\label{ex:artenvonesimnominativ092} In Strömen regnet.}
  \end{xlist}
\end{exe}


\index{Verb!Wetter--}\index{Verb!Experiencer--}

Bei den sogenannten \textit{Wetter-Verben} wie \textit{regnen} (oder \textit{schneien} und \textit{dämmern}, sehr ähnlich auch bei bestimmten Varianten von \textit{klingeln} oder \textit{duften}) ist das Pronomen nicht optional.
Daraus kann man schließen, dass \textit{es} bei Wetter-Verben auf jeden Fall regiert und Teil des Valenzrahmens ist.
Verben wie \textit{frieren} in (\ref{ex:artenvonesimnominativ091}) vergeben (anders als Wetter"=Verben) immer eine Experiencer-Rolle an eine Ergänzung im Dativ oder Akkusativ (hier \textit{mich}).
Anders als bei den Wetter-Verben ist \textit{es} allerdings dabei manchmal fakultativ und manchmal obligatorisch.
Bei \textit{frieren} in (\ref{ex:artenvonesimnominativ078}) bzw.\ (\ref{ex:artenvonesimnominativ091}) ist \textit{es} nicht obligatorisch, bei \textit{gehen} in (\ref{ex:artenvonesimnominativ093}) aber schon.


\begin{exe}
  \ex\label{ex:artenvonesimnominativ093}
  \begin{xlist}
    \ex[]{\label{ex:artenvonesimnominativ094} Mir geht es gut.}
    \ex[*]{\label{ex:artenvonesimnominativ095} Mir geht gut.}
  \end{xlist}
\end{exe}


Die Fälle mit obligatorischem \textit{es} etablieren auch für diese Klasse \textit{es} sicher als Ergänzung und damit als Teil des Valenzrahmens.
Wir behandeln daher Experiencer"=Verben und Wetter"=Verben bezüglich des \textit{es} einheitlich und sagen, dass \textit{es} bei ihnen eine entweder fakultative oder obligatorische Ergänzung ist.

Da \textit{es} hier nicht durch andere Pronomen oder NPs ersetzbar ist, regiert das Verb nicht nur den Kasus, sondern ganz konkret die Form des Pronomens.
Die Valenzliste von \textit{regnen} sieht also aus wie in (\ref{ex:artenvonesimnominativ096}).%
\footnote{Die Darstellung ist vereinfacht, da \textit{es} hier nicht als Merkmalsstruktur angegeben wird.
Außerdem können alternativ viele Sprecher \textit{das} verwenden, was man ggf.\ dazuschreiben müsste.}
Es fordert eine Ergänzung, die auf das Pronomen \textit{es} festgelegt ist.
Besonders ist dabei, dass diesen Ergänzungen vom Verb keine Rolle zugewiesen wird und sie semantisch leer sind.


\begin{exe}
  \ex{\label{ex:artenvonesimnominativ096} \textit{regnen} = [\textsc{Valenz}: \Rollen{\textit{es}}]}
\end{exe}


Ein weiterer Test zur Ausdifferenzierung verschiedener Arten von \textit{es} basiert auf dem Versuch, \textit{es} aus dem Vorfeld zu verdrängen.
Das sieht dann aus wie in (\ref{ex:artenvonesimnominativ097}), wobei normale Pronomina und Korrelate weiterhin nicht mehr berücksichtigt werden.


\begin{exe}
  \ex\label{ex:artenvonesimnominativ097}
  \begin{xlist}
    \ex[*]{\label{ex:artenvonesimnominativ098} Ein Kind öffnet es die Tür.}
    \ex[*]{\label{ex:artenvonesimnominativ099} Jetzt wird es gearbeitet.}
    \ex[ ]{\label{ex:artenvonesimnominativ100} Mich friert es.}
    \ex[ ]{\label{ex:artenvonesimnominativ101} In Strömen regnet es.}
  \end{xlist}
\end{exe}


Bei \textit{frieren} und \textit{regnen} muss \textit{es} nicht im Vorfeld stehen.
In diesem Test verhalten sich Experiencer"=Verben und Wetter"=Verben immer gleich, was die Annahme einer gemeinsamen Klasse weiter rechtfertigt.
Das \textit{es} in Sätzen wie (\ref{ex:artenvonesimnominativ074}) und bei unpersönlichen Passiven ist auf das Vorfeld festgelegt.
Da es wegfällt, sobald eine andere Konstituente im Vorfeld steht, ist seine einzige Funktion offensichtlich, das Vorfeld zu füllen, wenn Sprecher aus irgendwelchen Gründen nichts anderes ins Vorfeld stellen möchten.
Dieses reine Vorfeld-\textit{es} nennt man auch \textit{positionales Es}.\index{Pronomen!positional}
Auf die Gründe, warum überhaupt Konstituenten ins Vorfeld gestellt werden, gehen wir nicht ein, weil das im gegebenen Rahmen zu weit führen würde.
Intuitiv kann sich aber jeder Erstsprecher des Deutschen wahrscheinlich vorstellen, dass sich die angemessenen Äußerungskontexte für die Satzvarianten in (\ref{ex:artenvonesimnominativ102}) unterscheiden, dass es also einen funktionalen Unterschied zwischen den beiden Sätzen gibt.
Anders gesagt ist es keine Zufallsentscheidung von Sprechern, welche Konstituente sie ins Vorfeld stellen, bzw.\ ob sie ein inhaltlich leeres \textit{es} dort plazieren.


\begin{exe}
  \ex\label{ex:artenvonesimnominativ102}
  \begin{xlist}
    \ex{\label{ex:artenvonesimnominativ103} Ein Kind öffnet die Tür.}
    \ex{\label{ex:artenvonesimnominativ104} Es öffnet ein Kind die Tür.}
  \end{xlist}
\end{exe}


Das Besondere an Sätzen wie (\ref{ex:artenvonesimnominativ104}) gegenüber unpersönlichen Passiven ist, dass eine weitere NP im Nominativ (hier \textit{ein Kind}) vorhanden ist.
Das \textit{es} ist dabei nicht das Subjekt, wie das Kongruenzverhalten in (\ref{ex:artenvonesimnominativ105}) zeigt.
Vielmehr ist der andere Nominativ das mit dem Verb kongruierende und vom Verb regierte Subjekt.


\begin{exe}
  \ex\label{ex:artenvonesimnominativ105}
  \begin{xlist}
    \ex[ ]{\label{ex:artenvonesimnominativ106} Es öffnet eine Frau die Tür.}
    \ex[ ]{\label{ex:artenvonesimnominativ107} Es öffnen zwei Frauen die Tür.}
    \ex[*]{\label{ex:artenvonesimnominativ108} Es öffnet zwei Frauen die Tür.}
  \end{xlist}
\end{exe}


Die Tests ergeben die sich unterschiedlich verhaltenden Gruppen im Entscheidungsbaum in Abbildung~\ref{fig:artenvonesimnominativ109}.
Dem positionalen \textit{es} und der \textit{es}-Ergänzung bei Wetter- und Experiencer-Verben wird jeweils keine Rolle zugewiesen, und sie bezeichnen konsequenterweise auch keine Gegenstände in der Welt.
Man nennt Pronomina wie diese auch \textit{Expletivpronomina}.\index{Pronomen!expletiv}
Wie auf Seite~\pageref{abs:wortakzentimdeutschen171} in Abschnitt~\ref{sec:wortakzentimdeutschen} bereits erläutert, sind Expletivpronomina nicht betonbar.

\begin{figure}[!htbp]
  \centering
  \begin{forest}
    for tree={l sep=2em, s sep=2.5em},
    [Ersetzbar durch \textit{dieses}?, decide
      [\textbf{normales Pronomen}, finall, yes]
      [Mit Ergänzungssatz?, no, decide
        [\textbf{Korrelat}, finall, yes]
        [Obligatorisch\\im Vorfeld?, decide, no
          [\textbf{fakultatives/positionales \textit{es}}\\{(Vorfeld-\textit{es})}, finall, yes]
          [\textbf{obligatorische oder}\\\textbf{fakultative Ergänzung}\\{(Wetter\slash Experiencer-Verben)}, finall, no]
        ]
      ]
    ]
  \end{forest}
  \caption{Entscheidungsbaum zur Klassifikation von Nominativ \textit{es}}
  \label{fig:artenvonesimnominativ109}
  \index{Verb!Wetter--}
  \index{Verb!Experiencer--}
  \index{Pronomen!positional}
  \index{Korrelat}
\end{figure}

Der große Vorteil an dem hier vertretenen deskriptiven Vorgehen ist, dass kaum spezifische theoretische Begriffe zur Unterscheidung der verschiedenen \textit{es} benötigt werden.
Die Tests isolieren eindeutig die unterschiedlichen Verwendungsweisen.
Auch wenn die theoretische Interpretation dieses Befundes eine andere wäre, würde sich nichts daran ändern, dass es mindestens vier gut unterscheidbare Verwendungsweisen von \textit{es} gibt.


\Zusammenfassung{%
Subjekte sind die Nominativ-Ergänzungen von Verben oder Nebensätze (und Ähnliches), die deren Stelle einnehmen.
Es gibt mindestens vier verschiedene \textit{es} im Nominativ: normales Pronomen, Korrelat, fakultative bzw.\ obligatorische spezifische Ergänzung und  positionales \textit{es}.
}

\section{Objekte, Ergänzungen und Angaben}
\label{sec:objekteergaenzungenundangaben}

\subsection{Akkusative und direkte Objekte}
\label{sec:akkusativeunddirekteobjekte}

\index{Akkusativ}

Das Wesentliche zum Akkusativ wurde bereits in den Abschnitten~\ref{sec:kasus} und~\ref{sec:passiv} gesagt.
Die sogenannten \textit{Objektsätze}, also Sätze, die anstelle eines Akkusativs stehen, wurden in Abschnitt~\ref{sec:ergaenzungssaetze} behandelt.
Zu den \textit{Objektinfinitiven} folgt später Abschnitt~\ref{sec:infinitivkontrolle}.
Definition~\ref{def:dirobj} setzt nun den Begriff des \textit{direkten Objekts} in Beziehung zum Akkusativ.

\Definition{Direktes Objekt (=~Akkusativobjekt)}{\label{def:dirobj}%
\textit{Direkte Objekte} sind Akkusativ-Ergänzungen von Verben.
\index{Objekt!direkt}
\index{Ergänzung!Akkusativ--}
}

Das \textit{direkte Objekt} ist also nur eine terminologische Variante des \textit{Akkusativobjekts}.
Es muss an dieser Stelle nur noch auf einen ungewöhnlichen Typ von Verben verwiesen werden.
Bei diesen Verben stehen -- wie in (\ref{ex:akkusativeunddirekteobjekte158}) illustriert -- neben dem Nominativ zwei Akkusative (\textit{ihn} und \textit{das Schwimmen}).

\begin{exe}
  \ex\label{ex:akkusativeunddirekteobjekte158}
  \begin{xlist}
    \ex[ ]{\label{ex:akkusativeunddirekteobjekte159} Ich lehre ihn das Schwimmen.}
    \ex[*]{\label{ex:akkusativeunddirekteobjekte160} Das Schwimmen wird ihn gelehrt.}
    \ex[*]{\label{ex:akkusativeunddirekteobjekte161} Er wird das Schwimmen gelehrt.}
  \end{xlist}
\end{exe}

\index{Akkusativ!Doppel--}
\index{Angabe!Akkusativ--}

Doppelakkusative (neben \textit{lehren} eventuell noch Verben wie \textit{abfragen}) erlauben keine Passivierung, wie (\ref{ex:akkusativeunddirekteobjekte160}) und (\ref{ex:akkusativeunddirekteobjekte161}) zeigen.
Andere scheinbare Doppelakkusative sind Spezialfälle, in denen der zweite Akkusativ zwar redensartlich oder idiomatisch gebunden, aber nicht valenzgebunden ist, \zB \textit{kümmern} mit Akkusativen wie \textit{einen feuchten Kehricht}.
Zu beachten ist, dass hier keine Situation beschrieben wird, in der tatsächlich ein Kehricht existiert, der eine Rolle in einer \textit{kümmern}-Situation spielt.
Wir haben es vielmehr mit einer redensartlich festgelegten Angabe mit der Bedeutung von \textit{überhaupt nicht} zu tun.

Gelegentlich treten Akkusative als Angabe auf, wie bei der Zeitangabe im Akkusativ in (\ref{ex:akkusativeunddirekteobjekte163}).
Solche Angaben sind nicht passivierbar (\ref{ex:akkusativeunddirekteobjekte164}), nicht subklassenspezifisch, und sie können zu valenzgebundenen Akkusativen hinzutreten (\ref{ex:akkusativeunddirekteobjekte165}).
Wie so oft versagt übrigens auch hier die Grammatikerfrage \textit{Wen oder was bin ich geschwommen?}\index{Grammatikerfrage}

\begin{exe}
  \ex\label{ex:akkusativeunddirekteobjekte162}
  \begin{xlist}
    \ex[ ]{\label{ex:akkusativeunddirekteobjekte163} Ich bin [eine Stunde] geschwommen.}
    \ex[*]{\label{ex:akkusativeunddirekteobjekte164} [Eine Stunde] ist von mir geschwommen worden.}
    \ex[ ]{\label{ex:akkusativeunddirekteobjekte165} Ich habe [eine Stunde] [die Stofftiere] geföhnt.}
  \end{xlist}
\end{exe}

\subsection{Dative und indirekte Objekte}
\label{sec:dativeundindirekteobjekte}

\index{Dativ}

Wie zu den Akkusativen wurde auch zu den Dativen schon fast alles Wesentliche gesagt.
Parallel zum direkten Objekt definieren wir das indirekte Objekt mit Definition~\ref{def:indobj} als Dativ-Ergänzung.%
\footnote{Je nachdem werden auch die seltenen Genitivergänzungen wie bei \textit{gedenken}, \textit{bedürfen}, \textit{anklagen}, \textit{verdächtigen} als indirekte Objekte bezeichnet.
Darauf können wir hier im einzelnen nicht eingehen, vgl.\ aber auch Übung~\ref{exc:relationenundpraedikate11}.}
Die in der Definition vorkommenden sekundären Begriffe wie \textit{Nutznießerdativ} werden im weiteren Verlauf des Abschnittes eingeführt.

\Definition{Indirektes Objekt (=~Dativobjekt)}{\label{def:indobj}%
\textit{Indirekte Objekte} sind Dativ-Ergänzungen von Verben.
Dies sind der Dativ bei gewöhnlichen dreistelligen Verben, der \textit{Nutznießerdativ} und der \textit{Pertinenzdativ}, nicht aber der \textit{Bewertungsdativ}.
\index{Objekt!indirekt}
}


Es muss nun noch der Status verschiedener sogenannter \textit{freier Dative} als Ergänzung oder Angabe diskutiert werden.
Für die Dative in (\ref{ex:dativeundindirekteobjekte166}) soll entschieden werden, ob sie Ergänzungen oder Angaben sind.

\begin{exe}
  \ex\label{ex:dativeundindirekteobjekte166}
  \begin{xlist}
    \ex[ ]{\label{ex:dativeundindirekteobjekte167} Alma gibt ihm heute ein Buch.}
    \ex[ ]{\label{ex:dativeundindirekteobjekte168} Alma fährt mir heute aber wieder schnell.}
    \ex[ ]{\label{ex:dativeundindirekteobjekte169} Alma mäht mir heute den Rasen.}
    \ex[ ]{\label{ex:dativeundindirekteobjekte170} Alma klopft mir heute auf die Schulter.}
  \end{xlist}
\end{exe}

\index{Dativ!frei}

Über die Frage der freien Dative bzw.\ Dativ"=Angaben wurde viel geschrieben, und ein Großteil der Auseinandersetzung kommt dadurch zustande, dass unterschiedliche oder ungenaue Definitionen von Valenz zugrundegelegt werden.
Wir haben das \textit{bekommen}-Passiv in Abschnitt~\ref{sec:bekommenpassiv} genauso wie das \textit{werden}-Passiv als Valenzänderung beschrieben.
Damit ist die Frage nach dem Ergänzungsstatus der Dative eigentlich leicht zu beantworten.
Wir ziehen die versuchten Bildungen von \textit{bekommen}-Passiven in (\ref{ex:dativeundindirekteobjekte171}) hinzu.

\begin{exe}
  \ex\label{ex:dativeundindirekteobjekte171}
  \begin{xlist}
    \ex[ ]{\label{ex:dativeundindirekteobjekte172} Er bekommt von Alma heute ein Buch gegeben.}
    \ex[*]{\label{ex:dativeundindirekteobjekte173} Ich bekomme von Alma heute aber wieder schnell gefahren.}
    \ex[ ]{\label{ex:dativeundindirekteobjekte174} Ich bekomme von Alma heute den Rasen gemäht.}
    \ex[ ]{\label{ex:dativeundindirekteobjekte175} Ich bekomme von Alma heute auf die Schulter geklopft.}
  \end{xlist}
\end{exe}

\index{Dativ!Bewertungs--}
\index{Dativ!Nutznießer--}
\index{Dativ!Pertinenz--}

Nach diesem Test kann nur der Dativ in (\ref{ex:dativeundindirekteobjekte168}) ein echter freier Dativ (also eine Dativ-Angabe) sein, denn nur bei diesem ist das \textit{bekommen}-Passiv nicht bildbar.
Die drei Arten von Dativen in (\ref{ex:dativeundindirekteobjekte168})--(\ref{ex:dativeundindirekteobjekte170}) haben eigene Namen gemäß ihrer semantischen Funktion.
Dative wie der in (\ref{ex:dativeundindirekteobjekte168}) kodieren, dass der im Dativ bezeichnete Mensch sich den Inhalt des restlichen Satzes als Bewertung zueigen macht, und wir nennen ihn daher hier den \textit{Bewertungsdativ}.
In (\ref{ex:dativeundindirekteobjekte169}) wird im Dativ der Nutznießer der im Satz beschriebenen Handlung kodiert, und wir nennen sie hier \textit{Nutznießerdative}.
Der sogenannte \textit{Zugehörigkeitsdativ} oder \textit{Pertinenzdativ} in (\ref{ex:dativeundindirekteobjekte170}) kodiert ein Individuum, an dem an einem bestimmten Körperteil eine Handlung durchgeführt wird.

\index{Valenz}

Der Pertinenzdativ kann bei allen semantisch kompatiblen Verben hinzugefügt werden.
Der Nutznießerdativ ist kompatibel zu allen Verben, die nicht intransitiv sind und nicht bereits einen Dativ auf ihrer Valenzliste haben.
Weil dies sehr viele, aber durch wenige Bedingungen beschreibbare Verben sind, können die Bildung des Nutznießerdativs und des Pertinenzdativs elegant als Valenzanreicherungen analysiert werden, so dass man nicht zu jedem der betreffenden Verben einzeln auf der Valenzliste den (optionalen) Dativ kodieren muss.
Eine Lexikonregel fügt dabei der Valenzliste aller semantisch kompatiblen Verben die Nutznießerdative und Pertinenzdative hinzu.
Ein Satz wie (\ref{ex:dativeundindirekteobjekte169}) kommt also zustande, indem einem zweistelligen Verb (Nominativ, Akkusativ) \textit{mähen} zunächst eine Dativ-Ergänzung hinzugefügt wird, wodurch es zu einem Verb mit Nominativ,  Akkusativ und Dativ wird.
Um (\ref{ex:dativeundindirekteobjekte174}) zu erzeugen, muss dann nur nach Satz~\ref{satz:bekommenpass} passiviert werden.

Für den Pertinenzdativ liegt diese Analyse sogar noch näher, da hier immer auch noch ein weiteres Element (das den Körperteil bezeichnet) hinzugefügt werden muss, um die Sätze grammatisch zu machen, vgl.\ (\ref{ex:dativeundindirekteobjekte176}).
Das Valenzmuster des Verbs muss also beim Pertinenzdativ in erheblichem Ausmaß umgebaut werden.

\begin{exe}
  \ex[*]{\label{ex:dativeundindirekteobjekte176} Alma klopft mir heute.}
\end{exe}

Außerdem sind der Nutznießerdativ und der Pertinenzdativ bei Verben nicht möglich, die bereits einen Dativ auf der Valenzliste haben, vgl.\ (\ref{ex:dativeundindirekteobjekte178}).
Der Bewertungsdativ erlaubt dies aber, so dass sich Sätze mit zwei Dativen ergeben wie in (\ref{ex:dativeundindirekteobjekte179}).
Um diese Beispiele noch besser zu verstehen, sollten die Eigenschaften des Bewertungsdativs, die in Vertiefung \ref{vert:djwackernagel} (auf Seite~\pageref{vert:djwackernagel}) beschrieben werden, berücksichtigt werden.
In (\ref{ex:dativeundindirekteobjekte178}) ist gemäß dieser Eigenschaften die Interpretation von \textit{mir} als Bewertungsdativ ausgeschlossen.
Es wird dann abschließend Satz~\ref{def:freidat} aufgestellt.

\begin{exe}
  \ex\label{ex:dativeundindirekteobjekte177}
  \begin{xlist}
    \ex[*]{\label{ex:dativeundindirekteobjekte178} Alma gibt mir [dem Mann] ein Buch.}
    \ex[ ]{\label{ex:dativeundindirekteobjekte179} Alma gibt mir [den Kindern] zu viele Schoko-Rosinen.}
  \end{xlist}
\end{exe}


\Satz{Freie Dative als Ergänzungen und Angaben}{\label{def:freidat}%
Von den sogenannten \textit{freien Dativen} ist nur der Bewertungsdativ eine Angabe.
Der Nutznießerdativ und der Pertinenzdativ sind Ergänzungen, die durch eine Valenzanreicherung jedem Verb (außer intransitiven Verben und Verben, die schon einen Dativ auf der Valenzliste haben) hinzugefügt werden können.
\index{Ergänzung!Dativ--}
\index{Angabe!Dativ--}
}

\begin{Vertiefung}{Eigenschaften des Bewertungsdativs}

  \label{vert:djwackernagel}
  \index{Dativ!Bewertungs--}
  \index{Wackernagel-Position}

\noindent Der Bewertungsdativ kann nicht an jeder beliebigen Stelle im Satz stehen.
Die Sätze in (\ref{ex:dativeundindirekteobjekte180}) demonstrieren dies im Vergleich zu den anderen Dativen.

\begin{exe}
  \ex\label{ex:dativeundindirekteobjekte180}
  \begin{xlist}
    \ex[ ]{\label{ex:dativeundindirekteobjekte181} Alma gibt heute ihm ein Buch.}
    \ex[*]{\label{ex:dativeundindirekteobjekte182} Alma fährt heute mir aber wieder schnell.}
    \ex[ ]{\label{ex:dativeundindirekteobjekte183} Alma mäht heute mir den Rasen.}
    \ex[ ]{\label{ex:dativeundindirekteobjekte184} Alma klopft heute mir auf die Schulter.}
  \end{xlist}
\end{exe}

Außer dem Bewertungsdativ in (\ref{ex:dativeundindirekteobjekte182}) können alle anderen Dative auch im Mittelfeld weiter hinten stehen.
Eventuell muss man die Sätze auf bestimmte Weise betonen (i.\,d.\,R.\ so, dass der Dativ betont bzw.\ fokussiert wird), aber die Sätze sind auf keinen Fall ungrammatisch.
Der Bewertungsdativ ist im V2-Satz auf die Position nach dem finiten Verb (also ganz am Anfang des Mittelfelds) festgelegt (die sog.\ \textit{Wackernagel-Position}).

Eine weitere wichtige Eigenschaft des Bewertungsdativs ist, dass der Satz entweder eine Vergleichskonstruktion wie \textit{zu laut} (\ref{ex:dativeundindirekteobjekte186}) oder eine anderweitige Kennzeichnung als subjektive Äußerung durch Partikeln wie \textit{aber} (\ref{ex:dativeundindirekteobjekte187}) enthalten muss, um nicht ungrammatisch zu sein (\ref{ex:dativeundindirekteobjekte188}).

\begin{exe}
  \ex\label{ex:dativeundindirekteobjekte185}
  \begin{xlist}
    \ex[ ]{\label{ex:dativeundindirekteobjekte186} Du redest mir zu laut.}
    \ex[ ]{\label{ex:dativeundindirekteobjekte187} Du schreist mir aber ganz schön.}
    \ex[*]{\label{ex:dativeundindirekteobjekte188} Du schreist mir.}
  \end{xlist}
\end{exe}

\end{Vertiefung}

\subsection{PP-Ergänzungen und PP-Angaben}
\label{sec:ppergaenzungenundppangaben}

\index{Ergänzung!PP--}
\index{Objekt!präpositional}

Bisher wurde viel über nominale Ergänzungen und Angaben geredet.
Aber welche PPs als Ergänzungen (also \textit{Präpositionalobjekte}) und welche als Angabe betrachtet werden sollen, ist noch offen.\index{Präpositionalobjekt}
Auch hierzu gibt es sowohl bei den definitorischen Kriterien als auch bei den Entscheidungen im Einzelfall immer wieder Schwierigkeiten.
Eine Tendenz ist, dass die eigenständige Bedeutung der Präposition in einer PP-Ergänzung (im Gegensatz zur PP-Angabe) nicht mehr besonders deutlich erkennbar ist.
Dies hängt damit zusammen, dass der PP-Ergänzung ihre semantische Rolle direkt vom Verb zugewiesen wird und eben nicht die Präposition selber für die Semantik zuständig ist.
An \textit{unter} (mit Dativ) in (\ref{ex:ppergaenzungenundppangaben189}) wird dies deutlich.

\begin{exe}
  \ex\label{ex:ppergaenzungenundppangaben189}
  \begin{xlist}
    \ex{\label{ex:ppergaenzungenundppangaben190} Viele Menschen leiden unter Vorurteilen.}
    \ex{\label{ex:ppergaenzungenundppangaben191} Viele Menschen schwitzen unter Sonnenschirmen.}
  \end{xlist}
\end{exe}

Die PP-Ergänzung in (\ref{ex:ppergaenzungenundppangaben190}) hat keinerlei lokale Bedeutung mehr, und \textit{Vorurteilen} erhält seine Rolle eindeutig durch das Verb \textit{leiden}.
In (\ref{ex:ppergaenzungenundppangaben191}) wird \textit{unter} aber mit seiner eigentlichen räumlichen Semantik verwendet und ist ziemlich sicher selber für die Rollenzuweisung an \textit{Sonnenschirmen} zuständig.
Allein diese Beobachtung zeigt, dass es trotz aller Probleme auf keinen Fall aussichtslos ist, die Unterscheidung zwischen Ergänzung und Angabe zu machen.

Es gibt kein strenges Testkriterium, aus dem für alle denkbaren Fälle eine klare Definition abgeleitet werden kann.
Eindeutige Fälle von PP-Ergänzungen sind vor allem die, in denen die PP nicht weglassbar ist, wie bei \textit{übergeben} (\textit{an}), aber das sind leider nur wenige.
Außerdem gibt es einige Fälle, in denen eine bestimmte PP eine Ergänzung sein muss, weil das Verb zur Bedeutung der Präposition inkompatibel ist.
Dies ist \zB der Fall bei \textit{glauben} (\textit{an} mit Akkusativ), weil das Verb \textit{glauben} gar keine räumlich-direktionale Bedeutung hat wie \zB \textit{treten} (in \textit{an die Tür treten} usw.).

Ein Test auf den Ergänzungsstatus von PPs benutzt eine bestimmte Art der Paraphrase.
Dabei wird die potentielle PP-Ergänzung (PP\Sub{i}) aus dem Satz weggelassen und als Zusatz in der Form \textit{dies geschieht} PP\Sub{i} an den Satzrest angefügt.
Wenn das Resultat grammatisch ist und dieselbe Bedeutung wie der ursprüngliche Satz hat, dann handelt es sich um eine Angabe und nicht um eine Ergänzung.
In (\ref{ex:ppergaenzungenundppangaben192}) werden einige solcher Tests durchgeführt.

\begin{exe}
  \ex\label{ex:ppergaenzungenundppangaben192}
  \begin{xlist}
    \ex[*]{\label{ex:ppergaenzungenundppangaben193} Viele Menschen leiden.
      Dies geschieht unter Vorurteilen.}
    \ex[ ]{\label{ex:ppergaenzungenundppangaben194} Viele Menschen schwitzen.
      Dies geschieht unter Sonnenschirmen.}
    \ex[*]{\label{ex:ppergaenzungenundppangaben195} Mausi schickt einen Brief.
      Dies geschieht an ihre Mutter.}
    \ex[*]{\label{ex:ppergaenzungenundppangaben196} Mausi befindet sich.
      Dies geschieht in Hamburg.}
    \ex[?]{\label{ex:ppergaenzungenundppangaben197} Mausi liegt.
      Dies geschieht auf dem Bett.}
  \end{xlist}
\end{exe}

Im Fall von \textit{liegen} (\textit{auf, in, \ldots} mit Dativ) in (\ref{ex:ppergaenzungenundppangaben197}) ist das Ergebnis schwierig zu bewerten.
Ob der Satz wirklich so akzeptiert würde (und \textit{auf dem Bett} damit eine Angabe wäre), kann jeder Sprecher für sich entscheiden.
Trotz der umstrittenen Qualität des Testes ist er das zuverlässigste Kriterium, das existiert.

\Zusammenfassung{%
Dative, mit denen ein \textit{bekommen}-Passiv gebildet werden können (einschließlich Pertinenzdativ und Nutznießerdativ), sind keine Angaben.
PPs mit einer nicht vom Verb zugewiesenen Rolle sind Angaben und können mit dem Paraphrasentest aus dem Satz extrahiert werden.
}

% \section{Bindung}
% \label{sec:bindung}
% 
% In diesem Abschnitt wird eine besondere Relationen zwischen Ergänzungen und Angaben von Verben besprochen, nämlich die sogenannte \textit{Bindung}.
% Die Bindungsrelation spielte vor allem in den 1980er Jahre in der Syntaxtheorie eine große Rolle, wird inzwischen aber tendentiell auch sehr stark als semantisches Phänomen betrachtet.
% Hier wird nur der minimale Bereich der Bindungstheorie besprochen, der die Ergänzungen von Verben betrifft.
% Es geht dabei um die möglichen Interpretation von pronominalen (insbesondere auch reflexiven) und nicht-pronominalen NPs.
% Die Beispiele in (\ref{ex:bindung198}) benutzen die Markierung korreferenter NPs, wie sie in Abschnitt~\ref{sec:person} eingeführt wurde.
% Kurzgefasst sollen in diesen Sätzen NPs mit derselben Indexzahl so gelesen werden, dass sie dasselbe Ding bezeichnen.
% Wenn in einem Satz \textit{Mausi$_1$} und \textit{sie$_1$} stehen, dann soll der Satz so verstanden werden, dass mit \textit{sie$_1$} auch Mausi gemeint ist.
% 
% \begin{exe}
%   \ex\label{ex:bindung198}
%   \begin{xlist}
%     \ex[ ]{\label{ex:bindung199} Mausi$_1$ sieht sich$_1$.}
%     \ex[ ]{\label{ex:bindung200} Mausi$_1$ sieht sie$_2$.}
%     \ex[ ]{\label{ex:bindung201} Mausi$_1$ sieht Frida$_2$.}
%     \ex[*]{\label{ex:bindung202} Mausi$_1$ sieht sich$_2$.}
%     \ex[*]{\label{ex:bindung203} Mausi$_1$ sieht sie$_1$.}
%     \ex[*]{\label{ex:bindung204} Mausi$_1$ sieht Frida$_1$.}
%   \end{xlist}
% \end{exe}
% 
% Der Befund in (\ref{ex:bindung198}) deutet auf wichtige grammatische Eigenschaften von verschiedenen Pronomina und nicht-pronominalen NPs hin.
% Ein sogenanntes Reflexivpronomen wie \textit{sich} kann in Objektposition nur so verstanden werden, dass es dasselbe Ding bezeichnet wie das Subjekt oder ein anderes Objekt, in (\ref{ex:bindung199}) also Mausi.
% Würde man versuchen, mit \textit{sich} in diesem Satz nicht Mausi zu bezeichnen, gelänge dies nicht, wie das Grammatikalitätsurteil bei (\ref{ex:bindung202}) anzeigt.
% Wird ein normales Personalpronomen (\textit{sie}) -- das wir für diesen Zweck ein \textit{freies Pronomen} nennen -- in derselben Position verwendet wie in (\ref{ex:bindung200}), dann muss es ein anderes Ding (hier eine andere Person) als das Subjekt bezeichnen.
% Mit \textit{sie} in diesem Satz Mausi zu bezeichnen, funktioniert nicht, s.\ (\ref{ex:bindung203}).
% Etwas trivial ist schließlich die Feststellung zu (\ref{ex:bindung201}) und (\ref{ex:bindung204}), dass man \zB mit einem Eigennamen immer genau den Gegenstand bezeichnet, den man eben damit bezeichnet.
% Mit \textit{Frida} kann also innerhalb eines Satzes niemals derselbe Gegenstand gemeint sein wie mit \textit{Mausi}.
% Ausnahmen ergeben sich marginal, wenn eine Person mit zwei Namen bezeichnet wird.
% Auch dann sind die entsprechenden Sätze eher als Sprachspiele zu bewerten, wie \zB (\ref{ex:bindung205}).
% 
% \begin{exe}
%   \ex{\label{ex:bindung205} Lemmy sieht jeden morgen Ian Kilmister im Spiegel.}
% \end{exe}
% 
% Wenn eine NP die Bedeutung für eine pronominale NP liefert (\zB \textit{Mausi} für \textit{sich}), liegt eine Bindungsrelation vor:
% Die erste NP \textit{bindet} dann die zweite, vgl.\ Definition~\ref{def:bindung}.
% 
% \Definition{Bindung}{\label{def:bindung}%
% Eine NP\Sub{1} \textit{bindet} eine andere NP\Sub{2}, wenn NP\Sub{2} ihre Bedeutung (auch: Referenz) vollständig von NP\Sub{1} übernimmt.
% \index{Bindung}
% }
% 
% \index{Pronomen!reflexiv}
% 
% Interessant ist nun die Frage, in welchen syntaktischen Konstellationen eine Einheit die andere binden kann oder sogar muss.
% Dass Eigennamen bzw.\ referierende Ausdrücke niemals gebunden sein können, geht aus dem oben Gesagten bereits hervor.
% Die Konstellationen, in denen freie Pronomina nicht gebunden sein können, aber Reflexivpronomina gebunden sein müssen, sind diejenigen, in denen das zu bindende Pronomen die obliquere Ergänzung eines Verbs als der potentielle Binder ist.
% Bindung geht also immer abwärts von der strukturelleren zur obliqueren Ergänzung, ganz prototypisch von der Nominativ-Ergänzung zu den Ergänzungen im Akkusativ und Dativ.
% Für Nominativ und Akkusativ wurde dies bereits gezeigt.
% Ein Reflexivpronomen im Akkusativ bei einem transitiven Verb wird immer gebunden vom Nominativ desselben Verbs, vgl.\ in (\ref{ex:bindung199}) vs.\ (\ref{ex:bindung202}).
% Ein freies Pronomen im Akkusativ kann aber niemals vom Nominativ gebunden werden, vgl.\ (\ref{ex:bindung203}) vs.\ (\ref{ex:bindung200}).
% Für Dative wird dasselbe in (\ref{ex:bindung206}) gezeigt.
% 
% \begin{exe}
%   \ex\label{ex:bindung206}
%   \begin{xlist}
%     \ex[ ]{\label{ex:bindung207} Frida$_1$ dankt sich$_1$.}
%     \ex[*]{\label{ex:bindung208} Frida$_1$ dankt sich$_2$.}
%     \ex[*]{\label{ex:bindung209} Frida$_1$ dankt ihr$_1$.}
%     \ex[ ]{\label{ex:bindung210} Frida$_1$ dankt ihr$_2$.}
%   \end{xlist}
% \end{exe}
% 
% Sobald die potentiell gebundene Phrase nicht mehr vom selben Verb abhängt wie der potentielle Binder, weil sie \zB Teil eines Objektsatzes ist, verhalten sich die Pronomina wie in unabhängigen Sätzen, vgl.\ (\ref{ex:bindung211}).
% Das freie Nominativ-Pronomen des Objektsatzes (\ref{ex:bindung212}) und das freie Akkusativ-Pronomen an derselben Position (\ref{ex:bindung213}) können ohne Weiteres vom Subjekt des Matrixsatzes gebunden werden.
% Ein Reflexivpronomen im Akkusativ kann aber, wenn es im Objektsatz steht, nicht vom Subjekt des Matrixsatzes gebunden werden (\ref{ex:bindung214}).
% Es ist sozusagen aus dem Bereich herausgefallen, der für die Bindung eines Reflexivpronomens zulässig ist.
% 
% \begin{exe}
%   \ex\label{ex:bindung211}
%   \begin{xlist}
%     \ex[ ]{\label{ex:bindung212} Frida$_1$ weiß, dass sie$_1$ glücklich ist.}
%     \ex[ ]{\label{ex:bindung213} Frida$_1$ weiß, dass [eine Kollegin]$_2$ sie$_1$ sehen kann.}
%     \ex[*]{\label{ex:bindung214} Frida$_1$ weiß, dass [eine Kollegin]$_2$ sich$_1$ sehen kann.}
%   \end{xlist}
% \end{exe}
% 
% Satz~\ref{satz:bindung} fasst die rein syntaktische Bindungstheorie zusammen.
% Man spricht bei den drei Prinzipien auch von \textit{Prinzip A}, \textit{Prinzip B} und \textit{Prinzip C} der Bindungstheorie.
% 
% \Satz{Syntaktische Bindung}{\label{satz:bindung}%
% Syntaktisch werden mögliche Bindungsrelationen zwischen Ergänzungen eines Verbs durch die folgenden Prinzipien eingeschränkt:
% \begin{enumerate}
%   \item Reflexivpronomina müssen von einer weniger obliquen Ergänzung desselben Verbs gebunden werden.
%   \item Freie Pronomina dürfen nicht von einer weniger obliquen Ergänzung desselben Verbs gebunden werden, müssen aber von einem anderen Binder (ggf. außerhalb des Satzes) gebunden werden.
%   \item Referierende Ausdrücke sind nie gebunden.
% \end{enumerate}
% \index{Bindungstheorie}
% }
% 
% \Zusammenfassung{%
% Die Bindungstheorie beschreibt die Beschränkungen der Interpretationsmöglichkeiten von normalen Pronomina und Reflexivpronomina in bestimmten syntaktischen Strukturen.
% }

