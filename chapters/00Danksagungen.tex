\addchap{\lsAcknowledgementTitle} 

Um nicht über die Jahre einen unüberschaubaren Text mit Dankesworten wuchern zu lassen, fasse ich mich hier kurz und danke den vielen Personen, die zu Inhalt, Form und Erfolg dieses Buchs von der ersten bis zur aktuellen Auflage beigetragen haben, und zwar in Gruppen und alphabetisch.
Der Dank soll dadurch jedoch nicht weniger herzlich sein, als würde er von den blumigsten Worten begleitet.
Selbstverständlich liegt die Verantwortung für alle Fehler und Unangemessenheiten in diesem Buch bei mir.

Auf Seiten von Language Science Press danke ich Eric Fuß, Martin Haspelmath, Felix Kopecky und Sebastian Nordhoff.

Dozenten und Kollegen, denen ich meinen größten Dank für verschiedensten Beiträge zum Buch aussrechen möchte, sind Tim Felix Aufderheide, Felix Bildhauer, Michael Job, Götz Keydana, Bjarne Ørsnes, Andreas Pankau, Elizabeth Pankratz, Hans-Joachim Particke, Nicolai Sinn und viele andere, die ich vermutlich vergessen habe.

Kim Maser, Luise Rißmann, Anna Wehr haben als Hilfskräfte und Tutorinnen viel Wichtiges beigetragen, und ihnen gilt dafür mein größter Dank!

Unter den vielen Studenten, die durch Rückmeldungen und Diskussionen geholfen haben, das Buch zu verbessern, möchte ich besonders Sarah Dietzfelbinger, Ana Draganovi\'{c}, Sandra Goldbach, Lea Helmers, Hanin Ibrahim, Geza Lebro, Theresia Lehner, Kaya Triebler, Sydnes Tu, Samuel Reichert, Johanna Rehak, Aleksandr Schamberger, Cynthia Schwarz und Jana Weiß danken.

Begleiter meines Privatlebens, die direkt oder indirekt meine Arbeit an diesem Buch unterstützt haben, und denen ich dafür herzlich danken möchte, sind Thea Dittrich, Matthias B.\ Döring, Julia Schmidt.

Für die grundlegende Inspitation, Linguist zu werden, danke ich wie bereits zur ersten Auflage meinen Lehrern in Japanologie, Thomas M.\ Groß und Iris Hasselberg.

Schließlich möchte ich zwei Personen individuell danken, die auf herausragende Weise an dem Erfolg des Buchs und an meinem persönlichen Erfolg als Linguist beteiligt sind. Stefan Müller hat nicht nur dafür gesorgt, dass ich große Teile der Zeit von 2007 bis 2022 gut bezahlte Positionen mit vertretbarem Lehrdeputat innehaben durfte, sondern er hat mir auch auf diesen Positionen stets die Freiheiten eingeräumt, die ich brauchte, um meine eigenen Forschungsschwerpunkte -- bis hin zu meiner erfolgreichen Habilitation -- zu entwickeln und Projekte wie dieses Buch zu verwirklichen. Durch seine Mühen mit der Gründung und Etablierung von Language Science Press und der Herausgabe der \textit{Textbooks}-Reihe hat er meinem Buch zudem eine perfekte verlegerische Heimat gegeben. Herzlichen Dank daher an Stefan Müller!

Ulrike Sayatz danke ich aufs Herzlichste für die gemeinsame Entwicklung und Durchführung von außergewöhnlichen und erfolgreichen Forschungsvorhaben (und die starken Nerven, die das mit mir gelegentlich erfordert), die vielen Hinweise und Diskussionen zu diesem Buch, das geduldige Beharren auf der eigentlich trivialen Tatsache, dass die Graphematik zur Linguistik gehört und die moralische Unterstützung in anstrengenden Jahren und absurden institutionellen Kontexten.
