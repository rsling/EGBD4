\chapter{Lautsystem}
\label{sec:phonologie}

Die im letzten Kapitel besprochene artikulatorische Phonetik lieferte eine Beschreibung der physiologischen Grundlagen der Sprachproduktion.
Anhand des Vorrats an Zeichen im IPA-Alphabet haben wir außerdem definiert, welche Laute im in Deutschland gesprochenen Standarddeutschen vorkommen.
Die eigentliche Frage der systematischen Grammatik bezüglich der Lautgestalt von Wörtern und größeren Einheiten ist aber, nach welchen Regularitäten die Segmente verbunden werden, und welchen Stellenwert die einzelnen Segmente und Segmentverbindungen (wie \zB Silben) im gesamten Lautsystem haben.
In der Phonologie geht es daher um das \textit{Lautsystem} und seine Regularitäten.
In Abschnitt~\ref{sec:segmente} wird der Status einzelner Laute und ihrer Vorkommen behandelt.
Es wird diskutiert, wie Laute im Lexikon gespeichert werden können, und schließlich werden einige konkrete phonologische Strukturbedingungen des Deutschen (wie die Endrand-Desonorisierung) systematisch dargestellt.
Dann folgt eine recht ausführliche Analyse des \textit{Silbenbaus} (Abschnitt~\ref{sec:silbenundwoerter}).
Abschließend gibt Abschnitt~\ref{sec:wortakzent} einen Einblick in die \textit{Prosodie} (die \textit{Betonungslehre}) und damit in phonologische Aspekte auf Wortebene.

\section{Segmente}
\label{sec:segmente}

\subsection{Segmente und Verteilungen}
\label{sec:segmenteundverteilungen}

Der zentrale Begriff in der Phonologie ist zunächst wie in der Phonetik der des \textit{Segments}, vgl.\ Definition~\ref{def:segment}.\index{Segment}
Alternativ findet man auch den Begriff des \textit{Phonems}, auf den in Vertiefung~\ref{vert:phonephoneme} kurz eingegangen wird.
Allerdings geht es in der Phonologie anders als in der Phonetik um den systematischen Stellenwert der Segmente, nicht um eine reine Beschreibung ihrer Lautgestalt.
Um sich den Übergang von der Phonetik zur Phonologie klar zu machen, ist der Begriff der \textit{Verteilung} hilfreich.\index{Verteilung}
Schon in Abschnitt~\ref{sec:endranddesonorisierung1} wurde diskutiert, dass es bestimmte Positionen im Wort und in der Silbe gibt, an denen nur bestimmte Segmente vorkommen.
In jenem Abschnitt ging es zunächst lediglich um die Illustration einiger Beziehungen zwischen Schrift und Phonetik, aber in der Phonologie sind solche Phänomene von hohem theoretischen Stellenwert.
Das Beispiel war die Endrand-Desonorisierung, die dazu führt, dass in der letzten Position der Silbe Obstruenten immer stimmlos sind (\textit{Bad} als [baːt] und nicht *[baːd]).
Man muss nun aber dennoch davon ausgehen, dass die betreffenden Wörter systematisch gesehen -- und vor allem im Lexikon -- einen stimmhaften Plosiv an der entsprechenden Stelle enthalten, denn wenn (\zB in Flexionsformen) ein weiterer Vokal folgt, ist der Plosiv stimmhaft, vgl.\ \textit{Bades} [baːdəs] nicht *[baːtəs].
Ausgehend von dem Begriff der \textit{Verteilung} oder \textit{Distribution} (Definition~\ref{def:verteilung}) kann man in der Phonologie systematisch über solche Phänomene sprechen.

\Definition{Verteilung (Distribution)}{\label{def:verteilung}%
Die \textit{Verteilung} eines Segments ist die Menge der Umgebungen, in denen es vorkommt.
\index{Verteilung}
}


Die Beschreibung der Verteilung eines Segments nimmt typischerweise Bezug auf bestimmte Positionen in der Silbe oder im Wort, oder auf Positionen vor oder nach anderen Segmenten.
Es stellt sich damit die entscheidende Frage, ob zwei Segmente die \textit{gleiche Verteilung} oder eine \textit{teilweise} bzw.\ \textit{vollständig unterschiedliche Verteilung} haben.
Die Beispiele in (\ref{ex:segmenteundverteilungen001})--(\ref{ex:segmenteundverteilungen005}) illustrieren drei Typen von Verteilungen anhand des Vergleiches von je zwei Segmenten.


\begin{exe}
  \ex\label{ex:segmenteundverteilungen001}
    \begin{xlist}
      \ex{\label{ex:segmenteundverteilungen002} Tod [toːt], Kot [koːt]}
      \ex{\label{ex:segmenteundverteilungen003} Schott [ʃɔt], Schock [ʃɔk]}
    \end{xlist}
  \ex{\label{ex:segmenteundverteilungen004} Hang [haŋ], *[ŋah]}
  \ex\label{ex:segmenteundverteilungen005}
    \begin{xlist}
      \ex{\label{ex:segmenteundverteilungen006} Sog [zoːk], besingen [bəzɪŋən], *[soːk]}
      \ex{\label{ex:segmenteundverteilungen007} fließ [fliːs], Boss [bɔs], *[fliːz]}
      \ex{\label{ex:segmenteundverteilungen008} heißer [ha͡ɛsɐ], heiser [ha͡ɛzɐ], Base [baːzə], Basse [basə], *[bazə]}
    \end{xlist}
\end{exe}


(\ref{ex:segmenteundverteilungen001}) zeigt, dass [t] und [k] eine (bezüglich ihrer Positionen in der Silbe) vollständig übereinstimmende Verteilung haben.
Sie kommen beide am Anfang und am Ende von Silben vor.
Hingegen haben [h] und [ŋ] eine vollständig unterschiedliche  Verteilung, wie (\ref{ex:segmenteundverteilungen004}) zeigt.
[h] kommt nur am Anfang einer Silbe vor, [ŋ] kommt nur am Ende einer Silbe vor.
Die Beispiele in (\ref{ex:segmenteundverteilungen005}) demonstrieren, dass [s] und [z] eine teilweise übereinstimmende Verteilung haben.
[z] kann am Anfang der ersten Silbe eines Wortes stehen, aber [s] kann dies nicht, wie (\ref{ex:segmenteundverteilungen006}) zeigt.
[s] kann dafür am Ende der letzten Silbe eines Wortes stehen, was [z] nicht kann, wie in (\ref{ex:segmenteundverteilungen007}) demonstriert.
Beide können am Anfang einer Silbe in der Wortmitte stehen, [z] aber nur nach langem Vokal oder Diphthong wie in (\ref{ex:segmenteundverteilungen008}).

Wie man an den Beispielen sieht, gibt es Paare von Segmenten, anhand derer Wörter (wie \textit{heißer} und \textit{heiser}) unterschieden werden können, auch wenn die Wörter ansonsten völlig gleich lauten.
Dies funktioniert genau deshalb, weil die zwei Segmente jeweils mindestens eine teilweise übereinstimmende Verteilung haben.
Zwei Wörter, die sich nur in einem Segment an derselben Position unterscheiden, nennt man \textit{Minimalpaar}, und ein Minimalpaar illustriert jeweils einen \textit{phonologischen Kontrast}, s.\ Definition~\ref{def:phokonseg}.\index{Minimalpaar}

\Definition{Phonologischer Kontrast}{\label{def:phokonseg}%
Zwei phonetisch unterschiedliche Segmente bzw.\ Merkmale stehen in einem \textit{phonologischen Kontrast}, wenn sie eine teilweise oder vollständig übereinstimmende Verteilung haben und dadurch einen lexikalischen bzw.\ grammatischen Unterschied markieren können.
\index{Kontrast}
}

Ein phonologischer Kontrast besteht \zB zwischen [t] und [k], weil wir Wörter anhand dieser Segmente unterscheiden können.
Das Gleiche gilt für [s] und [z] und viele andere Paare von Segmenten.
Es gilt aber nicht für [h] und [ŋ], weil diese beiden Segmente keine übereinstimmende Verteilung haben, wie in (\ref{ex:segmenteundverteilungen004}) gezeigt wurde.
Diese Art der Verteilungen nennt man \textit{komplementär}, vgl.\ Definition~\ref{def:komplementaer}.

\Definition{Komplementäre Verteilung}{\label{def:komplementaer}%
Eine \textit{komplementäre Verteilung} zweier Segmente liegt dann vor, wenn die beiden Segmente in keiner gemeinsamen Umgebung vorkommen.
Komplementär verteilte Segmente können prinzipiell keinen phonologischen Kontrast markieren.
\index{Verteilung!komplementär}
}

Über Verteilungen lässt sich schon anhand des bisher eingeführten Inventars von Beispielen noch mehr sagen.
Bei der bereits besprochenen Endrand-Desonorisierung haben wir es mit Paaren von stimmlosen und stimmhaften Plosiven zu tun, die in bestimmten Umgebungen (im Silbenanlaut) einen Kontrast markieren, der aber in anderen Umgebungen (Silbenauslaut) verschwindet.\index{Endrand!Desonorisierung}
(\ref{ex:segmenteundverteilungen009})--(\ref{ex:segmenteundverteilungen011}) zeigen dies für [g] und [k], [d] und [t] sowie [b] und [p].

\begin{exe}
  \ex\label{ex:segmenteundverteilungen009}
  \begin{xlist}
    \ex{Weg [veːk], Weges [veːgəs]}
    \ex{Bock [bɔk], Bockes [bɔkəs]}
  \end{xlist}
  \ex\label{ex:segmenteundverteilungen010}
  \begin{xlist}
    \ex{Bad [baːt], Bades [baːdəs]}
    \ex{Blatt [blat], Blattes [blatəs]}
  \end{xlist}
  \ex\label{ex:segmenteundverteilungen011}
  \begin{xlist}
    \ex{Lob [loːp], Lobes [loːbəs]}
    \ex{Depp [dɛp], Deppen [dɛpən]}
  \end{xlist}
\end{exe}


Im Silbenauslaut des Deutschen gibt es prinzipiell keinen Unterschied zwischen stimmlosen und stimmhaften Plosiven.
Solche Effekte nennt man \textit{Neutralisierungen}, s.\ Definition~\ref{def:neutralisierung}.

\Definition{Neutralisierung}{\label{def:neutralisierung}%
Eine \textit{Neutralisierung} ist die Aufhebung eines phonologischen Kontrasts in einer bestimmten Position.
\index{Neutralisierung}
}

Im Silbenauslaut wird im Deutschen also der phonologische Kontrast zwischen [g] und [k], zwischen [d] und [t] usw.\ neutralisiert.
Allgemein gesprochen wird der Kontrast zwischen stimmlosen und stimmhaften Plosiven (vgl.\ Abschnitt~\ref{sec:stimmhaftigkeit}) in dieser Position neutralisiert.
Daher ist in Definition~\ref{def:phokonseg} von zwei phonetisch unterschiedlichen Segmenten bzw.\ \textit{Merkmalen} die Rede.
Phonologische Kontraste bestehen im Prinzip zwischen Merkmalen und erst in zweiter Ordnung zwischen ganzen Segmenten.

Das Feststellen von Verteilungen ist allerdings kein Selbstzweck.
Durch die Untersuchung aller Verteilungen in einer Sprache konstruiert man das phonologische System (die phonologische Komponente der Grammatik).
Dabei geht es darum, die Formen zu ermitteln, die im Lexikon gespeichert werden müssen, und die Strukturbedingungen (wie die Endrand-Desonorisierung) zu beschreiben, an die die Segmente in diesen Formen ggf.\ angepasst werden müssen.
Die \textit{lexikalisch gespeicherten} bzw.\ \textit{zugrundeliegenden Formen} und die \textit{phonologischen Strukturbedingungen} produzieren die konkreten phonetischen Verteilungen an der Oberfläche.

\subsection{Zugrundeliegende Formen und Strukturbedingungen}
\label{sec:zugrundeliegendeformenundstrukturbedingungen}

Wir bleiben jetzt beim Beispiel der Endrand-Desonorisierung, um die Idee von lexikalisch zugrundeliegenden Formen und phonologischen Strukturbedingungen einzuführen.
Die Endrand-Desonorisierung hat wie erwähnt zur Folge, dass für Obstruenten im Silbenauslaut der Stimmtonkontrast neutralisiert wird, denn alle Obstruenten im Silbenauslaut sind stimmlos.
Wenn man das gesamte Paradigma der Wörter in (\ref{ex:segmenteundverteilungen009})--(\ref{ex:segmenteundverteilungen011}) ansieht, fällt aber dennoch ein bedeutender Unterschied auf.
In manchen Wörtern steht im Silbenauslaut ein Konsonant, der in anderen Umgebungen stimmhaft ist, wie in [veːk] und [veːgəs].
In anderen Wörtern steht ein stimmloser Konsonant, der auch in diesen anderen Umgebungen stimmlos bleibt, wie in [bɔk] und [bɔkəs].
Es ist daher naheliegend, anzunehmen, dass Wörter wie \textit{Weg} (oder \textit{Bad}, \textit{Lab} usw.) eine \textit{zugrundeliegende Form} haben, in der der letzte Obstruent stimmhaft ist.
Diese zugrundeliegende Form ist eine der wesentlichen Informationen, die zum \textit{lexikalischen Wort} gehören (vgl.\ Abschnitt~\ref{sec:woerterundwortformen}).

Die eigentliche Grammatik stellt allerdings allgemeine Anforderungen an die Wohlgeformtheit von Strukturen, hier die \textit{phonologischen Strukturbedingungen}.
Der \textit{Prozess} der Endrand-Desonorisierung (als Veränderung der Merkmale eines Segments) ist in diesem Sinn das Ergebnis einer Anpassung von Silben an die Strukturbedingung, dass Silben nicht auf stimmhafte Obstruenten enden können.%
\footnote{Man kann die phonologische Grammatik in Form von \textit{Prozessen} bzw.\ \textit{Regeln} (im technischen Sinne) formulieren, die Formen als Eingabematerial nehmen und modifiziert als Ausgabematerial wieder ausgeben.\index{phonologischer Prozess}
Die Endrand-Desonorisierung wäre dann einfach eine Regel in diesem technischen Sinn.
Alternativ kann man davon ausgehen, dass eine phonologische Grammatik aus Beschreibungen zulässiger Strukturen besteht, an die konkrete Formen angepasst werden.
Wie diese Anpassung vor sich geht, ist auch wieder eine sehr technische Frage.
Innerhalb einer phonembasierten Theorie (Vertiefung~\ref{vert:phonephoneme}) bieten sich wieder andere Möglichkeiten, die Beziehung von Formen und Strukturbedingungen zu erfassen.
Die technischen Unterschiede sind für unsere Zwecke mehr als nachrangig.
Eine deskriptive Grammatik ist wahrscheinlich am besten bedient, wenn sie sich darauf beschränkt, zu beschreiben, wie Formen im Lexikon und an der Oberfläche aussehen, also systematische Beziehungen -- eben \textit{Regularitäten} (Abschnitt~\ref{sec:regelregularitaetundgeneralisierung}) -- feststellt.}
Die zugrundeliegende Form muss also genau die Informationen zu einem Wort enthalten, die ausreichen, um seine lautliche Gestalt in allen möglichen Formen und Umgebungen ableiten zu können.
Definition~\ref{def:pholproz} fasst zusammen.

\Definition{Zugrundeliegende Form}{\label{def:pholproz}%
Die \textit{zugrundeliegende Form} (eines Wortes) ist genau die Folge von Segmenten, die im Lexikon gespeichert wird, und auf die alle zugehörigen phonetischen Formen zurückgeführt werden können.
Die Formen werden ggf. an die phonologischen \textit{Strukturbedingungen} (die Regularitäten der phonologischen Grammatik) angepasst.
\index{zugrundeliegende Form}
\index{Strukturbedingung}
\index{Lexikon}
}

Neben der Endrand-Desonorisierung ist ein anderes illustratives Beispiel für zugrundeliegende Formen und Strukturbedingungen die Einfügung des Glottalplosivs.\index{Glottalplosiv}
Wie in Abschnitt~\ref{sec:laryngale} bereits besprochen, steht im Deutschen am Wortanfang und vor betonten Silben innerhalb von Wörtern stets ein Konsonant.
In scheinbar vokalisch anlautenden Wörtern wie \textit{Ort} oder \textit{Insel} wird der laryngale Plosiv oder Glottalplosiv [ʔ] eingefügt.
Man artikuliert [ʔɔ͡ət] und [ʔɪnzəl].
Ein Beispiel für dasselbe Phänomen vor einer betonten Silbe innerhalb eines Wortes ist das Wort \textit{Verein}, das [fɐʔa͡ɛn] artikuliert wird.
Wir haben es also mit einer Strukturbedingung für die Form von Silben und Wörtern zu tun.
Zugrundeliegend muss [ʔ] damit nicht spezifiziert werden, weil nur durch seine An- bzw.\ Abwesenheit niemals zwei Wörter unterschieden werden können.
Es gibt also aus systematischen Gründen keine Minimalpaare.
\textit{Asche} [ʔaʃə] und \textit{Tasche} [taʃə] sind zwei verschiedene Wörter und ein Minimalpaar.
Weil die Anwesenheit des Glottalplosivs aber vollständig vorhersagbar ist und er in den Umgebungen, in denen er auftritt, nicht weggelassen werden kann, ist *[aʃə] unmöglich.
Genau deswegen bilden *[aʃə] und [ʔaʃə] auch kein Minimalpaar.

Eine andere Art der Reduktion wird später für auslautendes [ŋ] vorgenommen.
Einerseits ist [ŋ] die Vertretung für [n] vor velaren Plosiven wie in \textit{Bänke} [bɛŋkə].
In diesen Fällen liegt es nah, davon auszugehen, dass sich der Nasal an den Plosiv in seinem Artikulationsort anpasst.
Andererseits tritt das Segment auch einzeln am Silbenende auf, wie in \textit{Gang} [gaŋ].
Man kann [ŋ] auch in diesen Fällen phonologisch auf eine zugrundeliegende Folge von [n] und [g] zurückführen (s.\ Abschnitt~\ref{sec:diesystematikderraender}).

Die Phonologie stellt also eine Abstraktion gegenüber der Phonetik dar.\index{Phonologie}
Die Phonetik eines Wortes beschreibt, wie es tatsächlich ausgesprochen wird, und jedes einzelne Wort einer Sprache kann ohne Betrachtung der anderen Wörter phonetisch beschrieben werden.
Die phonologische Repräsentation eines Wortes erfordert hingegen zusätzliches Wissen um Strukturbedingungen (\zB in Form der Endrand-Desonorisierung), um aus ihr phonetische Formen abzuleiten.
Dieses Wissen erschließt sich durch die Betrachtung des gesamten Sprachsystems, also jedes Wortes in Bezug zu allen anderen Wörtern und in allen möglichen Umgebungen.
Anders gesagt müssen die \textit{Verteilungen der Segmente} bekannt sein.

\begin{table}[!htbp]
  \resizebox{\textwidth}{!}{
    \begin{tabular}{ccc}
      \lsptoprule
      \multicolumn{2}{c}{\textbf{Grammatik}} & \textbf{Externe Systeme} \\
      \midrule
      \textbf{Lexikon} & \textbf{Phonologie} & \textbf{Phonetik} \\
      \midrule
      /~/& $\Rightarrow$ & [~]\\
      zugrundeliegende Form & Anpassung an Strukturbedingungen & phonetische Realisierung \\
      \lspbottomrule
    \end{tabular}
  }
  \caption{Lexikon, Phonologie und Phonetik}
  \label{tab:zugrundeliegendeformenundstrukturbedingungen012}
\end{table}
\index{Lexikon}

Zugrundeliegende phonologische Formen schreibt man konventionellerweise nicht in [~], sondern in /~/, also \zB /veg/, /bad/ und /lab/ oder /ɔʁt/ und \mbox{/ɪnzəl/}.%
\footnote{Warum die Länge in /~/ nicht notiert wird, wird in Abschnitt~\ref{sec:gespanntheitbetonungundlaenge} erläutert.}
Schematisch kann man die Verhältnisse wie in Tabelle~\ref{tab:zugrundeliegendeformenundstrukturbedingungen012} darstellen.
Mit \textit{externen Systemen} sind nicht zur Grammatik gehörige Systeme wie Gehör und Sprechapparat gemeint.
In den Abschnitten~\ref{sec:endranddesonorisierung} bis~\ref{sec:vokalisierungen} werden beispielhaft einige Strukturbedingungen und Verteilungen besprochen, um zu illustrieren, wie ein phonologisches System konstruiert werden kann.
Dabei ist es manchmal nicht trivial, zu entscheiden, ob bestimmte Repräsentationen besser in /~/ oder [~] stehen sollten.
Wir tendieren dazu, [~] im Zweifelsfall den Vorzug zu geben.

\subsection{Endrand-Desonorisierung}
\label{sec:endranddesonorisierung}

\index{Endrand!Desonorisierung}

Die Endrand-Desonorisierung lässt sich als Strukturbedingung unter Bezug auf phonetische bzw.\ phonologische Merkmale (Abschnitt~\ref{sec:phonetischemerkmale}), bestimmte Positionen in Wort oder Silbe und die Oberklassen für Segmente (Abschnitt~\ref{sec:oberklassenfuerartikulationsarten}) sehr einfach und kompakt mit Satz~\ref{satz:auslautverhaertung} beschreiben.

\Satz{Endrand-Desonorisierung}{\label{satz:auslautverhaertung}Alle Segmente mit [\textsc{Obstruent}:~$+$] sind am Silbenende [\textsc{Stimme}:~$-$].}

Mit "`alle Segmente"' ist hier gemeint, dass auch in Abfolgen von mehreren Obstruenten am Silbenende alle stimmlos werden.
Obwohl in \textit{Bads} das /d/ also nicht ganz am Ende der Silbe, sondern in der Obstruenten-Abfolge /ds/ steht, wird es stimmlos, und die Form lautet daher [baːts].
Wenn wir zugrundeliegende Formen an diese Bedingung anpassen wollen, muss also die Silbenstruktur bekannt sein.
Um diese geht es in Abschnitt~\ref{sec:silben} noch im Detail, hier werden die Silbengrenzen einfach vorgegeben und durch Punkte markiert.
Nur zur Veranschaulichung steht \phopro\ für \textit{wird phonetisch  realisiert als}.%
\footnote{In (\ref{ex:endranddesonorisierung014}) ist \textit{Bad} standardkonform mit langem [aː] notiert.
Die Variante mit kurzem [a] (also [bat]) ist regional.}

\begin{exe}
  \ex\label{ex:endranddesonorisierung013}
  \begin{xlist}
    \ex{\label{ex:endranddesonorisierung014} /bad/ \phopro [baːt]}
    \ex{\label{ex:endranddesonorisierung015} /badəs/ \phopro [baː.dəs]}
    \ex{\label{ex:endranddesonorisierung016} /bat/ \phopro [baːt]}
  \end{xlist}
\end{exe}

Abhängig von der zugrundeliegenden Form und der Silbenstruktur muss eine Veränderung stattfinden -- oder eben nicht.
In (\ref{ex:endranddesonorisierung014}) steht /d/ am Silbenende und ist zugrundeliegend mit [\textsc{Stimme}: $+$] spezifiziert.
Weil /d/ den Wert [\textsc{Obstruent}: $+$] hat, wird der Wert des Stimmton-Merkmals auf [\textsc{Stimme}: $-$] gesetzt.
In (\ref{ex:endranddesonorisierung015}) ist die Silbenstruktur anders, die Bedingung für die Endrand-Desonorisierung ist nicht erfüllt, und die Form bleibt unverändert.
In (\ref{ex:endranddesonorisierung016}) steht zwar ein Obstruent /t/ am Silbenende, aber es muss keine Anpassung stattfinden, weil /t/ von vornherein [\textsc{Stimme}: $-$] ist.

\subsection{Gespanntheit, Betonung und Länge}
\label{sec:gespanntheitbetonungundlaenge}

\index{Vokal!Länge}
\index{Vokal!Gespanntheit}
Die Formulierung von Strukturbedingungen kann helfen, die Menge der Merkmale zu reduzieren, die man zugrundeliegend spezifizieren muss.
Anders gesagt kann man sich überlegen, ob die Werte für bestimmte Merkmale automatisch aus anderen Merkmalen und den Positionen der jeweiligen Segmente vorhergesagt werden können.
Solche Reduktionen sind typisch für die Phonologie im Gegensatz zur Phonetik, weil eine einfache Systembeschreibung aus allgemeinen wissenschaftlichen Ökonomiegründen einer komplexeren vorzuziehen ist.

In Abschnitt~\ref{sec:phonetischemerkmale} wurde die Vokallänge als gewöhnliches Merkmal (\textsc{Lang}) eingeführt.
Gleichzeitig wurde festgestellt, dass nur die Vokale [i], [y], [u], [e], [ø], [ɛ], [o] und [a] lange und kurze Varianten haben.
Bezüglich der Akzentuierung bzw.\ Betonung ist ebenfalls bereits bekannt, dass alle Vokale bis auf [ə] und [ɐ] betonbar sind, und dass bei den Vokalen mit Längenunterschied die Länge an die Betonung gebunden ist.
Dieser Abschnitt verfolgt nun zwei Ziele.
Erstens wird das Merkmal \textsc{Gespannt} vorgeschlagen, um genau diejenigen Vokale zusammenzufassen, die sowohl lang als auch kurz vorkommen.
Zweitens wird dadurch das Merkmal \textsc{Lang} aus allen zugrundeliegenden Formen eliminiert, und das Merkmal \textsc{Lage} wird auf drei Werte reduziert.
Wir führen also zunächst das Merkmal \textsc{Gespannt} ein und spezifizieren es zugrundeliegend als [\textsc{Gespannt}: $+$] für die oben genannten Vokale, die lange und kurze Varianten haben.
In (\ref{ex:gespanntheitbetonungundlaenge017}) wird das Merkmal deklariert.
Beispiel (\ref{ex:gespanntheitbetonungundlaenge018}) zeigt die resultierende zugrundeliegende Spezifikation für /i/ und /ɪ/.

\begin{exe}
  \ex{\label{ex:gespanntheitbetonungundlaenge017}\textsc{Gespannt}: $+$, $-$}
  \ex\label{ex:gespanntheitbetonungundlaenge018}
  \begin{xlist}
    \ex /i/ = [\textsc{Lage}: \textit{vorne}, \textsc{Höhe}: \textit{hoch}, \textsc{Gespannt}: $+$, \textsc{Rund}: $-$]
    \ex /ɪ/ = [\textsc{Lage}: \textit{vorne}, \textsc{Höhe}: \textit{hoch}, \textsc{Gespannt}: $-$, \textsc{Rund}: $-$]
  \end{xlist}
\end{exe}

Es ergibt sich das neue Vokaltrapez in Abbildung~\ref{fig:gespanntheitbetonungundlaenge019}, das um den Preis erkauft wird, dass [ɛ] und [a] jeweils bald als gespannt, bald als ungespannt angesehen werden.
Das gespannte [a] ist phonetisch nicht vom ungespannten [a] unterscheidbar, und Gleiches gilt für gespanntes und ungespanntes [ɛ].
In der phonologischen Notation schreiben wir hier /ă/ und /ɛ̆/ für die \textit{ungespannten} Varianten, um den Unterschied zu markieren.
Generell ist Abbildung~\ref{fig:gespanntheitbetonungundlaenge019} nicht streng phonetisch zu lesen, sondern als abstrakte phonologische Darstellung.
Phonetisch gilt weiterhin Abbildung~\ref{fig:vokaleunddiphthonge007} (Seite~\pageref{fig:vokaleunddiphthonge007}).
Schwa und [ɐ] fehlen hier, weil sie außerhalb des Systems der gespannten und ungespannten Vokale stehen (vgl.\ Satz~\ref{satz:schwabetont}).

\begin{figure}[!htpb]
  \centering
  \begin{tikzpicture}[scale=3.5,baseline=default]
    \large
    \tikzset{
    vowel/.style={fill=white, anchor=mid, text depth=0ex, text height=1ex},
    vowelgespannt/.style={circle,fill=gray!30, anchor=mid, text depth=0ex, text height=1ex,minimum size=4ex},
    dot/.style={circle,fill=black,minimum size=0.4ex,inner sep=0pt,outer sep=-1pt},
    }

    \coordinate (hf) at (0,2); % high front
    \coordinate (hb) at (2,2); % high back
    \coordinate (lf) at (1,0); % low front
    \coordinate (lb) at (2,0); % low back
    \def\V(#1,#2){barycentric cs:hf={(3-#1)*(2-#2)},hb={(3-#1)*#2},lf={#1*(2-#2)},lb={#1*#2}}

    % Chart key (vorne -- hinten).
    \draw [{Latex[round]}-] (\V (-.25,0)) -- (\V (-.25,.5))  node [above left] {\footnotesize vorne};
    \draw [-{Latex[round]}] (\V (-.25,1.5)) -- (\V (-.25,2)) node [above left] {\footnotesize hinten};
    \path (\V (-.25,1)) node[above] {\footnotesize zentral};

    % Chart key (hoch--tief).
    \draw [{Latex[round]}-] (\V (0,-.25)) -- +(270:.5cm)  node [above right,rotate=90] (vokaltrapez1) {\footnotesize hoch};
    \draw [{Latex[round]}-] (\V (3,-2.5)) -- +(270:-.5cm) node [above left,rotate=90] (vokaltrapez2) {\footnotesize tief};
    \path (\V (1.5,-1)) node[above,rotate=90] {\footnotesize mittel};

    % Grid.
    \draw [gray,thick] (\V(0,0)) -- (\V(0,2));
    \draw [gray,thick] (\V(3,0)) -- (\V(3,2));
    \draw [gray,thick] (\V(0,0)) -- (\V(3,0));
    \draw [gray,thick] (\V(0,2)) -- (\V(3,2));

    \path (\V(0,0))      node[vowelgespannt] (i)   {i};
    \path (\V(0.25,0))   node[vowelgespannt] (y)   {y};
    \path (\V(0.4,0.5))  node[vowel]         (ii)  {ɪ};
    \path (\V(0.65,0.5)) node[vowel]         (yy)  {ʏ};
    \path (\V(1,0))      node[vowelgespannt] (e)   {e};
    \path (\V(1.25,0))   node[vowelgespannt] (oe)  {ø};
    \path (\V(2,0))      node[vowelgespannt] (ee)  {ɛ};
    \path (\V(1.4,0.7))  node[vowel]         (eee) {ɛ̆};
    \path (\V(1.65,0.7)) node[vowel]         (oee) {œ};
    \path (\V(3,1))      node[vowelgespannt] (a)   {a};
    \path (\V(2.5,1))    node[vowel]         (aa)  {ă};
    \path (\V (1,2))     node[vowelgespannt] (o)   {o};
    \path (\V (1.5,1.4)) node[vowel]         (oo)  {ɔ};
    \path (\V (0,2))     node[vowelgespannt] (u)   {u};
    \path (\V (0.5,1.5)) node[vowel]         (uu)  {ʊ};

    \draw (i)  -- (ii);
    \draw (y)  -- (yy);
    \draw (e)  -- (eee);
    \draw (oe) -- (oee);
    \draw (ee) -- (eee);
    \draw (a)  -- (aa);
    \draw (o)  -- (oo);
    \draw (u)  -- (uu);
  \end{tikzpicture}
  \caption[Phonologisches Vokaltrapez]{Phonologisches Vokaltrapez (Grau für [\textsc{Gespannt}: $+$])}
  \label{fig:gespanntheitbetonungundlaenge019}
  \index{Vokaltrapez}
\end{figure}

Die Vokale in den ersten Silben von \textit{Liebe} [liːbə], \textit{Tüte} [tyːtə], \textit{Wut} [vuːt], \textit{Weg} [veːk], \textit{schön} [ʃøːn], \textit{Käse} [kɛːzə], \textit{rot} [ʁoːt], \textit{rate} [ʁaːtə] gelten also gemäß dieser leicht veränderten Merkmalsmenge als \textit{gespannt}.
In diesen Beispielen sind sie betont und daher lang.
Ungespannte Vokale können betont werden, aber sie werden dadurch nicht lang, \zB \textit{Rinder} [ʁɪndɐ].
Formen wie *[ʁɪːndɐ] sind ausgeschlossen.
Tabelle~\ref{tab:gespanntheitbetonungundlaenge020} gibt einen systematischen Überblick in Form von Beispielen.

\begin{table}[!htbp]
  \centering
  \begin{tabular}{cllp{0.25cm}cll}
    \lsptoprule
    \textbf{gespannt} & \textbf{Beispiel} & \textbf{IPA} & & \textbf{ungespannt} & \textbf{Beispiel} & \textbf{IPA} \\
    \midrule
    i  & \textit{bieten} & biːtən && ɪ & \textit{bitten}  & bɪtən   \\
    y  & \textit{fühlt}  & fyːlt  && ʏ & \textit{füllt}   & fʏlt    \\
    u  & \textit{Mus}    & muːs   && ʊ & \textit{muss}    & mʊs     \\
    e  & \textit{Kehle}  & keːlə  && ɛ & \textit{Kelle}   & kɛlə    \\
    ɛ  & \textit{stähle} & ʃtɛːlə && ɛ & \textit{Ställe}  & ʃtɛlə   \\
    ø & \textit{Höhle}  & høːlə && œ & \textit{Hölle} & hœlə \\
    o  & \textit{Ofen}   & ʔoːfən && ɔ & \textit{offen}   & ʔɔfən   \\
    a  & \textit{Wahn}   & vaːn   && a & \textit{wann}    & van     \\
    \lspbottomrule
  \end{tabular}
  \caption{Gespannte und ungespannte Vokale im Kernwortschatz}
  \label{tab:gespanntheitbetonungundlaenge020}
\end{table}

Was Gespanntheit phonetisch auszeichnet, ist nicht einfach zu bestimmen.
Man kann versuchen, die Kategorie der Gespanntheit mit einer erhöhten Muskelanspannung oder einer Veränderung der Position der Zungenwurzel in Verbindung zu bringen.
Aus Sicht der Phonologie ist der \textit{systematische} Aspekt aber wichtiger als der artikulatorische.
Für die gespannten Vokale gelten gemeinsame Strukturbedingungen, und daher sollte sie die Grammatik idealerweise als eine Klasse von Segmenten auf"|fassen -- genauso wie die stimmhaften und stimmlosen Obstruenten usw.
Mit den Ortsmerkmalen der Vokale und der Lippenrundung alleine könnte man die gespannten (und damit längbaren) Vokale aber nicht von den ungespannten unterscheiden.
Klassen definieren wir über Merkmale und Werte (vgl.\ Abschnitt~\ref{sec:merkmaleundwerte}), und daher ist das neue Merkmal gerechtfertigt.

Weil die halbvorderen und halbhinteren Vokale jetzt durch die Gespanntheit von den vorderen und hinteren unterscheidbar werden, kann ein weiteres Merkmal in seinen möglichen Werten reduziert werden.


\begin{exe}
  \ex \textsc{Lage}: \textit{vorne}, \textit{zentral}, \textit{hinten}
\end{exe}


Je nach Auf"|fassung, was der Kernwortschatz ist, gilt in ihm (auf jeden Fall aber im Erbwortschatz), dass gespannte Vokale immer betont und damit immer lang sind.%
\footnote{Zum Kernwortschatz und Erbwortschatz s.\ Abschnitt~\ref{sec:kernundperipherie}.}
Innerhalb des Kernwortschatzes gibt es damit die in Abbildung~\ref{fig:gespanntheitbetonungundlaenge019} durch Striche verbundenen Paare aus langen gespannten betonten und kurzen ungespannten betonten oder unbetonten Vokalen.
Während die ungespannten betont oder unbetont auftreten können, sind die gespannten immer betont, vgl.\ Satz~\ref{satz:gespanntheitkern}.


\Satz{Gespanntheit im Kernwortschatz}{\label{satz:gespanntheitkern}%
Im Kernwortschatz sind gespannte Vokale immer betont und lang.
Zu jedem gespannten Vokal gibt es einen entsprechenden ungespannten Vokal.
Der ungespannte ist betont oder unbetont, aber immer kurz.}

Im erweiterten Wortschatz, der mehr Wörter mit drei und mehr Silben enthält, gilt die eingangs erwähnte Strukturbedingung, dass bei den gespannten Vokalen die Betonung die Länge kontrolliert.
Beispiele für unbetonte gespannte und damit kurze Vokale sind [i] in (\ref{ex:gespanntheitbetonungundlaenge022}), [e] in (\ref{ex:gespanntheitbetonungundlaenge023}), [u] in (\ref{ex:gespanntheitbetonungundlaenge024}), [o] in (\ref{ex:gespanntheitbetonungundlaenge025}), [ø] in (\ref{ex:gespanntheitbetonungundlaenge026}) und [y] in (\ref{ex:gespanntheitbetonungundlaenge027}).


\begin{exe}
  \ex\label{ex:gespanntheitbetonungundlaenge021}
  \begin{xlist}
    \ex{\label{ex:gespanntheitbetonungundlaenge022} \textit{Idee} [ʔideː]\\
      \textit{Initiative} [ʔinit͡sʝatiːvə]\\
      \textit{inspirieren} [ʔɪnspiʁiːʁən] }
    \ex{\label{ex:gespanntheitbetonungundlaenge023} \textit{Methyl} [metyːl]\\
      \textit{Québec} [kebɛk]\\
      \textit{integriert} [ʔɪntegʁi͡ɐt]\\
      \textit{debattieren} [debatiːʁən] }
    \ex{\label{ex:gespanntheitbetonungundlaenge024} \textit{Utopie} [ʔutopiː]\\
      \textit{Uran} [ʔuʁaːn] }
    \ex{\label{ex:gespanntheitbetonungundlaenge025} \textit{Motiv} [motiːf]\\
      \textit{politisch} [poliːtɪʃ]\\
      \textit{Phonologie} [fonologiː] }
    \ex{\label{ex:gespanntheitbetonungundlaenge026} \textit{Ökonomie} [ʔøkonomiː]\\
      \textit{manövrieren} [manøvʁiːʁən] }
    \ex{\label{ex:gespanntheitbetonungundlaenge027} \textit{Büro} [byʁoː]\\
    \textit{Cuvée} [kyveː] }
  \end{xlist}
\end{exe}


Weil Wörter mit solchen Vokalen im alltäglichen Gebrauch durchaus häufig vorkommen, wird in Satz~\ref{satz:gespannterweitert} nicht von \textit{peripherem Wortschatz}, sondern vorsichtiger vom \textit{erweiterten Wortschatz} gesprochen.


\Satz{Gespanntheit im erweiterten Wortschatz}{\label{satz:gespannterweitert}%
Im erweiterten Wortschatz sind gespannte Vokale lang, wenn sie betont sind, und kurz, wenn sie unbetont sind.
Auch im erweiterten Wortschatz gibt es keine ungespannten langen Vokale.
}

Völlig außerhalb dieses Systems stehen Schwa und [ɐ] gemäß Satz~\ref{satz:schwabetont}.

\Satz{Schwa}{\label{satz:schwabetont}%
Schwa und [ɐ] sind immer kurz und nie betont.
}

Damit müssen die zugrundeliegenden Formen genau wie bei der Endrand"=Desonorisierung gemäß der neu eingeführten Strukturbedingungen angepasst werden.
Länge muss nicht mehr zugrundeliegend spezifiziert werden, und man erhält Beispiele wie in (\ref{ex:gespanntheitbetonungundlaenge028}).

\begin{exe}
  \ex\label{ex:gespanntheitbetonungundlaenge028} \begin{xlist}
    \ex /veg/ \phopro [veːk]
    \ex /hølə/ \phopro [høːlə]
    \ex /ofən/ \phopro [ʔoːfən]
  \end{xlist}
\end{exe}

\subsection{Verteilung von [ç] und [χ]}
\label{sec:verteilungvonund}

Die sogenannten \textit{ich}- und \textit{ach}-Segmente sind komplementär verteilt.
Es gibt kein Wort, in dem sie einen lexikalischen Unterschied markieren.
Einige Beispielwörter, in denen [ç] und [χ] vorkommen, illustrieren diese Verteilungen in (\ref{ex:verteilungvonund029}).

\begin{exe}
  \ex\label{ex:verteilungvonund029}
  \begin{xlist}
    \ex{\label{ex:verteilungvonund030} krieche, schlich, Bücher, Küche, Recht, Köche}
    \ex{\label{ex:verteilungvonund031} Tuch, Geruch, hoch, Koch, Schmach, Bach}
  \end{xlist}
\end{exe}

Ausschlaggebend für das Vorkommen von [ç] und [χ] ist der unmittelbar vorangehende Kontext.
Nach /i/, /ɪ/, /y/, /ʏ/, /e/, /ɛ/, /ɛ̆/, /ø/, /œ/ kommt [ç] vor, nach /u/, /ʊ/, /o/, /ɔ/, /a/ und /ă/ hingegen [χ].
Nach Schwa kommt keins der beiden Segmente vor.
Ein Blick auf das phonologische Vokaltrapez in Abbildung~\ref{fig:gespanntheitbetonungundlaenge019} zeigt sofort, was der relevante Merkmalsunterschied zwischen den beiden Gruppen von Vokalen ist.
Nach Vokalen, die [\textsc{Lage}: \textit{vorne}] sind, steht [ç].
Nach allen anderen Vokalen steht hingegen [χ].
Es handelt sich hier um eine Angleichung des Artikulationsorts des Frikativs an den hinterer Vokale, eine sogenannte \textit{Assimilation}.\index{Assimilation}

Es muss jetzt nur noch entschieden werden, wie die zugrundeliegende Form in diesem Fall aussieht.
Aufschlussreich ist hier die Betrachtung von Wörtern wie \textit{Milch} /mɪlç/, \textit{Storch} /ʃtɔʁç/ oder \textit{Röckchen} /ʁœkçən/, in denen [ç], aber niemals [χ] nach einem Konsonanten vorkommt.
Dies ist generell der Fall, und es ist deswegen günstiger, anzunehmen, dass /ç/ zugrundeliegt und [χ] das phonetische Resultat einer Assimilation ist.
Das heißt, dass [χ] kein zugrundeliegendes Segment ist und nicht in /~/ gehört.
Mit der entsprechenden Strukturbedingung aus Satz~\ref{satz:cassimilation} ergeben sich die Beispiele wie in (\ref{ex:verteilungvonund032}).

\Satz{/ç/-Assimilation}{\label{satz:cassimilation}%
[ç] kann nicht nach Vokalen stehen, die nicht [\textsc{Lage}: \textit{vorne}] sind.
Zugrundeliegendes /ç/ wird daher nach zentralen und hinteren Vokalen weiter hinten artikuliert, nämlich als [χ].}

\begin{exe}
  \ex\label{ex:verteilungvonund032}
  \begin{xlist}
    \ex{/ɪç/ \phopro [ʔɪç]}
    \ex{/ăç/ \phopro [ʔaχ]}
  \end{xlist}
\end{exe}

\subsection{/ʁ/-Vokalisierungen}
\label{sec:vokalisierungen}

In Abschnitt~\ref{sec:orthographischesr} wurden phonetische Korrelate von geschriebenem \textit{r} besprochen.
Die Schrift ist hier besonders systematisch, denn orthographisches \textit{r} entspricht immer einem zugrundeliegenden /ʁ/ (s.\ auch Abschnitt~\ref{sec:buchstabenundphonologischesegmente}).
In (\ref{ex:vokalisierungen033}) sind Beispiele zusammengestellt (inklusive der Silbengrenzen in Form von Punkten), die dies illustrieren.

\begin{exe}
  \ex\label{ex:vokalisierungen033}
  \begin{xlist}
    \ex{kleiner [kla͡ɛ.nɐ], kleinere [kla͡ɛ.nə.ʁə]}
    \ex{Bär [bɛ͡ɐ], Bären [bɛː.ʁən]}
    \ex{knarr [kna͡ə], knarre [kna.ʁə]}
  \end{xlist}
\end{exe}

Wenn ein zugrundeliegendes /ʁ/ am Silbenanfang steht, wird es als Konsonant [ʁ] realisiert.
Demgegenüber findet am Silbenende immer eine Vokalisierung von /ʁ/ statt.
Nach gespannten Vokalen wird /ʁ/ zu [ɐ], nach ungespannten zu [ə].
Nach (stets unbetontem) Schwa wird /ʁ/ gar nicht realisiert, und Schwa wird zu [ɐ].
Diese Vorgänge formal genau aufzuschreiben, würde den hier gegebenen Rahmen sprengen.
Aus Sicht der Phonologie sind die Unterschiede zwischen [ə] und [ɐ] auch nicht erheblich, denn diese Segmente stellen nur minimal unterschiedliche Färbungen des Schwa-Segments dar.
Beispiele folgen in (\ref{ex:vokalisierungen034}).

\begin{exe}
  \ex \label{ex:vokalisierungen034}
  \begin{xlist}
    \ex /kla͡ɛnəʁ/ \phopro [kla͡ɛ.nɐ]
    \ex /tiʁ/ \phopro [ti͡ɐ]
    \ex /bɪʁkə/ \phopro [bɪ͡ə.kə]
  \end{xlist}
\end{exe}

Die entsprechende Strukturbedingung und ihre Effekte werden in Satz~\ref{satz:rvokalisierung} beschrieben.

\Satz{/ʁ/-Vokalisierung}{\label{satz:rvokalisierung}%
Zugrundeliegendes /ʁ/ kann nicht am Silbenende stehen.
Es wird in dieser Position als Schwa-Segment im sekundären Diphthong realisiert.
Nach gespanntem Vokal folgt [ɐ], nach ungespanntem folgt [ə].
Schwa und /ʁ/ werden zusammen durch [ɐ] substituiert.
}

\Zusammenfassung{%
In der Phonologie ist der Status der Segmente im Gesamtsystem relevant.
Dabei werden vor allem ihre Verteilung und ihre Merkmale betrachtet.
Wenn man alle Formen eines Worts berücksichtigt, kann man umgebungsabhängige bzw.\ positionsabhängige Änderungen von Merkmalswerten beobachten.
Um solche Phänomene adäquat zu beschreiben, nimmt man abstraktere zugrundeliegende Formen an, die an phonologische Strukturbedingungen wie die Endrand-Desonorisierung angepasst werden.
}

