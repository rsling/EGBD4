\chapter{Statusrektion}
\label{sec:statusrektion}

\section{Analytische Tempora}
\label{sec:analytischetempora}

Mit diesem Abschnitt beginnt die nähere Betrachtung von Rektionsrelationen zwischen Verben.
Dazu gehören Konstruktionen mit Hilfsverben (die in diesem Abschnitt diskutiert werden), Modalverben (Abschnitt~\ref{sec:modalverbenundaehnliches}) und Verben, die den zweiten Status (\textit{zu}"=Infinitiv) regieren (Abschnitt~\ref{sec:infinitivkontrolle}).

Die analytischen Tempusformen wurden schon in Abschnitt~\ref{sec:tempusformen} eingeführt.
Bei ihnen wird ein bestimmter Tempuseffekt durch ein Verb und ein Hilfsverb erzeugt.
Im einfachen Fall ist das Hilfsverb finit, wie in \textit{hat gekauft} usw.
Dieser Abschnitt zeigt vor allem, wie die Bildungen der analytischen Tempora miteinander und mit anderen Hilfsverben i.\,w.\,S.\ kombinierbar sind.
Außerdem wird kurz die Bedeutung der Perfekta diskutiert.
Zunächst stellt Tabelle~\ref{tab:analytischetempora215} nochmals die Bildungen der analytischen Tempora auf reduzierte Weise zusammen.

\index{Verb!Hilfs--}
\index{Tempus!analytisch}
\index{Verb!Perfekt}
\index{Verb!Futur}
\index{Verb!Status}

\begin{table}[!htbp]
  \centering
  \begin{tabular}{lll}
    \lsptoprule
    & \textbf{Hilfsverb} & \textbf{regierter Status} \\
    \midrule
    \textbf{Futur} & \textit{werden} & 1 (Infinitiv) \\
    \textbf{Perfekt} & \textit{haben}\slash\textit{sein} & 3 (Partizip) \\
    \lspbottomrule
  \end{tabular}
  \caption{Analytische Tempora des Deutschen}
  \label{tab:analytischetempora215}
\end{table}

Die Reduktion auf zwei analytische Tempora in Tabelle~\ref{tab:analytischetempora215} ist aus schulgrammatischer Sicht ggf.\ genauso verwunderlich wie die Reduktion auf zwei morphologische Tempusformen (Präsens und Präteritum), die wir in Abschnitt~\ref{sec:tempusformen} vorgenommen haben.
Der Verbleib des \textit{Futurs II} und des \textit{Plusquamperfekts} ist zu erklären.
Die Tempuskonstruktionen in (\ref{ex:analytischetempora216}) und (\ref{ex:analytischetempora219}) zeigen, wie die analytischen Tempora aufgebaut sind.

\begin{exe}
  \ex\label{ex:analytischetempora216}
  \begin{xlist}
    \ex{\label{ex:analytischetempora217} dass der Hufschmied das Pferd [behuft hat]}
    \ex{\label{ex:analytischetempora218} dass der Hufschmied das Pferd [behufen wird]}
  \end{xlist}
  \ex\label{ex:analytischetempora219}
  \begin{xlist}
    \ex{\label{ex:analytischetempora220} dass der Hufschmied das Pferd [behuft hatte]}
    \ex{\label{ex:analytischetempora221} dass der Hufschmied das Pferd [[behuft haben] wird]}
  \end{xlist}
\end{exe}

In (\ref{ex:analytischetempora217}) steht ein einfaches Perfekt mit finitem \textit{hat} im Präsens und abhängigem drittem Status, und in (\ref{ex:analytischetempora218}) steht ein einfaches Futur mit finitem \textit{wird} im Präsens und abhängigem erstem Status.
Das sogenannte Plusquamperfekt in (\ref{ex:analytischetempora220}) entspricht nun genau dem Perfekt, nur dass das Hilfsverb im Präteritum statt im Präsens steht (\textit{hatte}).
Das Futurperfekt in (\ref{ex:analytischetempora221}) ist bei genauem Hinsehen die Kombination aus dem finiten Futur-Hilfsverb \textit{wird} im Präsens, von dem eine Perfektbildung im Infinitiv abhängt, denn \textit{haben} ist offensichtlich der erste Status des Hilfsverbs \textit{haben}.

\index{Präsensperfekt}
\index{Präteritumsperfekt}
\index{Futurperfekt}

Die Konstruktion \textit{behuft haben wird} zeigt also, dass es nicht richtig ist, zu sagen, dass analytische Tempora immer aus genau einem finiten Hilfsverb und genau einem infiniten Vollverb bestünden.
Analytisch betrachtet ist das Futurperfekt nichts weiter als eine Kombination aus Futur und Perfekt, weswegen die Bezeichnung \textit{Futurperfekt} auch wesentlich angemessener als \textit{Futur II} ist.
Es handelt sich nicht um ein unabhängig zu definierendes Tempus, sondern um eine Kombination zweier analytischer Tempora.
Auf ganz ähnliche Weise ist das Plusquamperfekt ein Perfekt im Präteritum.
Traditionell als \textit{Perfekt} bezeichnete Formen wie in (\ref{ex:analytischetempora217}) heißen korrekt \textit{Präsensperfekt} und stehen neben dem \textit{Präteritumsperfekt} wie in (\ref{ex:analytischetempora221}).
Aus Satz~\ref{satz:anatemp} und dem Wissen aus Kapitel~\ref{sec:verbalflexion} ergibt sich nun, dass Präsens und Präteritum (morphologisch) sowie das Futur (analytisch) die stets finiten Tempora des Deutschen sind.
Das Perfekt kann finit und infinit verwendet und mit jedem der drei finiten Tempora kombiniert werden.

\Satz{Analytische Tempora und Finitheit}{\label{satz:anatemp}%
Als analytische Tempora müssen nur das einfache Perfekt und das Futur angenommen werden.
Das Perfekt kann (anders als das Futur) bei infiniter Verwendung des Hilfsverbs (\textit{haben} oder \textit{sein}) selbst von Hilfsverben, Modalverben usw.\ abhängen.
Die traditionell so genannten Bildungen des Perfekts, Plusquamperfekts und Futur II sind lediglich die finiten Perfektbildungen (Präsensperfekt, Präteritumsperfekt, Futurperfekt).
}

Die Konstituentenklammern in (\ref{ex:analytischetempora216}) und (\ref{ex:analytischetempora219}) deuten an, dass es sich hier um Verbkomplexbildung handelt, und dass innerhalb des Verbkomplexes die Verben paarweise kombiniert werden, so dass jedes Hilfsverb mit demjenigen Verb kombiniert wird, dessen Status es regiert.
Im Verbkomplex können neben den Tempushilfsverben aber auch andere Hilfsverben, \zB das Passiv-Hilfsverb \textit{werden}, und Modalverben vorkommen.
Die hier gezeigte Analyse, in der das Perfekt selber infinite Formen haben kann, erlaubt nun erst die Analyse von Konstruktionen wie in (\ref{ex:analytischetempora222}).

\begin{exe}
  \ex\label{ex:analytischetempora222}
  \begin{xlist}
    \ex{\label{ex:analytischetempora223} dass der Hufschmied das Pferd [[behuft haben] will]}
    \ex{\label{ex:analytischetempora224} dass der Hufschmied das Pferd [[[behuft gehabt] haben] will]}
  \end{xlist}
\end{exe}

\index{Perfekt!Doppel--}

In (\ref{ex:analytischetempora223}) bettet das Modalverb \textit{wollen} (selber im Präsens) ein infinites Perfekt von \textit{behufen} ein (markiert durch \textit{haben} und das Partizip \textit{behuft}).
In (\ref{ex:analytischetempora224}) bettet dasselbe Modalverb ein Perfekt (\textit{haben}) von einem Perfekt (\textit{gehabt}) ein, ein sogenanntes \textit{Doppelperfekt}.
Bevor kurz die Bedeutung dieser Konstruktionen diskutiert wird, soll betont werden, dass beim Doppelperfekt ganz einfach das Perfekt gemäß Tabelle~\ref{tab:analytischetempora215} rekursiv eingebettet wird, wie die Analyse in Abbildung~\ref{fig:analytischetempora225} zeigt.

\begin{figure}[!htbp]
  \centering
  \begin{forest}
    [\textbf{V}, calign=last
      [\textbf{V}, calign=last
        [\textbf{V}, calign=last
          [\textbf{V}, tier=preterminal
            [\textit{behuft}]
          ]
          [\textbf{V}, tier=preterminal
            [\textit{gehabt}]
            {\draw [->, bend left=45] (.south) to node [below, midway] {\footnotesize\textsc{Status} (3)} (!uu11.south);}
          ]
        ]
        [\textbf{V}, tier=preterminal
          [\textit{haben}]
          {\draw [->, bend left=45] (.south) to node [below, midway] {\footnotesize\textsc{Status} (3)} (!uu121.south);}
        ]
      ]
      [\textbf{V}, tier=preterminal
        [\textit{will}]
        {\draw [->, bend left=45] (.south) to node [below, midway] {\footnotesize\textsc{Status} (1)} (!uu121.south);}
      ]
    ]
  \end{forest}

  \caption{Verbkomplex mit Modalverb und Doppelperfekt}
  \label{fig:analytischetempora225}
\end{figure}

\index{Perfekt!Semantik}

Die Bedeutung des Präsensperfekts ist nun in erster Näherung genau die des Präteritums, wobei das Präteritum eher der Schriftsprache und das Präsensperfekt eher der gesprochenen Sprache zuzuordnen ist.
Einen semantischen Unterschied zwischen (\ref{ex:analytischetempora227}) und (\ref{ex:analytischetempora228}) auszumachen, fällt schwer.


\begin{exe}
  \ex\label{ex:analytischetempora226}
  \begin{xlist}
    \ex{\label{ex:analytischetempora227} Das Pferd lief im Kreis.}
    \ex{\label{ex:analytischetempora228} Das Pferd ist im Kreis gelaufen.}
  \end{xlist}
\end{exe}


Neben stilistischen Unterschieden gibt es allerdings bei bestimmten Verbtypen eine semantische Ambiguität des Präsensperfekts, die das Präteritum nicht aufweist.
Sie wird in (\ref{ex:analytischetempora229}) illustriert.


\begin{exe}
  \ex\label{ex:analytischetempora229}
  \begin{xlist}
    \ex{\label{ex:analytischetempora230} Im Jahr 1993 hat der Kommerz den Techno erobert.}
    \ex{\label{ex:analytischetempora231} Im Jahr 1993 eroberte der Kommerz den Techno.}
  \end{xlist}
\end{exe}


\label{abs:analytischetempora232}Während das Präteritum in (\ref{ex:analytischetempora231}) nur so verstanden werden kann, dass es 1993 ein als punktuell betrachtetes Ereignis gab, lässt das Präsensperfekt in (\ref{ex:analytischetempora230}) neben dieser Lesart eine zweite zu.
Bei dieser zweiten Lesart gab es ein ausgedehntes Ereignis der Eroberung, das bereits vor 1993 begonnen haben könnte, das aber auf jeden Fall 1993 zur Vollendung kam.
Die tiefe Verwurzelung dieser Doppeldeutigkeit in der Verbsemantik zeigt sich daran, dass sie nur dann auftritt, wenn das lexikalische Verb von einem bestimmten semantischen Typ ist.
Es muss ein Ereignis beschreiben, das eine zeitliche Ausdehnung hat (\zB das Voranschreiten des Eroberungsprozesses) und dann in einem Zeitpunkt endet, in dem es erfolgreich abgeschlossen ist (die vollständige Eroberung).
Genau deswegen entsteht die Doppeldeutigkeit in (\ref{ex:analytischetempora228}) nicht, denn solange ein Pferd auch im Kreis läuft, es wird nie zwingend einen abschließenden Punkt geben, an dem das Im-Kreis-Laufen erfolgreich abgeschlossen ist.

Wie ist es nun mit dem Doppelperfekt?
Das Doppelperfekt könnte in Beispielen wie (\ref{ex:analytischetempora224}) funktional als Ersatz eines infiniten Präteritumsperfekts geeignet sein.
Das Präteritumsperfekt ist (als Kombination des immer finiten Präteritums und des Perfekts) immer eine finite Form.
Um die Vor-Vorzeitigkeit auch in infiniten Formen zu markieren, könnte also eine zweifache Bildung des Perfekts (doppeltes \textit{haben}) verwendet werden.
Es ist allerdings schwer, die entsprechenden Lesarten in Sätzen mit Doppelperfekt zwingend zu erkennen, und das Doppelperfekt kommt eben auch finit vor.
Selbst für Dialekte, in denen es (außer bei den Hilfsverben) kein Präteritum mehr gibt, und in denen Doppelperfekta finit vorkommen (\textit{hat behuft gehabt}), ist die Vor-Vorzeitigkeitsbedeutung nicht zwingend zu erkennen.
Das Doppelperfekt wird dort auch gebraucht, wenn nur eine Perfektbedeutung intendiert ist.
Wir können hier also die genaue Funktion und Verwendungsweise dieser ungewöhnlichen Bildung nicht ganz klären.
Morphosyntaktisch fügt sich das Doppelperfekt aber einwandfrei in das System der analytischen Tempora des Deutschen ein.
Nur sehr kurz wird hier angemerkt, dass auch das Passiv infinite Formen hat, wenn das Passiv-Hilfsverb infinit ist.
Es ergeben sich infinite Passive wie \textit{behuft werden}, \textit{behuft worden} und \textit{behuft geworden}.
Diese können mit temporalen Hilfsverben dann das Perfekt (\ref{ex:analytischetempora234}), das Präteritumsperfekt (\ref{ex:analytischetempora235}), das Futur (\ref{ex:analytischetempora236}) oder das Futurperfekt (\ref{ex:analytischetempora237}) bilden.


\begin{exe}
  \ex\label{ex:analytischetempora233}
  \begin{xlist}
    \ex{\label{ex:analytischetempora234} dass Tarek [[behuft worden] ist]}
    \ex{\label{ex:analytischetempora235} dass Tarek [[behuft worden] war]}
    \ex{\label{ex:analytischetempora236} dass Tarek [[behuft werden] wird]}
    \ex{\label{ex:analytischetempora237} dass Tarek [[[behuft geworden] sein] wird]}
  \end{xlist}
\end{exe}

\Zusammenfassung{%
Futurperfekt und Präteritumsperfekt sind keine eigenständigen Tempora, sondern werden analytisch auf Basis des Perfekts und des Präteritums bzw.\ des Futurs gebildet.
}

\section{Modalverben und Ähnliches}
\label{sec:modalverbenundaehnliches}

In Abschnitt~\ref{sec:analytischetempora} wurden Hilfsverben besprochen, die vor allem den dritten Status, aber auch den ersten Status regieren.
Ein Modalverb ist hingegen das prototypische Regens des ersten Status.
In diesem Abschnitt geht es um besondere Phänomene rund um die Modalverben.
Abschnitt~\ref{sec:ersatzinfinitivundoberfeldumstellung} beschreibt eine Stellungsvariante im Verbkomplex, die vor allem mit Modalverben auftritt.
In Abschnitt~\ref{sec:kohaerenz} geht es um die besondere Art der \textit{kohärenten} Verbkomplexbildung, die insbesondere für Modalverben typisch ist.
Kohärenz spielt dann auch in Abschnitt~\ref{sec:modalverbenundhalbmodalverben} eine große Rolle, in dem eine den Modalverben ähnliche Klasse (die \textit{Halbmodalverben}) diskutiert wird.

\subsection{Ersatzinfinitiv und Oberfeldumstellung}
\label{sec:ersatzinfinitivundoberfeldumstellung}

\index{Verb!Infinitiv!Ersatz--}
\index{Oberfeldumstellung}

In Abschnitt~\ref{sec:verbkomplex} wurde gesagt, dass die Rektionshierarchie im Verbkomplex die Reihenfolge der Verben meistens eindeutig bestimmt.
Das finite Verb steht am Ende, die von ihm abhängigen infiniten Verben stehen in absteigender Hierarchie davor.
Mit der dort eingeführten Numerierung ergeben sich für Verbkomplexe mit drei Verben wie \textit{kaufen}(3) \textit{wollen}(2) \textit{wird}(1) dann Bezeichnungen wie \textit{321-Komplex}.
Eine wichtige Ausnahme zu dieser Regularität zeigen Beispiel (\ref{ex:ersatzinfinitivundoberfeldumstellung238}) und die Analyse in Abbildung~\ref{fig:ersatzinfinitivundoberfeldumstellung239}.

\begin{exe}
  \ex{\label{ex:ersatzinfinitivundoberfeldumstellung238} dass der Junge [hat [[schwimmen] wollen]]}
\end{exe}

\begin{figure}[!htbp]
  \centering
  \begin{forest}
    [\textbf{V}, calign=first
      [\textbf{V}, tier=preterminal
        [\textit{hat}\\1]
      ]
      [\textbf{V}, calign=last
        [\textbf{V}, tier=preterminal
          [\textit{schwimmen}\\3]
        ]
        [\textbf{V}, tier=preterminal
          [\textit{wollen}\\2]
          {\draw [->, bend left=20] (.south) to node [below, near end] {\footnotesize\textsc{Status} (1)} (!uu11.south);}
          {\draw [<-, bend left=60] (.south) to node [below, midway] {\footnotesize\textsc{Status} (1)} (!uuu11.south);}
        ]
      ]
    ]
  \end{forest}

  \caption{Verbkomplex mit Oberfeldumstellung}
  \label{fig:ersatzinfinitivundoberfeldumstellung239}
\end{figure}

Unter bestimmten Umständen ist die Rektionsfolge im Verbkomplex mit drei Verben also nicht 321.
Es liegt eine sogenannte \textit{Oberfeldumstellung} vor (132), die am typischsten mit dem Perfekt (Hilfsverb \textit{haben}) und Modalverben, aber auch anderen Verben (\zB \textit{sehen}) auftritt.
Dabei regiert dann typischerweise das Perfekt-Hilfsverb nicht den 3.~Status (\textit{gesehen}), sondern den 1.~Status (\textit{sehen}), den sogenannten \textit{Ersatzinfinitiv}.
\index{Status}

\Satz{Oberfeldumstellung und Ersatzinfinitiv}{\label{satz:oberfeldumst}%
Bei der Oberfeldumstellung wird das finite Verb im Verbkomplex von der letzten an die erste Position des Komplexes umgestellt.
Wenn das Perfekt-Hilfsverb auf diese Weise umgestellt wird, regiert es den Infinitiv anstelle des Partizips (Ersatzinfinitiv).
\index{Oberfeldumstellung}
\index{Verb!Infinitiv!Ersatz--}
\index{Verb!Partizip}
}

Das Phrasenschema für den Verbkomplex (Schema~\ref{str:vk} auf Seite~\pageref{str:vk}) müsste angepasst werden, was hier aus Platzgründen nicht erfolgt.
Ebenso kann hier keine besondere Erklärung für diese Ausnahme geliefert werden, und wir betrachten das Phänomen schlicht als Grille des grammatischen Systems.
Der Begriff \textit{Oberfeldumstellung} stammt aus einer Erweiterung des Feldermodells für den Verbkomplex, die allerdings längst nicht den intuitiven Charakter hat wie das Feldermodell für Sätze.


\subsection{Kohärenz}
\label{sec:kohaerenz}

\index{Kohärenz}
\index{Verbkomplex}
\index{Verb!Infinitiv}

Mit \textit{Kohärenz} ist hier nicht der textlinguistische Begriff gemeint, also der inhaltliche Zusammenhang und die argumentative Geschlossenheit eines Textes, sondern ein syntaktisches Phänomen aus dem Bereich der infiniten Verben im Deutschen.
Köharenz (oder eben Inkohärenz) spielt eine Rolle bei der Konstituentenanalyse einer VP mit Modalverben und anderen Verben, die ein infinites Verb regieren.
Konkret muss entschieden werden, ob alle Verben, die infinite Verben regieren, mit diesen einen Verbkomplex bilden, oder ob es auch Verben gibt, die andere Phrasenstrukturen realisieren.
Dabei stehen die möglichen Analysen wie in der schematischen Abbildung~\ref{fig:kohaerenz240} zur Diskussion.
Zur Verdeutlichung der Verhältnisse wird die Indexnotation aus Abschnitt~\ref{sec:verbkomplex} verwendet.

\begin{figure}[!htbp]
  \centering
  \begin{forest}
    [VP\Sub{1+2}, calign=last
      [\ldots, fit=band]
      [\textbf{V\Sub{2+1}}, calign=last
        [\textbf{V\Sub{2}}]
        [\textbf{V\Sub{1}}]
        {\draw [->, bend left=45] (.south) to (!u1.south);}
      ]
    ]
  \end{forest}\hspace{4em}\begin{forest}
    [VP\Sub{1}, calign=last
      [\ldots, fit=band]
      [VP\Sub{2}, calign=last
        [\ldots]
        [\textbf{V\Sub{2}}, tier=terminal]
      ]
      [\textbf{V\Sub{1}}, tier=terminal]
      {\draw [->, bend left=45] (.south) to (!u22.south);}
    ]
  \end{forest}
  \caption[Inkohärente und kohärente Konstruktion]{Kohärente (links) und inkohärente (rechts) Konstruktion}
  \label{fig:kohaerenz240}
\end{figure}

\begin{figure}[!htbp]
  \centering
  \begin{forest}
    [VP\Sub{1+2}, calign=last
      [NP, tier=preterminal
        [\textit{Vanessa}, narroof]
      ]
      [NP, tier=preterminal
        [\textit{die Pferde}, narroof]
      ]
      [\textbf{V\Sub{2+1}}, calign=last
        [\textbf{V\Sub{2}}, tier=preterminal
          [\textit{behufen}]
        ]
        [\textbf{V\Sub{1}}, tier=preterminal
          [\textit{will}]
        ]
      ]
    ]
  \end{forest}
  \caption{Kohärente Konstruktion mit \textit{wollen}}
  \label{fig:kohaerenz241}
\end{figure}

\begin{figure}[!htbp]
  \begin{forest}
    l sep+=3em, s sep+=2em
    [VP\Sub{1}, calign=last
      [NP, tier=preterminal
        [\textit{Vanessa}, narroof]
      ]
      [VP\Sub{2}, calign=last
        [NP, tier=preterminal
          [\textit{die Pferde}, narroof]
        ]
        [\textbf{V\Sub{2}}, tier=preterminal
          [\textit{zu behufen}]
        ]
      ]
      [\textbf{V\Sub{1}}, tier=preterminal
        [\textit{wünscht}]
      ]
    ]
  \end{forest}
  \caption{Inkohärente Konstruktion mit \textit{wünschen}}
  \label{fig:kohaerenz242}
\end{figure}

\index{Verb!Status}
\index{Verbphrase}
Entweder bilden Verben, die Statusrektion haben, mit dem regierten infiniten Verb einen Verbkomplex, der dann den Kopf einer VP bildet und die anderen Ergänzungen regiert.
Oder das infinite Verb (mit seinen Ergänzungen und Angaben) bildet zunächst eine eigene VP, die sich ähnlich wie ein Nebensatz verhält, und die als Ganzes eine Ergänzung zu dem den Status regierenden Verb ist.
Hier wird nur eine von mehreren empirischen Beobachtungen erwähnt, die darauf hindeuten, dass die Analyse in Abbildung~\ref{fig:kohaerenz241} für Modalverben wie \textit{wollen} immer die angemessene ist.
Gleichzeitig zeigt sich, dass eine Analyse wie in Abbildung~\ref{fig:kohaerenz242} für Verben wie \textit{wünschen} als mögliche Alternative zu der in Abbildung~\ref{fig:kohaerenz241} angenommen werden muss.%
\footnote{Der Gesamtheit der Kohärenzphänomene können wir hier nicht im Ansatz gerecht werden.
Selbst große wissenschaftliche Grammatiken des Deutschen breiten zu diesem Thema nicht die volle Komplexität der Daten und Tests aus, \zB der \textit{Grundriss} \citep[359--361]{Eisenberg2013b} oder die Duden-Grammatik \citep[§1314--§1323]{Duden8}.
Einen sehr gründlichen theorieneutralen Überblick über die Phänomene, die zu diesem Thema gehören, sowie über die existierende Literatur findet man in \citet[253--275]{Mueller2008}.
}

Für die Entwicklung eines Tests muss man sich die folgenden Fakten vor Augen führen.
Generell können satzähnliche Gruppen im Deutschen nach rechts hinter das finite Verb herausgestellt werden, und zwar auch innerhalb eines Nebensatzes.
Bei der Besprechung des Feldermodells wurde dies in Abschnitt~\ref{sec:konstituentenstellungundfeldermodell} \zB für Relativsätze gezeigt.
In (\ref{ex:kohaerenz243}) wird versucht, die potentielle eingebettete VP auf diese Weise herauszustellen.


\begin{exe}
  \ex\label{ex:kohaerenz243}
  \begin{xlist}
    \ex[ ]{\label{ex:kohaerenz244} Oma glaubt, dass Vanessa \Ti\ wünscht, [die Pferde zu behufen]\ORi.}
    \ex[*]{\label{ex:kohaerenz245} Oma glaubt, dass Vanessa \Ti\ will, [die Pferde behufen]\ORi.}
  \end{xlist}
\end{exe}


Der Befund ist eindeutig.
Das Modalverb erlaubt diese Herausstellung nicht, \textit{wünschen} aber schon.
Die größere Bewegungsfreiheit des regierten lexikalischen infiniten Verbs zusammen mit seinen Ergänzungen und Angaben bei Verben wie \textit{wünschen} ist ein guter Hinweis darauf, dass sie eine Konstruktion wie in Abbildung~\ref{fig:kohaerenz242} erlauben.
Das lexikalische Verb bildet in diesen Fällen zunächst eine (subjektlose) VP, die dann als Ganzes eine Ergänzung zu einem anderen Verb ist.
Im Gegensatz zur kohärenten Bildung eines Verbkomplexes aus einer finiten und den infiniten Verbformen (wie es bei den Modalverben prinzipiell der Fall ist) nennt man diese Konstruktion dann \textit{inkohärent}.
Ob ein statusregierendes Verb inkohärent konstruieren kann (\textit{optional inkohärent}) oder immer kohärent konstruiert (\textit{obligatorisch kohärent}), ist eine lexikalische Eigenschaft.

Als Faustregel gilt, dass Verben, die den ersten und dritten Status regieren (\zB auch Hilfsverben wie \textit{haben} und \textit{sein} bei Perfektbildungen), typischerweise kohärent konstruieren, während Verben, die den zweiten Status regieren, typischerweise optional inkohärent konstruieren (vgl.\ aber \zB Abschnitt~\ref{sec:modalverbenundhalbmodalverben}).
Modalverben konstruieren immer kohärent, weswegen das Phänomen in diesem Abschnitt verortet wurde.
Es folgt die entsprechende Definition~\ref{def:kohaerenz}.

\Definition{Kohärenz}{\label{def:kohaerenz}%
Ein Verb V\Sub{1}, das ein (infinites) Verb V\Sub{2} regiert, \textit{konstruiert kohärent}, wenn V\Sub{1} und V\Sub{2} einen Verbkomplex bilden.
Bei der \textit{inkohärenten Konstruktion} bildet das abhängige Verb V\Sub{2} eine eigene Konstituente (VP), die sich ähnlich wie ein Nebensatz verhält.
\index{Kohärenz}
}

Gerne wird noch der Unterschied zwischen \textit{obligatorisch} und \textit{fakultativ kohärent konstruierenden} Verben gemacht.
Eine Diskussion dieser Phänomene und Analysen würde hier zu weit führen.
Gleiches gilt für die Beschreibung der damit in Verbindung stehenden \textit{dritten Konstruktion}, einer angenommenen weiteren Möglichkeit neben kohärenter und inkohärenter Konstruktion.\index{dritte Konstruktion}
Dabei können wie in (\ref{ex:kohaerenz246}) bei bestimmten Verben Teilkonstituenten (hier \textit{die Pferde}) aus einem nach rechts versetzen Infinitiv (hier \textit{gründlich zu putzen}) links vom VK verbleiben.

\begin{exe}
  \ex[?]{\label{ex:kohaerenz246} Ich glaube, dass Vanessa die Pferde versucht hat, gründlich zu putzen.}
\end{exe}

\subsection{Modalverben und Halbmodalverben}
\label{sec:modalverbenundhalbmodalverben}

\index{Verb!Status}
\index{Verb!Modal--}

Eine Gruppe von Verben, die den zweiten Status regiert (prototypisch \textit{scheinen}), zeigt ein besonderes syntaktisches Verhalten im Kontrast zu den Modalverben einerseits und anderen Verben, die den zweiten Status regieren (\zB \textit{beschließen}), andererseits.
Zunächst fällt auf, dass die Beispielsätze in (\ref{ex:modalverbenundhalbmodalverben247}) alle strukturell identisch aussehen.

\begin{exe}
  \ex\label{ex:modalverbenundhalbmodalverben247}
  \begin{xlist}
    \ex{\label{ex:modalverbenundhalbmodalverben248} dass der Hufschmied das Pferd behufen will.}
    \ex{\label{ex:modalverbenundhalbmodalverben249} dass der Hufschmied das Pferd zu behufen scheint.}
    \ex{\label{ex:modalverbenundhalbmodalverben250} dass der Hufschmied das Pferd zu behufen beschließt.}
  \end{xlist}
\end{exe}

\index{Kongruenz!Subjekt--Verb--}

Die Verbformen \textit{will}, \textit{scheint} und \textit{beschließt} kongruieren mit dem Subjekt \textit{der Hufschmied}, was leicht überprüft werden kann, wenn ein anderer Nominativ wie \textit{wir} oder \textit{du} eingesetzt wird.
Bezüglich der Kohärenz machen wir wieder nur den einfachen Umstellungstest.
Das Verb \textit{scheinen} konstruiert obligatorisch kohärent (die Nachstellung ist nicht möglich) und ist in diesem Test den Modalverben ähnlich, s.\ (\ref{ex:modalverbenundhalbmodalverben251}).
\index{Kohärenz}

\begin{exe}
  \ex\label{ex:modalverbenundhalbmodalverben251}
  \begin{xlist}
    \ex[*]{\label{ex:modalverbenundhalbmodalverben252} dass der Hufschmied will, das Pferd behufen}
    \ex[*]{\label{ex:modalverbenundhalbmodalverben253} dass der Hufschmied scheint, das Pferd zu behufen}
    \ex[ ]{\label{ex:modalverbenundhalbmodalverben254} dass der Hufschmied beschließt, das Pferd zu behufen}
  \end{xlist}
\end{exe}

Andere Tests zeigen allerdings eine andere Klassenbildung für diese Verbtypen.
Z.\,B.\ sind Frage-Antwort-Paare wie in (\ref{ex:modalverbenundhalbmodalverben255}) bei den Modalverben und Verben wie \textit{beschließen} möglich, nicht aber bei \textit{scheinen}.
Diese Frage-Antwort-Paare deuten auf einen Unterschied in der Rollenzuweisung hin.
Während Modalverben und Verben wie \textit{beschließen} eine Rolle an das Subjekt vergeben, vergibt \textit{scheinen} keine Rolle.
Das bedeutet, dass man nicht davon sprechen kann, dass es \textit{scheinen}-Situationen gibt, in denen ein Mitspieler die Rolle des Scheinenden spielt.

\begin{exe}
  \ex\label{ex:modalverbenundhalbmodalverben255}
  \begin{xlist}
    \ex[ ]{\label{ex:modalverbenundhalbmodalverben256} Frage: Wer will das Pferd behufen?\\
      Antwort: Der Hufschmied will das.}
    \ex[*]{\label{ex:modalverbenundhalbmodalverben257} Frage: Wer scheint das Pferd zu behufen?\\
      Antwort: Der Hufschmied scheint das.}
    \ex[ ]{\label{ex:modalverbenundhalbmodalverben258} Frage: Wer beschließt, das Pferd zu behufen?\\
      Antwort: Der Hufschmied beschließt das.}
  \end{xlist}
\end{exe}

\index{Rolle}\index{Subjekt}
Verben wie \textit{scheinen} verbinden sich mit dem lexikalischen Verb, und die Rek\-tions- und Rollen-Eigenschaften des lexikalischen Verbs bleiben vollständig unberührt.
Im Grunde sieht es so aus, dass in Sätzen wie (\ref{ex:modalverbenundhalbmodalverben256}) und (\ref{ex:modalverbenundhalbmodalverben258}) das Prinzip der Rollenzuweisung (Satz~\ref{satz:thetaprinz}) verletzt wird.
Sowohl \textit{behufen} als auch \textit{wollen} haben eine Rolle zu vergeben, das Gleiche gilt bei \textit{beschließen} und \textit{behufen}.
Ein anderer Test liefert weitere relevante Beobachtungstatsachen.
Es ist der Versuch, ein subjektloses Verb wie \textit{grauen} einzubetten (s.\ Abschnitt~\ref{sec:subjekte} und Abschnitt~\ref{sec:passiv}).

\begin{exe}
  \ex\label{ex:modalverbenundhalbmodalverben259}
  \begin{xlist}
    \ex[*]{\label{ex:modalverbenundhalbmodalverben260} Dem Hufschmied will grauen.}
    \ex[ ]{\label{ex:modalverbenundhalbmodalverben261} Dem Hufschmied scheint zu grauen}
    \ex[*]{\label{ex:modalverbenundhalbmodalverben262} Dem Hufschmied beschließt zu grauen.}
  \end{xlist}
\end{exe}

Während \textit{wollen} und \textit{beschließen} nicht funktionieren, wenn das eingebettete Verb kein Subjekt hat, ist dies für \textit{scheinen} kein Problem.
Das ist erneut ein Hinweis darauf, dass \textit{scheinen} keinerlei Anforderungen an die Valenzstruktur und an die Rollenvergabe der eingebetteten Verben stellt.
Es ergibt sich aus den empirischen Beobachtungen nun eine Dreiteilung in Modalverben wie \textit{wollen}, \textit{Halbmodalverben} bzw.\ \textit{Anhebungsverben} wie \textit{scheinen} und \textit{Kontrollverben} wie \textit{beschließen}.
Tabelle~\ref{tab:modalverbenundhalbmodalverben263} fasst die relevanten Eigenschaften zusammen, die jetzt noch genauer erläutert werden.

\begin{table}[!htbp]
  \resizebox{1\textwidth}{!}{
    \begin{tabular}{lcllll}
      \lsptoprule
      & \multirow{2}{*}{\textbf{Status}} & \multirow{2}{*}{\textbf{Kohärenz}} & \textbf{eigenes} & \textbf{Subjekts-} \\
      & & & \textbf{Subjekt} & \textbf{Rolle} & \textbf{Beispiel}\\
      \midrule
      \textbf{Modalverben} & 1 & obl.\ kohärent & ja & Identität & \textit{wollen} \\
      \textbf{Halbmodalverben} & 2 & obl.\ kohärent & nein & nein & \textit{scheinen} \\
      \textbf{Kontrollverben} & 2 & opt.\ inkohärent & ja & Kontrolle & \textit{beschließen} \\
      \lspbottomrule
    \end{tabular}
  }
  \caption{Modalverben, Halbmodalverben und Kontrollverben}
  \label{tab:modalverbenundhalbmodalverben263}
\end{table}

\index{Subjekt}

\index{Verb!Modal--}

Es dreht sich bei dieser Dreiteilung letztlich alles um die Subjekte und Rollen der regierenden und regierten Verben.
Die letzten beiden Spalten vor den Beispielen in Tabelle~\ref{tab:modalverbenundhalbmodalverben263} verdienen daher eine explizite Erklärung.
Die Modalverben verlangen, dass das regierte Verb ein Subjekt hat und identifizieren ihr eigenes Subjekt bei der Verbkomplexbildung mit dem des regierten Verbs.
Man kann dies als eine spezielle Vereinigung der Valenzlisten und der Rollenmuster des Modalverbs und des regierten lexikalischen Verbs betrachten, bei der die Subjektanforderung des Modalverbs und die des infiniten Verbs zu einer einzigen verschmelzen.
Dabei kann man auch die Ausnahme abbilden, dass scheinbar zwei Rollen an das Subjekt vergeben werden (vom Modalverb und vom lexikalischen Verb).

\index{Verb!Halbmodal--}

Die Halbmodalverben verlangen beim regierten Verb kein Subjekt und haben selber keins.
Sie nehmen bei der Verbkomplexbildung dem regierten Verb seine Valenz und sein Rollenmuster ab.
Ein Halbmodalverb verändert also die Valenzstruktur des lexikalischen Verbs gar nicht, sondern kopiert sie nur.

\index{Verb!Kontroll--}

Die Kontrollverben haben ein eigenes Subjekt, konstruieren optional inkohärent und verlangen, dass das regierte Verb eine eigene Subjektrolle hat.
Die Identität zwischen dem Subjekt des Kontrollverbs und dem nicht ausgedrückten Subjekt des regierten Verbs (\textit{zu}"=Infinitiv) kommt nicht über eine Vereinigung der Valenzlisten zustande, sondern über eine besondere Relation (die \textit{Kontrollrelation}), die im folgenden Abschnitt genauer besprochen wird.

\Zusammenfassung{%
In der kohärenten Konstruktion bildet ein regierendes Verb mit seinem regierten infiniten Verb einen Verbkomplex.
Bei der inkohärenten Konstruktion bildet das regierte Verb eine eigene VP.
Modalverben konstruieren obligatorisch kohärent und verschmelzen typischerweise ihre Subjektanforderung mit der des regierten Verbs.
Halbmodalverben (wie \textit{scheinen}) konstruieren obligatorisch kohärent und übernehmen die Kasus- und Rollenanforderung des regierten Verbs vollständig.
}

\section{Infinitivkontrolle}
\label{sec:infinitivkontrolle}

\index{Verb!Infinitiv!zu--}

Das in diesem Abschnitt diskutierte Phänomen betrifft sowohl die Subjekt- und Objektrelationen als auch den Bereich der Rektionsrelationen zwischen Verben.
Daher steht dieser Abschnitt am Ende dieses Kapitels.
Mehrfach (\zB in den Abschnitten~\ref{sec:ergaenzungssaetze} sowie~\ref{sec:kohaerenz} und~\ref{sec:modalverbenundhalbmodalverben}) wurde schon festgestellt, dass es Vorkommen des 2.~Status gibt, bei denen eine unabhängige VP im 2.~Status \zB die Subjekt- oder eine Objektstelle füllt.
Beispiele stehen in (\ref{ex:infinitivkontrolle264}).


\begin{exe}
  \ex\label{ex:infinitivkontrolle264}
  \begin{xlist}
    \ex{\label{ex:infinitivkontrolle265} [Das Geschirr zu spülen] nervt Matthias.}
    \ex{\label{ex:infinitivkontrolle266} Doro wagt, [die Küche zu betreten].}
  \end{xlist}
\end{exe}

\index{Subjekt!Infinitiv}
\index{Objekt!Infinitiv}
\index{Korrelat}

Diese Subjekt- und Objektinfinitive verhalten sich im Prinzip wie Subjekt- und Objektsätze.
Sie können genau wie diese mit Korrelat auftreten, wie (\ref{ex:infinitivkontrolle267}) zeigt.

\begin{exe}
  \ex\label{ex:infinitivkontrolle267}
  \begin{xlist}
    \ex{\label{ex:infinitivkontrolle268} Es nervt Matthias, [das Geschirr zu spülen].}
    \ex{\label{ex:infinitivkontrolle269} Doro wagt es, [die Küche zu betreten].}
  \end{xlist}
\end{exe}

Die VP [\textit{die Küche zu betreten}] selber hat nun offensichtlich kein ausgedrücktes Subjekt, was gut zum Fehlen der Kongruenzmerkmale bei \textit{zu betreten} passt.
Es muss daher geklärt werden, woher die Bedeutung des fehlenden Subjekts für das Verb \textit{betreten} in diesem und ähnlichen Fällen genommen wird.%
\footnote{In manchen Theorien wird ein unsichtbares Subjektpronomen PRO angenommen, das seine Bedeutung vollständig von einer anderen Konstituente im Satz kopiert.
Es ergeben sich dann Notationen wie [\textit{PRO die Küche zu betreten}].}
Einfach gefragt:
Was ist der Agens in der von (\ref{ex:infinitivkontrolle269}) beschriebenen \textit{betreten}-Situation?
Die betreffende Relation ist zwar im Kern semantisch, aber gleichzeitig stark durch die Grammatik konditioniert.
Im gegebenen Satz wird das Subjekt von \textit{wagen}, also \textit{Doro}, als Subjekt des Objektinfinitivs (\textit{betreten}) verstanden, und man würde daher von \textit{Subjektkontrolle} (des Objektinfinitivs) sprechen.
Subjektkontrolle heißt also immer Kontrolle durch ein Subjekt, und Paralleles gilt für \textit{Objektkontrolle}, also Kontrolle durch ein Objekt.
Dabei ist es unerheblich, ob das kontrollierende Element tatsächlich im Satz realisiert ist.
Das Passiv von \textit{versuchen} in (\ref{ex:infinitivkontrolle272}) zeigt dies.
Die PP \textit{vom Installateur} kontrolliert den Infinitiv auch dann, wenn sie weggelassen wird.

\begin{exe}
  \ex\label{ex:infinitivkontrolle270}
  \begin{xlist}
    \ex{\label{ex:infinitivkontrolle271} Der Installateur hat gestern versucht, die Küche zu betreten.}
    \ex{\label{ex:infinitivkontrolle272} Gestern wurde (vom Installateur) versucht, die Küche zu betreten.}
  \end{xlist}
\end{exe}

Die Definition von \textit{Infinitivkontrolle} (Definition~\ref{def:kontrolle}) erfasst jetzt bereits alle wesentlichen Fälle von Kontrolle, von denen einige dann im Folgenden erst bebeispielt werden.

\Definition{Infinitivkontrolle}{\label{def:kontrolle}%
Die \textit{Kontrollrelation} besteht zwischen einer nominalen Valenzstelle eines Verbs und einem von diesem Verb abhängigen (subjektlosen) \textit{zu}"=Infinitiv.
Die Bedeutung des nicht ausgedrückten Subjekts des abhängigen \textit{zu}"=Infinitivs wird dabei durch die mit der nominalen Valenzstelle verbundene Bedeutung beigesteuert.
\index{Kontrolle}
}

Der abhängige \textit{zu}-Infinitiv kann wie in (\ref{ex:infinitivkontrolle266}) an der Stelle eines Akkusativs stehen, aber auch an der Stelle eines Nominativs wie in (\ref{ex:infinitivkontrolle265}), außerdem als Angabe wie in (\ref{ex:infinitivkontrolle273}).
In diesem Satz kontrolliert \textit{Matthias} als das Subjekt von \textit{spülen} das nicht ausgedrückte Subjekt von \textit{danken}.
Derjenige, der dankt, kann in diesem Satz nur Matthias sein.

\begin{exe}
  \ex{\label{ex:infinitivkontrolle273} Matthias spült das Geschirr, um den beiden zu danken.}
\end{exe}

In erster Näherung liegt es im Fall der regierten \textit{zu}-Infinitive am regierenden Verb und seiner Valenz, von welcher Ergänzung die Kontrolle ausgeht.
Dabei gibt es deutlich präferierte Muster, auf die wir uns hier in der Beschreibung beschränken.
Wenn ein Subjektinfinitiv vorliegt, ist das kontrollierende Element meist das Objekt, egal ob es ein Akkusativ oder ein Dativ ist.
(\ref{ex:infinitivkontrolle274}) fasst die wichtigen Fälle zusammen.

\begin{exe}
  \ex\label{ex:infinitivkontrolle274}
  \begin{xlist}
    \ex{\label{ex:infinitivkontrolle275} Das Geschirr zu spülen, nervt ihn.}
    \ex{\label{ex:infinitivkontrolle276} Das Geschirr zu spülen, fällt ihm leicht.}
    \ex{\label{ex:infinitivkontrolle277} Das Geschirr zu spülen, beschert ihm einen zufriedenen Mitbewohner.}
    \ex{\label{ex:infinitivkontrolle278} Sich für Hilfe zu bedanken, freut ihn immer besonders.}
  \end{xlist}
\end{exe}

In (\ref{ex:infinitivkontrolle275}) bis (\ref{ex:infinitivkontrolle277}) liegt immer Objektkontrolle des Subjektinfinitivs vor, und zwar durch den Akkusativ eines zweiwertigen Verbs (\ref{ex:infinitivkontrolle275}), den Dativ eines zweiwertigen Verbs (\ref{ex:infinitivkontrolle276}) und den Dativ eines dreiwertigen Verbs (\ref{ex:infinitivkontrolle277}).
Zu beachten ist dabei in (\ref{ex:infinitivkontrolle277}), dass bei Vorliegen eines Akkusativs und Dativs der Dativ den Vorzug als kontrollierendes Element erhält.
In (\ref{ex:infinitivkontrolle278}) gibt es für manche Sprecher eine Lesart, in der sogenannte \textit{arbiträre Kontrolle} vorliegt.
Dabei kontrolliert keine Ergänzung des einbettenden Verbs den Infinitiv, sondern der Infinitiv wird unpersönlich oder allgemein verstanden, so als stünde das unpersönliche Subjektpronomen \textit{man}.
Der Infinitiv hätte dann die Bedeutung von \textit{dass jemand sich für Hilfe bedankt}.

Beim Objektinfinitiv wie in (\ref{ex:infinitivkontrolle280}) ist Subjektkontrolle zu erwarten, wenn sonst keine Ergänzungen vorliegen.
Wenn aber ein weiterer Akkusativ (\ref{ex:infinitivkontrolle281}) oder Dativ (\ref{ex:infinitivkontrolle282}) vorkommen, stehen im Prinzip das Subjekt und das andere Objekt als kontrollierende Elemente zur Verfügung.
Das Objekt erhält dabei regelmäßig den Vorzug als kontrollierendes Element, sowohl der Akkusativ in (\ref{ex:infinitivkontrolle281}) als auch der Dativ in (\ref{ex:infinitivkontrolle282}).

\begin{exe}
  \ex\label{ex:infinitivkontrolle279}
  \begin{xlist}
    \ex{\label{ex:infinitivkontrolle280} Er wagt, die Küche zu betreten.}
    \ex{\label{ex:infinitivkontrolle281} Er bittet seinen Mitbewohner, das Geschirr zu spülen.}
    \ex{\label{ex:infinitivkontrolle282} Doro erlaubt Matthias, sich den Wagen zu leihen.}
  \end{xlist}
\end{exe}

Wenn der Infinitiv als Angabe (mit \textit{anstatt}, \textit{ohne}, \textit{um} usw.) vorkommt, liegt fast immer Subjektkontrolle vor wie in (\ref{ex:infinitivkontrolle284})--(\ref{ex:infinitivkontrolle287}).

\begin{exe}
  \ex\label{ex:infinitivkontrolle283}
  \begin{xlist}
    \ex{\label{ex:infinitivkontrolle284} Matthias arbeitet, um Geld zu verdienen.}
    \ex{\label{ex:infinitivkontrolle285} Matthias begrüßt Doro, ohne aus der Rolle zu fallen.}
    \ex{\label{ex:infinitivkontrolle286} Matthias hilft Doro, anstatt untätig daneben zu stehen.}
    \ex{\label{ex:infinitivkontrolle287} Matthias bringt Doro den Wagen zurück, ohne den Lackschaden \\zu erwähnen.}
  \end{xlist}
\end{exe}

\Zusammenfassung{%
Bei Kontrollverben wird die Bedeutung des nicht ausgedrückten Subjekts der eingebetteten VP von anderen nominalen Valenzstellen des regierenden Verbs beigesteuert.
Ob das Subjekt oder ein Objekt die Kontrolle ausübt, hängt hauptsächlich vom Verb(typ) des Kontrollverbs ab.
}

\Uebungen

\Uebung{relationenundpraedikate01} \label{exc:relationenundpraedikate01} Was sind die Subjekte und Objekte in den folgenden Sätzen (nur Matrixsätze)?
Differenzieren Sie bei den Objekten nach Akkusativ-, Dativ- und Präpositionalobjekt.

\begin{enumerate}
  \item Mausi schickt den Brief an ihre Mutter.
  \item Es kann nicht sein, dass der Brief nicht angekommen ist.
  \item Der Grammatiker glaubt, dass die Modalverben eine gut definierbare Klasse sind.
  \item Den Eisschrank zu plündern, ist eine gute Idee.
  \item Wen jemand bewundert, bewundert, wer die Bewunderung empfindet.
  \item Ich werfe den Dart in ein Triple-Feld.
  \item Es dürstet die durstigen Rottweiler.
  \item Der immer die dummen Fragen gestellt hat, fragte Matthias, ob das wirklich Musik sein soll.
  \item Vor dem Hund muss man niemanden retten.
  \item Es verschwindet spurlos im Nebel.
\end{enumerate}


\Uebung{relationenundpraedikate02} \label{exc:relationenundpraedikate02} Bestimmen Sie die Typen der folgenden Verben gemäß Tabelle~\ref{tab:werdenpassivundverbtypen131} auf Seite~\pageref{tab:werdenpassivundverbtypen131} als unergatives, unakkusatives oder transitives Verb bzw.\ unergatives oder unakkusatives Dativverb oder ditransitives Verb.
Ziehen Sie außerdem die präpositional zwei- und dreiwertigen Verben gemäß Seite~\pageref{abs:valenz093} hinzu.

\begin{enumerate}
  \item kreischen
  \item schenken
  \item nützen
  \item trocknen
  \item kosten (in der Bedeutung von \textit{Kosten verursachen})
  \item antworten
  \item arbeiten
  \item bedürfen
  \item blitzen
  \item verzeihen
  \item abtrocknen
  \item überlaufen
  \item fallen
  \item verschieben
  \item schwindeln (in der Bedeutung von \textit{Schwindel spüren})
\end{enumerate}


\Uebung[\tristar]{relationenundpraedikate03} \label{exc:relationenundpraedikate03} Überlegen Sie, wie die Valenz und das Rollenmuster von Verben wie \textit{wiegen} und \textit{wundern} in Sätzen wie (\ref{ex:infinitivkontrolle288}) ist.
Integrieren Sie die Verben in die hier vorgestellten Verbtypen, auch bezüglich ihrer Passivierbarkeit.

\begin{exe}
  \ex\label{ex:infinitivkontrolle288}
  \begin{xlist}
    \ex{Dieser Kuchen wiegt einen Zentner.}
    \ex{Ihr Fehlverhalten wundert mich.}
  \end{xlist}
\end{exe}

\Uebung{relationenundpraedikate04} \label{exc:relationenundpraedikate04} Klassifizieren Sie die Dative als Nutznießerdative, Pertinenzdative, Bewertungsdative oder andere Dative.

\begin{enumerate}
  \item Ariel spielt mir die Gavotte.
  \item Dem Verein genügt ein zweiter Platz nicht.
  \item Die Tochter folgt ihrer Mutter nach Schweden.
  \item Dem Grammatiker ist dieser Text zu naiv.
  \item Fass unserem Hund bitte nicht an die Nase.
  \item Du gibst der Oma dem Hund zu viel Nassfutter.
\end{enumerate}

\Uebung[\tristar]{relationenundpraedikate05} \label{exc:relationenundpraedikate05} Suchen Sie ein Argument gegen die Annahme, dass man Fälle wie (\ref{ex:infinitivkontrolle289}) aus \citet[299]{Eisenberg2013b} zu den Pertinenzdativen rechnen sollte.

\begin{exe}
    \ex{\label{ex:infinitivkontrolle289} Man nimmt ihnen den Vater.}
\end{exe}

\Uebung{relationenundpraedikate06} \label{exc:relationenundpraedikate06} Wenden Sie den Test auf Regiertheit auf die eingeklammerten PPs an.
Bewerten Sie die Ergebnisse.

\begin{enumerate}
  \item Matthias interessiert sich [für elektronische Musik].
  \item Die Band spielt [nach den \textit{Verschwundenen Pralinen}].
  \item Matthias spielt [für eine Jazzband].
  \item Der Rottweiler bewahrte Marina [vor der Langeweile].
  \item Doro fragt [nach den verschwundenen Pralinen].
  \item Ich bat sie [um einen Rat].
  \item Ihr Rottweiler baute sich [vor dem Schrank mit dem Hundefutter] auf.
\end{enumerate}

\Uebung[\tristar]{relationenundpraedikate07} \label{exc:relationenundpraedikate07} Was fällt angesichts der Wortstellung innerhalb des Verbkomplexes in (\ref{ex:infinitivkontrolle290}) auf?

\begin{exe}
  \ex{\label{ex:infinitivkontrolle290} Ich weiß, dass der Kollege das Buch wird lesen müssen.}
\end{exe}

\Uebung{relationenundpraedikate08} \label{exc:relationenundpraedikate08} Welche von den im Nebensatz eingebetteten finiten Verben konstruieren immer kohärent?
Schreiben Sie zum Testen explizit den Satz gemäß dem Rechtsversetzungstest hin.

\begin{enumerate}
  \item Ich glaube, dass Michelle Marina den Hund zu verstehen hilft.
  \item Ich glaube, dass Michelle neues Hundefutter holen fährt.
  \item Ich glaube, dass Michelle den Hund in Verwahrung zu nehmen verspricht.
  \item Ich glaube, dass Michelle sehr gut mit Hunden umgehen kann.
  \item Ich glaube, dass Michelle den Hund spielen sieht.
  \item Ich glaube, dass Michelle den Hund gut zu erziehen versucht.
\end{enumerate}

\Uebung{relationenundpraedikate09} \label{exc:relationenundpraedikate09} Bestimmen Sie das kontrollierende Element der \textit{zu}-Infinitive.
Benennen Sie auch dessen grammatische Funktion (Subjekt, Akkusativ- oder Dativobjekt).

\begin{enumerate}
  \item Den Wagen zu waschen, scheint Matthias Spaß zu machen.
  \item Es versprach Matthias einen anstrengenden Nachmittag, das ganze Geschirr spülen zu müssen.
  \item Matthias bittet Doro, ihm den Wagen zu leihen.
  \item Doro lädt Matthias in den Klub ein, um abzutanzen.
  \item Es ist eine gute Idee von Matthias, den Wagen für Ralf-Erec zur Inspektion zu fahren.
\end{enumerate}

\Uebung{relationenundpraedikate10} \label{exc:relationenundpraedikate10} Welche der folgenden Sätze sind in der gegebenen Indexierung aufgrund der Bindungstheorie ungrammatisch?

\begin{enumerate}
  \item Michelle$_{\textrm{1}}$ freut sich$_{\textrm{2}}$ auf nächste Woche.
  \item Michelle$_{\textrm{1}}$ gibt ihr$_{\textrm{2}}$ die CD.
  \item Marina$_{\textrm{1}}$ freut sich$_{\textrm{1}}$, dass sie$_{\textrm{2}}$ sich$_{\textrm{2}}$ die CDs gekauft hat.
  \item Marina$_{\textrm{1}}$ freut sich$_{\textrm{1}}$, dass sie$_{\textrm{2}}$ sich$_{\textrm{1}}$ die CDs gekauft hat.
  \item Michelle$_{\textrm{1}}$ will ihr$_{\textrm{1}}$ die CDs schenken.
  \item Marina$_{\textrm{1}}$ hat ihre$_{\textrm{2}}$ CDs gehört.
  \item Marina$_{\textrm{1}}$ hat ihre$_{\textrm{1}}$ CDs gehört.
  \item Michelle$_{\textrm{1}}$ weiß, dass Marina$_{\textrm{2}}$ sich mit ihrem$_{\textrm{2}}$ Rottweiler angefreundet hat, der ihr$_{\textrm{1}}$ zuerst große Angst eingeflößt hatte.
  \item Michelle$_{\textrm{1}}$ weiß, dass Marina$_{\textrm{1}}$ sich mit ihrem$_{\textrm{2}}$ Rottweiler angefreundet hat, der ihr$_{\textrm{1}}$ zuerst große Angst eingeflößt hatte.
\end{enumerate}

\Uebung[\tristar]{} \label{exc:relationenundpraedikate11} Diskutieren Sie anhand von Verben wie \textit{gedenken}, \textit{bedürfen}, \textit{anklagen} und \textit{verdächtigen} die Rolle von Genitiv-Ergänzungen bei der Passivierung.

\Uebung[\tristar]{} \label{exc:relationenundpraedikate12} Diskutieren Sie die Verwendung von \textit{es} in den Beispielen in (\ref{ex:infinitivkontrolle291}), die \cite[51]{Mueller2008} aus existierender Literatur zusammengestellt hat.
Hinweis: Bestimmen Sie zunächst den Kasus und die Art der Rollenzuweisung.
Welche \textit{es} verhalten sich ähnlich, was ist anders?

\begin{exe}
  \ex\label{ex:infinitivkontrolle291}
  \begin{xlist}
    \ex{\label{ex:infinitivkontrolle292} Er hat es weit gebracht.}
    \ex{\label{ex:infinitivkontrolle293} Ich habe es heute eilig.}
    \ex{\label{ex:infinitivkontrolle294} Sie hat es ihm angetan.}
    \ex{\label{ex:infinitivkontrolle295} Er hat es auf sie abgesehen.}
    \ex{\label{ex:infinitivkontrolle296} Ich meine es gut mit dir.}
  \end{xlist}
\end{exe}

