\chapter{Passiv}
\label{sec:Passiv}

\section{Semantische Rollen}
\label{sec:semantischerollen}

\subsection{Verbsemantik und Rollen}
\label{sec:verbsemantikundrollen}

In den folgenden Abschnitten wird es immer wieder nötig sein, auf die Bedeutung von Verben bezugzunehmen.
Es wurde zwar in Abschnitt~\ref{sec:kasus} abgelehnt, Kasus an sich eine Bedeutung zuzusprechen (besonders für den Nominativ und den Akkusativ, eingeschränkt für den Dativ und Genitiv), aber bestimmte Muster von Kasusverteilungen bei verschiedenen Typen von Verben lassen sich besser verstehen, wenn man ein System zugrundelegt, nach dem die Verbbedeutung die Wahl verschiedener Kasus beeinflusst.
Es ändert sich also nichts daran, dass Kasus an sich keine Bedeutung hat, sondern wir systematisieren nur das Verhältnis von Verbbedeutung und Kasusmustern.

Dazu wird ein System von sogenannten \textit{semantischen Rollen} zugrundegelegt, die manchmal auch \textit{thematische Rollen} oder \textit{Theta-Rollen} bzw.\ $\theta$\textit{-Rollen} genannt werden.
Man kann Verben so verstehen, dass sie ein Ereignis (\zB \textit{kaufen}) oder einen Zustand (\zB \textit{liegen}) bezeichnen, wobei für Ereignisse und Zustände mit einem Sammelbegriff von \textit{Situationen} gesprochen wird.
In einer von einem Verb beschriebenen Situation gibt es typischerweise beteiligte Gegenstände i.\,w.\,S.\ (wie das Gekaufte bei \textit{kaufen}).
Diese Gegenstände spielen eine typische \textit{Rolle} in den Situationen, wie sie durch das Verb beschrieben werden.
Diese Rolle kann man semantisch spezifizieren.
Der Käufer in einer \textit{kaufen}-Situation handelt \zB aktiv und willentlich, das Gekaufte handelt nicht auf diese Weise, aber es wechselt den Besitzer im Rahmen der Situation.
Allgemein soll Definition~\ref{def:semrolle} gelten.

\Definition{Semantische Rolle}{\label{def:semrolle}%
Eine\textit{ semantische Rolle} ist die charakteristische Rolle, die ein beteiligter Gegenstand (Mitspieler) in einer von einem Verb beschriebenen Situation spielt.
Mitspieler können konkrete oder abstrakte Gegenstände (einschließlich Lebewesen und Menschen), andere Situationen usw.\ sein.
\index{Rolle}
\index{Mitspieler}
}

Typischerweise abstrahiert man von für einzelne Verben spezifischen Rollen (wie \textit{Käufer} und \textit{Gekauftes}) und entwickelt ein reduziertes Inventar von semantischen Rollen, die mit grammatischen Phänomenen in Verbindung stehen.
Wie viele und welche dies konkret sind, wird unterschiedlich gesehen.
In fast allen Ansätzen gibt es die Rollen \textit{Agens} (den \textit{Handelnden}), \textit{Patiens} (den \textit{Erdulder}) und \textit{Experiencer} (den bewusst \textit{Erlebenden}).
Für die hier besprochenen Phänomene reicht es, zwischen Agens, Experiencer und allen anderen Rollen zu unterscheiden.

Was ein Agens ist, haben wir im Grunde schon illustriert.
Die Sätze in (\ref{ex:verbsemantikundrollen001}) zeigen die hier vertretene Dreiteilung.

\begin{exe}
  \ex\label{ex:verbsemantikundrollen001}
  \begin{xlist}
    \ex{\label{ex:verbsemantikundrollen002} Michelle kauft einen Rottweiler.}
    \ex{\label{ex:verbsemantikundrollen003} Der Rottweiler schläft.}
    \ex{\label{ex:verbsemantikundrollen004} Der Rottweiler erfreut Marina.}
  \end{xlist}
\end{exe}

In der Bedeutung von (\ref{ex:verbsemantikundrollen002}) gibt es eine \textit{kaufen}-Situation.
Dabei ist Michelle der willentlich handelnde Mitspieler, also ein Agens.
Der Rottweiler handelt nicht, und es wird kein besonderer psychischer Zustand in ihm ausgelöst, womit er weder ein Agens noch ein Experiencer ist.
Hierbei ist eine wichtige Einschränkung zu beachten: Es wird sicherlich irgendeinen psychischen Zustand beim Hund und bei Michelle auslösen, an dem \textit{kaufen}-Ereignis beteiligt zu sein (\zB Freude bei Michelle und anfängliche Skepsis bei dem Rottweiler), aber die Bedeutung von \textit{kaufen} enthält keine spezifische Beschreibung solcher Zustände bei den Mitspielern.
Es geht bei der Rollenvergabe also nur um den Beitrag der Semantik des konkreten Verbs.
Was wir sonst noch bei der Interpretation eines Satzes wissen oder inferieren können, hat nichts mit den semantischen Rollen zu tun.

Die Mitspieler einer \textit{schlafen}-Situation wie in Satz (\ref{ex:verbsemantikundrollen003}) unterscheiden sich deutlich von denen einer \textit{kaufen}-Situation.
Es gibt nur ein beteiligtes Objekt, das weder Agens noch Patiens ist, hier der Rottweiler.
In (\ref{ex:verbsemantikundrollen004}) ist Marina ein Experiencer, denn das Verb \textit{erfreuen} bezeichnet Situationen, in denen ein ganz spezifischer psychischer Zustand (der der Freude) ausgelöst wird.
Ob der Rottweiler hier ein Agens ist, ist schwieriger zu beurteilen, da nicht ganz klar ist, ob er durch seine schiere Anwesenheit oder durch sein Verhalten erfreut.
Selbst wenn es sein Verhalten wäre, wäre fraglich, ob man bei einem Hund von \textit{willentlichem Handeln} sprechen könnte.
In Abschnitt~\ref{sec:passiv} wird sich eine Lösung für dieses Problem abzeichnen, die gleichzeitig neue Probleme mit sich bringt (vgl.\ Vertiefung \ref{vert:agensprobleme} auf Seite~\pageref{vert:agensprobleme}).
Die Definitionen~\ref{def:agens} und~\ref{def:experiencer} fassen das bisher Gesagte zusammen.

\Definition{Agens}{\label{def:agens}%
Das \textit{Agens} ist der willentlich handelnde Mitspieler in einer von einem Verb bezeichneten Situation.
\index{Agens}
}

\Definition{Experiencer}{\label{def:experiencer}%
Ein \textit{Experiencer} ist ein Mitspieler in einer von einem Verb bezeichneten Situation, bei dem ein spezifischer psychischer Zustand ausgelöst wird.
\index{Experiencer}
}

Es ergeben sich nun bestimmte Muster von semantischen Rollen bei Verben, die wir als Liste angeben können.
Wir bezeichnen hier dabei alle anderen Rollen außer Agens und Experiencer mit dem Platzhalter \textit{Rx} (für \textit{Rolle x} im Sinn von \textit{beliebige andere Rolle}), weil ihre Differenzierung für unsere Zwecke nicht erforderlich ist.
In ausführlicheren Analysen stünde statt \textit{Rx} eine größere Anzahl konkreter anderer Rollen.
Damit hat \zB \textit{kaufen} ein Rollenmuster \Rollen{Agens, Rx}.
Für die bisher besprochenen Verben ergibt sich insgesamt (\ref{ex:verbsemantikundrollen005}).

\begin{exe}
  \ex\label{ex:verbsemantikundrollen005}
  \begin{xlist}
    \ex{\label{ex:verbsemantikundrollen006}\textit{kaufen}: \Rollen{Agens,~Rx} }
    \ex{\label{ex:verbsemantikundrollen007}\textit{schlafen}: \Rollen{Rx} }
    \ex{\label{ex:verbsemantikundrollen008}\textit{erfreuen}: \Rollen{Rx,~Experiencer}\\
      oder vielleicht \Rollen{Agens,~Experiencer} }
  \end{xlist}
\end{exe}

Die Rollenverteilungen sind bei allen Vorkommen dieser Verben dieselben.
Das Rollenmuster ist also eine lexikalische Eigenschaft der Verben, und man kann daher auch von \textit{Verbtypen} sprechen, die durch Rollenmuster definiert werden.
Ein Verb wie \textit{erschrecken} hat dann denselben Typ wie \textit{erfreuen}.
\textit{anheben} hat denselben Typ wie \textit{kaufen}, usw.
Es gibt aber eben kein \textit{kaufen}-Ereignis, bei dem das Gekaufte willentlich handelt, Rollen- oder Sprachspiele ausgenommen.%
\footnote{Solche Spiele leben gerade davon, dass die besprochenen Regularitäten auf kreative Weise gebrochen werden.}
Ein üblicherweise angenommenes Prinzip, das sich zum Beispiel in Abschnitt~\ref{sec:artenvonesimnominativ} als sehr nützlich erweisen wird, besagt dabei, dass ein Verb jede Rolle nur einmal vergeben kann, vgl.\ Satz~\ref{satz:thetaprinz}.

\Satz{Prinzip der Rollenzuweisung}{\label{satz:thetaprinz}%
Jedes Verb kann eine von ihm zu vergebende Rolle nur einmal (also nur an eine Konstituente) vergeben.
Nicht alle Rollen müssen in jedem Satz vergeben werden (\zB bei fakultativen Ergänzungen).
\index{Rolle}
}

Bisher wurde nur von Verben als Rollen-Zuweiser gesprochen.
Das ist ein bisschen zu eng gefasst, da auch lexikalisierte Gefüge wie \textit{zu denken geben}, Adjektive wie \textit{wütend} und Substantive wie \textit{Aufteilung} Rollen vergeben.
Obwohl der Prädikatsbegriff nicht leicht präzise zu definieren ist (s.\ Abschnitt~\ref{sec:praedikateundpraedikativekonstituenten}), kann man allgemeiner davon sprechen, dass Rollen von Prädikaten zugewiesen werden.
Unabhängig davon muss man annehmen, dass die Präpositionen in PP-Angaben den regierten NPs eine Rolle zuweisen, und dass der Kasus freier NP-Angaben direkter Ausdruck einer freien Rolle ist.\index{Angabe!präpositional}
In (\ref{ex:verbsemantikundrollen009}) wird \textit{dem Tisch} die Rolle (der Ort der Situation) von der Präposition \textit{unter} zugewiesen.
Die lokale PP \textit{unter dem Tisch} bringt ihre Rolle sozusagen ganz unabhängig vom Verb mit.

\begin{exe}
  \ex{\label{ex:verbsemantikundrollen009} Der Rottweiler schläft [unter dem Tisch].}
\end{exe}

\subsection{Semantische Rollen und Valenz}
\label{sec:semantischerollenundvalenz}

\index{Valenz}\index{Rolle}

Interessant ist für die Grammatik (wie soeben angedeutet) die Verknüpfung der von einem Verb zugewiesenen semantischen Rollen mit seiner Valenz.
In Abschnitt~\ref{sec:valenz} haben wir uns nicht ganz erfolgreich bemüht, Valenz ohne Bezug zur Semantik zu definieren.
Valenz ist laut Definition~\ref{def:ergang} und Definition~\ref{def:valenz} die Liste der von einer Einheit subklassenspezifisch lizenzierten anderen Einheiten.
Man kann auch versuchen, den Unterschied zwischen Ergänzungen und Angaben stärker an die Rollensemantik eines Verbs zu knüpfen.
Im Vorgriff auf die Abschnitte~\ref{sec:dativeundindirekteobjekte} und~\ref{sec:ppergaenzungenundppangaben} nehmen wir die Beispiele in (\ref{ex:semantischerollenundvalenz010}) und (\ref{ex:semantischerollenundvalenz013}).

\begin{exe}
  \ex\label{ex:semantischerollenundvalenz010}
  \begin{xlist}
    \ex{\label{ex:semantischerollenundvalenz011} Michelle schenkt [ihrer Freundin] die Hundeleine.}
    \ex{\label{ex:semantischerollenundvalenz012} Michelle fährt [ihrer Freundin] zu schnell.}
  \end{xlist}
  \ex\label{ex:semantischerollenundvalenz013}
  \begin{xlist}
    \ex{\label{ex:semantischerollenundvalenz014} Michelle denkt [an Marina].}
    \ex{\label{ex:semantischerollenundvalenz015} Michelle rennt [an die Tür].}
  \end{xlist}
\end{exe}

Beim Dativ zu \textit{schenken} in (\ref{ex:semantischerollenundvalenz011}) und bei der \textit{an}-PP zu \textit{denken} in (\ref{ex:semantischerollenundvalenz014}) würde man von Ergänzungen sprechen.\index{Ergänzung}\index{Angabe}
In (\ref{ex:semantischerollenundvalenz012}) und (\ref{ex:semantischerollenundvalenz015}) wird der Dativ bzw.\ die \textit{an}-PP aber eher als Angabe analysiert.
Bemerkenswert ist, dass der Unterschied zwischen Ergänzungen und Angaben hier mit einem Unterschied in der Rollenzuweisung einhergeht.
Die Rolle des Geschenk-Empfängers bei \textit{schenken}-Situationen, die immer dem Dativ-Mitspieler zugewiesen wird, wird durch das Verb eindeutig festgelegt.
Dasselbe gilt für den \textit{an}-PP-Mitspieler bei \textit{denken}-Situationen wie in (\ref{ex:semantischerollenundvalenz014}).
Der Dativ in (\ref{ex:semantischerollenundvalenz012}) hingegen bezeichnet jemanden, der die Situation einschätzt.\index{Dativ!frei}
Die Rolle des Einschätzers wird aber sicherlich nicht von \textit{fahren} zugewiesen, denn es ist nicht Teil der Bedeutung von \textit{fahren}, dass an \textit{fahren}-Situationen ein Mitspieler beteiligt ist, der die Geschwindigkeit beurteilt.
Genauso ist die Rolle des \textit{an}-PP-Mitspielers in (\ref{ex:semantischerollenundvalenz015}) nicht wie bei \textit{denken} durch \textit{rennen} festgelegt.
Die nicht im Valenzrahmen des Verbs verankerten Angaben haben also eine vom Verb unabhängige Rolle, was besonders für PPs typisch ist, aber eben auch bei nicht regierten Kasus auftritt.
Passend dazu verliert die \textit{an}-PP bei \textit{denken} ihre für die Präposition \textit{an} spezifische Rolle (Zielort), da sie eine Ergänzung ist und das Verb die Rollenzuweisung alleine steuert.
In weiteren Abschnitten werden diese Verhältnisse immer wieder aufs Tapet kommen.
Eine zweifelsfreie Trennung von Ergänzungen und Angaben wird damit zwar besser angenähert, bleibt praktisch aber schwierig.
In Abschnitt~\ref{sec:artenvonesimnominativ} werden wir überdies sehen, dass das Pronomen \textit{es} bei Verben wie \textit{regnen} als Ergänzung auftritt, ohne dass ihm eine Rolle zugewiesen wird.

\Zusammenfassung{%
Semantische Rollen klassifizieren die Mitspieler an einer durch ein Verb beschriebenen Situation, \zB nach Agentivität, aber es gibt keine vom Verbtyp unabhängige Beziehung zwischen Rollen und Kasus.
}

\section{Passiv}
\label{sec:passiv}

\subsection{\textit{werden}-Passiv und Verbtypen}
\label{sec:werdenpassivundverbtypen}

\index{Passiv!werden--}

Über das \textit{werden}-Passiv (das oft auch nur als das \textit{Passiv} schlechthin oder das \textit{Vorgangspassiv} bezeichnet wird) wurde in diesem Buch schon wiederkehrend gesprochen, \zB in Abschnitt~\ref{sec:genusverbi} oder im Kontext der Subjekts- und Objektsgenitive in Abschnitt~\ref{sec:rektionundvalenzindernp}.
Hier wird die Bildung noch einmal zusammengefasst und vertieft, vor allem indem die Unterklassifikation der Vollverben genauer herausgearbeitet wird.
Ausgangspunkt sind die Paare von Aktiv- und Passivsätzen in (\ref{ex:werdenpassivundverbtypen110}) bis (\ref{ex:werdenpassivundverbtypen125}).

\newcommand{\Lab}[1]{\ensuremath{_{\mathrm{#1}}}}
\begin{exe}
  \ex\label{ex:werdenpassivundverbtypen110}
  \begin{xlist}
    \ex[ ]{\label{ex:werdenpassivundverbtypen111} Johan wäscht den Wagen.}
    \ex[ ]{\label{ex:werdenpassivundverbtypen112} Der Wagen wird (von Johan) gewaschen.}
  \end{xlist}
  \ex\label{ex:werdenpassivundverbtypen113}
  \begin{xlist}
    \ex[ ]{\label{ex:werdenpassivundverbtypen114} Alma schenkt dem Schlossherrn den Roman.}
    \ex[ ]{\label{ex:werdenpassivundverbtypen115} Der Roman wird dem Schlossherrn (von Alma) geschenkt.}
  \end{xlist}
  \ex\label{ex:werdenpassivundverbtypen116}
  \begin{xlist}
    \ex[ ]{\label{ex:werdenpassivundverbtypen117} Johan bringt den Brief zur Post.}
    \ex[ ]{\label{ex:werdenpassivundverbtypen118} Der Brief wird (von Johan) zur Post gebracht.}
  \end{xlist}
  \ex\label{ex:werdenpassivundverbtypen119}
  \begin{xlist}
    \ex[ ]{\label{ex:werdenpassivundverbtypen120} Der Maler dankt den Fremden.}
    \ex[ ]{\label{ex:werdenpassivundverbtypen121} Den Fremden wird (vom Maler) gedankt.}
  \end{xlist}
  \ex\label{ex:werdenpassivundverbtypen122}
  \begin{xlist}
    \ex[ ]{\label{ex:werdenpassivundverbtypen123} Johan arbeitet hier immer montags.}
    \ex[ ]{\label{ex:werdenpassivundverbtypen124} Montags wird hier (von Johan) immer gearbeitet.}
  \end{xlist}
  \ex\label{ex:werdenpassivundverbtypen125}
  \begin{xlist}
    \ex[ ]{\label{ex:werdenpassivundverbtypen126} Der Ball platzt bei zu hohem Druck.}
    \ex[*]{\label{ex:werdenpassivundverbtypen127} Bei zu hohem Druck wird (vom Ball) geplatzt.}
  \end{xlist}
  \ex\label{ex:werdenpassivundverbtypen128}
  \begin{xlist}
    \ex[ ]{\label{ex:werdenpassivundverbtypen129} Der Rottweiler fällt Michelle auf.}
    \ex[*]{\label{ex:werdenpassivundverbtypen130} Michelle wird (von dem Rottweiler) aufgefallen.}
  \end{xlist}
\end{exe}

\index{Verb!transitiv}
\index{Passiv!unpersönlich}
\index{Valenz}

Die (b)-Sätze in (\ref{ex:werdenpassivundverbtypen110})--(\ref{ex:werdenpassivundverbtypen128}) sind jeweils Passivbildungen zu den Aktivsätzen in (a).
Außer dass das Vollverb im Partizip auftritt (\textit{wäscht}--\textit{gewaschen} usw.) und das Hilfsverb \textit{werden} als finites Verb hinzukommt, ergibt sich ein relativ klares Muster bezüglich der Valenzänderungen vom Aktivverb zum zugehörigen Passivverb.
Wie schon mehrfach erwähnt, wird der Nominativ des Aktivs entfernt, kann aber optional als PP mit \textit{von} formuliert werden.
Diese PP kann man als fakultative Ergänzung oder als Angabe analysieren.
Wir sagen tendentiell, dass es eine fakultative Ergänzung ist, weil sie spezifisch für eine formal klar abgrenzbare Klasse von Verben ist, nämlich die passivierten Verben.

Wenn das Verb im Aktiv einen Akkusativ hat (wie \textit{waschen}, \textit{schenken}, \textit{bringen}), wird dieser zum Nominativ des Passivs.
Falls das aber nicht so ist, ergeben sich im Passiv Sätze ohne Nominativ und damit ohne Subjekt wie (\ref{ex:werdenpassivundverbtypen121}), (\ref{ex:werdenpassivundverbtypen124}), die manchmal auch \textit{unpersönliches Passiv} genannt werden.
Alle anderen Ergänzungen bleiben unverändert, also hier der Dativ von \textit{schenken} (\textit{dem Schlossherrn}), die PP-Ergänzung bei \textit{bringen} (\textit{zur Post}) und der Dativ bei \textit{danken} (\textit{den Fremden}).
Diesen Valenzunterschied zwischen Aktiv und Passiv sehen wir bei allen Verben außer \textit{platzen} (\ref{ex:werdenpassivundverbtypen125}) und \textit{auf"|fallen} (\ref{ex:werdenpassivundverbtypen128}), die als einzige Verben in den Beispielen überhaupt keine Passivbildung zulassen.
Im Gegensatz zur naiven Aussage, nur transitive Verben (also solche mit einer Nominativ- und einer Akkusativ-Ergänzung) könnten ein Passiv bilden, können also \zB auch Verben, die nur einen Nominativ und einen Dativ regieren (\textit{danken}) passiviert werden.
Sogar bestimmte intransitive Verben wie \textit{arbeiten} können passiviert werden, andere allerdings nicht (\textit{platzen}).
Auch Verben mit PP-Ergänzungen (\textit{glauben an}) oder Genitiv (\textit{gedenken}), die hier aus Platzgründen weggelassen wurden, sind passivierbar.
Das \textit{werden}-Passiv betrifft also vor allem den Nominativ und nur nachrangig den Akkusativ.

Es bleibt die Frage, warum Verben wie \textit{platzen} und \textit{auf"|fallen} nicht passivierbar sind, obwohl sie dasselbe Valenzmuster haben wie \textit{arbeiten} bzw.\ \textit{danken}.
Die Antwort lässt sich auf die Rollenverteilung bei diesen Verben zurückführen.
Bei passivierbaren Verben wird dem Nominativ des Aktivs prototypisch vom Verb eine Agens"=Rolle zugewiesen.\index{Agens}
In \textit{platzen}- und \textit{auf"|fallen}-Situationen gibt es aber keinen willentlich Handelnden, die entsprechenden Situationen sind vielmehr unwillkürliche Widerfahrnisse.
Es gilt also Satz~\ref{satz:werdenpass}, der im Grunde informell eine Lexikonregel beschreibt (vgl.\ Abschnitt~\ref{sec:lexikonregeln}).
Eine solche Lexikonregel würde dabei nicht nur Verben mit veränderter Valenzstruktur erzeugen, sondern wäre auch für morphologische Wortformenbildung zuständig.


\index{Lexikon!Regel}

\Satz{\textit{werden}-Passiv}{\label{satz:werdenpass}%
Das \textit{werden}"=Passiv kann prototypisch von Verben mit einem agentiven Nominativ durch Veränderungen der Valenz des Verbs (\zB durch eine Lexikonregel) gebildet werden.
Die Nominativ-Ergänzung des Aktivs wird zu einer fakultativen \textit{von}"=PP des Passivs.
Falls der Aktiv einen Akkusativ hat, wird er zum Nominativ des Passivs.
\index{Passiv!als Valenzänderung}
\index{Passiv!werden--}
}

\index{Verb!unergativ}
\index{Verb!unakkusativ}
\index{Verb!intransitiv}

Es gibt nach dieser Darstellung zwei Arten von sogenannten intransitiven Verben, nämlich solche, die einen agentiven Nominativ haben (sogenannte \textit{unergative} Verben) wie \textit{arbeiten} und solche, die einen nicht-agentiven Nominativ haben (sogenannte \textit{unakkusative} Verben) wie \textit{platzen}.%
\footnote{Statt von \textit{unergativen} und \textit{unakkusativen Verben} sprechen manche von \textit{unergativischen} und \textit{unakkusativischen Verben}.}
Als Test bietet sich die Passivierbarkeit an (nur unergative intransitive Verben sind passivierbar).
Damit ergibt sich eine Klassifikation für die hier besprochenen Verbtypen wie in Tabelle~\ref{tab:werdenpassivundverbtypen131}, wobei agentive Nominative als Nom\_Ag gekennzeichnet sind.
Dass Satz~\ref{satz:werdenpass} das Wort \textit{prototypisch} enthält, liegt daran, dass die Generalisierung nur ungefähr zutrifft.
Vertiefung~\ref{vert:agensprobleme} bespricht eins der Probleme.

\begin{table}[!htbp]
    \begin{tabular}{lllll}
      \lsptoprule
      \textbf{Valenz} & \textbf{Passiv} & \textbf{Name} & \textbf{Beispiel} \\
      \midrule
      Nom\_Ag & ja & Unergative & \textit{arbeiten} \\
      Nom & nein & Unakkusative & \textit{platzen} \\
      Nom\_Ag, Akk & ja & Transitive & \textit{waschen} \\
      Nom\_Ag, Dat & ja & unergative Dativverben & \textit{danken} \\
      Nom, Dat & nein & unakkusative Dativverben & \textit{auf"|fallen} \\
      Nom\_Ag, Dat, Akk & ja & Ditransitive & \textit{geben} \\
      \lspbottomrule
    \end{tabular}
  \caption{Typen von Vollverben nach Valenz und Agentivität}
  \label{tab:werdenpassivundverbtypen131}
\end{table}


\begin{Vertiefung}{Probleme der Agens-Definition}

  \label{vert:agensprobleme}
  \index{Agens}

\noindent Die hier verwendeten Definitionen für das Agens und für das \textit{werden}-Passiv erlauben es, die Passivierbarkeit eines Verbs als Zeichen dafür zu interpretieren, dass das Verb ein agentives Subjekt hat.
Wir können damit versuchen, zu entscheiden, ob das Subjekt bei \textit{ängstigen} tatsächlich agentiv ist, s.\ (\ref{ex:werdenpassivundverbtypen132}).
Gleichzeitig tritt aber ein neues Problem auf, das in (\ref{ex:werdenpassivundverbtypen135}) zu beobachten ist.

\begin{exe}
  \ex\label{ex:werdenpassivundverbtypen132}
  \begin{xlist}
    \ex[ ]{\label{ex:werdenpassivundverbtypen133} Der Rottweiler ängstigt Marina.}
    \ex[*]{\label{ex:werdenpassivundverbtypen134} Marina wird von dem Rottweiler geängstigt.}
  \end{xlist}
  \ex\label{ex:werdenpassivundverbtypen135}
  \begin{xlist}
    \ex[ ]{\label{ex:werdenpassivundverbtypen136} Eine Wolke überholt den Pteranodon.}
    \ex[ ]{\label{ex:werdenpassivundverbtypen137} Der Pteranodon wird von einer Wolke überholt.}
  \end{xlist}
\end{exe}

(\ref{ex:werdenpassivundverbtypen134}) legt nahe, dass man das Subjekt von \textit{ängstigen} eher nicht als Agens klassifizieren sollte, weil das Verb schlecht passivierbar ist.
Allerdings lässt sich \textit{überholen} mit dem Subjekt \textit{eine Wolke}, das ganz sicher kein willentlich handelndes Wesen bezeichnet, passivieren.
Satz (\ref{ex:werdenpassivundverbtypen137}) ist einwandfrei grammatisch.
Eine Lösung für dieses Problem zu erarbeiten, würde den hier gegebenen Rahmen sprengen.
Sie kann nur darin liegen, den Begriff des Agens nicht über eine einzige Eigenschaft \textit{willentlich handelnd} kategorisch zu definieren.
Es wird dringend \citet{Dowty1991} zur Lektüre empfohlen.

\end{Vertiefung}

\subsection{\textit{bekommen}-Passiv}
\label{sec:bekommenpassiv}

Das \textit{werden}-Passiv ist nicht die Passivbildung schlechthin im Deutschen.
Verschiedene andere Bildungen sind im Prinzip auch Passive, unter anderem das \textit{bekommen}-Passiv in (\ref{ex:bekommenpassiv138}).

\begin{exe}
  \ex\label{ex:bekommenpassiv138}
  \begin{xlist}
    \ex[ ]{\label{ex:bekommenpassiv139} Mein Kollege bekommt den Wagen (von Johan) gewaschen.}
    \ex[ ]{\label{ex:bekommenpassiv140} Der Schlossherr bekommt den Roman (von Alma) geschenkt.}
    \ex[ ]{\label{ex:bekommenpassiv141} Mein Kollege bekommt den Brief (von Johan) zur Post gebracht.}
    \ex[ ]{\label{ex:bekommenpassiv142} Die Fremden bekommen (von dem Maler) gedankt.}
    \ex[?]{\label{ex:bekommenpassiv143} Mein Kollege bekommt hier immer montags (von Johan) gearbeitet.}
    \ex[*]{\label{ex:bekommenpassiv144} Mein Kollege bekommt bei zu hohem Druck (von dem Ball) geplatzt.}
    \ex[*]{\label{ex:bekommenpassiv145} Michelle bekommt (von dem Rottweiler) aufgefallen.}
  \end{xlist}
\end{exe}

Hier treten mit kleinen Veränderungen die in Abschnitt~\ref{sec:werdenpassivundverbtypen} als Beispiele verwendeten Verben in einer Konstruktion mit dem Verb \textit{bekommen} im Partizip auf.
Wie beim \textit{werden}-Passiv wird der agentive Nominativ des Aktivs zur fakultativen \textit{von}-PP.
Dass \textit{platzen} (\ref{ex:bekommenpassiv144}) und \textit{auf"|fallen} (\ref{ex:bekommenpassiv145}) nicht passivierbar sind, folgt wie beim \textit{werden}-Passiv aus dem Fehlen eines agentiven Nominativs.
Allein diese Ähnlichkeit erklärt bereits, warum die Konstruktion auch \textit{Passiv} genannt wird.%
\footnote{Mit stilistischen Unterschieden können auch \textit{erhalten} und \textit{kriegen} statt \textit{bekommen} verwendet werden.}
Das Subjekt des \textit{bekommen}-Passivs ist aber im Gegensatz zum \textit{werden}-Passiv immer eine NP, die im Aktivsatz als Dativ auftritt.
Satz~\ref{satz:bekommenpass} formuliert die Charakteristika des \textit{bekommen}"=Passivs.
\index{Agens}

\Satz{\textit{bekommen}-Passiv}{\label{satz:bekommenpass}%
Das \textit{bekommen}"=Passiv kann von allen Verben mit einem agentiven Nominativ und einem regierten Dativ gebildet werden.
Die obligatorische Nominativ-Ergänzung des Aktivs wird zu einer fakultativen \textit{von}"=PP des Passivs.
Der Dativ des Aktivs wird zum Nominativ des Passivs.
\index{Passiv!als Valenzänderung}
\index{Passiv!bekommen--}
\index{Valenz}
}


Im Fall von \textit{schenken} (\ref{ex:bekommenpassiv140}) und \textit{danken} (\ref{ex:bekommenpassiv142}) ist dieser Dativ eindeutig bereits auf der Valenzstruktur des Verbs verankert.
Für die Sätze (\ref{ex:bekommenpassiv139}) und (\ref{ex:bekommenpassiv141}) muss man dann Aktivsätze wie in (\ref{ex:bekommenpassiv146}) zugrundelegen, die einen oft sogenannten \textit{freien Dativ} enthalten.
Der Status solcher Dative wird noch genauer in Abschnitt~\ref{sec:dativeundindirekteobjekte} diskutiert.

\begin{exe}
  \ex\label{ex:bekommenpassiv146}
  \begin{xlist}
    \ex{\label{ex:bekommenpassiv147} Johan wäscht meinem Kollegen den Wagen.}
    \ex{\label{ex:bekommenpassiv148} Johan bringt meinem Kollegen den Brief zur Post.}
  \end{xlist}
\end{exe}

Es ist bisher nicht ohne Weiteres ableitbar, warum (\ref{ex:bekommenpassiv143}) ungrammatisch oder zumindest fragwürdig sein soll.
Der entsprechende Aktivsatz (\ref{ex:bekommenpassiv149}) ist es aber auch.

\begin{exe}
  \ex[*]{\label{ex:bekommenpassiv149} Johan arbeitet meinem Kollegen hier immer montags.}
\end{exe}

\index{Verb!unergativ}
\index{Dativ!Bewertungs--}

Der Befund deutet darauf hin, dass unergative Verben nicht mit der Art Dativ kombinierbar sind, die man zur Bildung des \textit{bekommen}-Passivs benötigt.
Man formuliert hier eher eine \textit{für}-PP wie in (\ref{ex:bekommenpassiv150}).

\begin{exe}
  \ex{\label{ex:bekommenpassiv150} Johan arbeitet für meinen Kollegen hier immer montags.}
\end{exe}

Die Dative, die mit unergativen Verben kombinierbar sind, sind sogenannte \textit{Bewertungsdative} wie in (\ref{ex:bekommenpassiv151}), die auch bei anderen Verbtypen niemals ein \textit{bekommen}-Passiv erlauben, vgl.\ (\ref{ex:bekommenpassiv154}).


\begin{exe}
  \ex\label{ex:bekommenpassiv151}
  \begin{xlist}
    \ex[ ]{\label{ex:bekommenpassiv152} Alma singt meinem Kollegen zu laut.}
    \ex[*]{\label{ex:bekommenpassiv153} Mein Kollege bekommt von Alma zu laut gesungen.}
  \end{xlist}
  \ex\label{ex:bekommenpassiv154}
  \begin{xlist}
    \ex[ ]{\label{ex:bekommenpassiv155} Alma isst meinem Kollegen den Kuchen zu schnell.}
    \ex[*]{\label{ex:bekommenpassiv156} Mein Kollege bekommt den Kuchen von Alma zu schnell gegessen.}
  \end{xlist}
\end{exe}

An dieser Stelle können wir mit dem Wissen aus dem Abschnitt~\ref{sec:subjekte} und diesem Abschnitt die Tabelle mit den Eigenschaften der Kasus (Tabelle~\ref{tab:kasus030} von Seite~\pageref{tab:kasus030}) erweitern, s.\ Tabelle~\ref{tab:bekommenpassiv157}.
Auch die Verbkongruenz des Nominativs und die Beteiligung an den verschiedenen Passiven zeigen, dass die Kasus nicht alle funktional gleich sind, und dass die Annahme einer Hierarchie der Kasus gut begründbar ist.

\begin{table}[!htbp]
  \begin{tabular}{lp{0.1cm}llll}
    \lsptoprule
     \textbf{Eigenschaft} && \textbf{Nominativ} & \textbf{Akkusativ} & \textbf{Dativ} & \textbf{Genitiv} \\
    \hline
    verbregiert && fast immer & oft & oft & selten \\
    Verbkongruenz && ja & nein & nein & nein \\
    passivbeteiligt && ja & ja & ja & nein \\
    eigene Semantik && nein & fast nie & manchmal & manchmal \\
    attributiv && nein & nein & nein & ja \\
    präpositionsregiert && nie & oft & oft & oft \\
    \lspbottomrule
  \end{tabular}
  \caption{Eigenschaften der Kasus (erweitert)}
  \label{tab:bekommenpassiv157}
\end{table}

Zu den sogenannten \textit{Objekten} wird jetzt in Abschnitt~\ref{sec:objekteergaenzungenundangaben} noch mehr gesagt.
Insbesondere kommen wir bei den Dativen in Abschnitt~\ref{sec:dativeundindirekteobjekte} auf die hier zuletzt besprochenen Fälle nochmals zurück.

\Zusammenfassung{%
Passive mit \textit{werden} und \textit{bekommen} können prototypisch von Verben mit agentivem Nominativ gebildet werden, wobei dieser Nominativ zur optionalen \textit{von}-PP wird und entweder der Akkusativ (bei \textit{werden}) oder der Dativ (bei \textit{bekommen}) zum Nominativ des Passivs wird.
Es gibt zwei Arten von intransitiven Verben (einschließlich Nominativ-Dativ-Verben), nämlich Unakkusative mit nicht-agentivem Nominativ wie \textit{platzen} und Unergative mit agentivem Nominativ wie \textit{arbeiten}.
}

