\author{Roland Schäfer}
\title{Einführung in die grammatische Beschreibung des Deutschen}
\subtitle{Vierte Auflage}

\BackBody{Diese Einführung in die deutsche Grammatik unterscheidet sich von allen anderen Werken mit oberflächlich gesehen ähnlicher Zielsetzung dadurch, dass sie einerseits die Sprache an sich, nicht aber die Linguistik als ihren Gegenstand auffasst. Andererseits ist ihre Konzeption jedoch anspruchsvoll genug, dass die wesentlichen Generalisierungen, die linguistische Theorien abzubilden versuchen, nach ihrer Lektüre genausogut oder besser erfasst werden können. Das Buch setzt sich ausdrücklich keinen spezifischen theoretischen Rahmen, steht aber oberflächenorientierten und stark lexikalistischen Theorien nah. Während es ausdrücklich auch für Lehramtsstudiengänge und polyvalente Studiengänge konzipiert ist, wird eine vordergründige spezifische Ausrichtung auf das Lehramt ausdrücklich abgelehnt. Das Buch behandelt nach einer Einleitung, einer Positionierung zur Rolle der Grammatik im Deutschunterricht und einer Diskussion von Wortklassen die elemntaren Teilbereiche der Grammatik ab: Phonetik, Phonologie, Morphologie, Syntax und Grammatik.\\

\noindent\textbf{Roland Schäfer} ist Linguist. Nach einem Magisterstudium der Allgemeinen und Vergleichenden Sprachwissenschaft und Sprachwissenschaftlichen Japanologie an der Philipps-Universität Marburg, einer Promotion in Anglistischer Sprachwissenschaft an der Georg-August-Universität Göttingen und einer Habilitation zum Thema \textit{Probabilistic German Morphosyntax} an der Humboldt-Universität zu Berlin (Venia legendi für Allgemeine Sprachwissenschaft und Germanistische Linguistik) sowie wissenschaftlichen Tätigkeiten an der Georg-August-Universität Göttingen, der Freien Universität Berlin, der Universität Göteborg und der Humboldt-Universität zu Berlin nahm er im Jahr 2022 den Ruf auf die Professur für Germanistische Sprachwissenschaft mit dem Schwerpunkt Grammatik und Lexikon an der Friedrich-Schiller-Universität Jena an und lehrt dort seit dem Wintersemester 2022\slash 2023.
Aktuelle Forschungsvorhaben beschäftigen sich mit individueller grammatischer Variation, Graphematik und erkenntnistheoretischen Grundfragen der Linguistik.}

\BookDOI{10.5281/zenodo.1421660}
\renewcommand{\lsISBNdigital}{978-3-96110-116-0}
\renewcommand{\lsISBNhardcover}{978-3-96110-117-7}
\renewcommand{\lsISBNsoftcover}{978-3-96110-118-4}
\renewcommand{\lsISBNsoftcoverus}{978-1727793741}
\renewcommand{\lsSeries}{tbls}
\renewcommand{\lsSeriesNumber}{2}
\renewcommand{\lsURL}{http://langsci-press.org/catalog/book/224}


